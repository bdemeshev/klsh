\documentclass[pdftex,12pt,a4paper]{article}

\input{/home/boris/science/tex_general/title_bor_utf8}


\begin{document}
\parindent=0 pt % отступ равен 0

\textbf{Загоночная контрольная!}


\begin{enumerate}


\item Оркестр из трех музыкантов ($A$,~$B$,~$C$) играет в подземном переходе. Поодиночке они
могли бы заработать, соответственно, $6$, $18$ и $30$
рублей в час. Играя по двое, они бы получили: $A$~и~$B$~---~36, $A$~и~$C$~---~48, $B$~и~$C$~---~54.
А~вместе они имеют~$72$.


Используя вектор Шепли, определите, как им поделить деньги по справедливости.

\item Вася стреляет по мишени два раза. В первый раз он попадает с вероятностью $0.5$. Во второй раз он попадает с вероятностью $0.7$ из-за того, что пристреливается. 
\begin{enumerate}
\item Нарисуйте дерево возможного развития событий
\item Какова вероятность того, что Вася попадёт ровно один раз?
\item Какова вероятность того, что он попал во второй раз, если известно, что он попал ровно один раз?
\end{enumerate}

\end{enumerate}

Паниковать на контрольной строжайше запрещается :)

\vspace{40pt}

\textbf{Загоночная контрольная!}

\begin{enumerate}


\item Оркестр из трех музыкантов ($A$,~$B$,~$C$) играет в подземном переходе. Поодиночке они
могли бы заработать, соответственно, $6$, $18$ и $30$
рублей в час. Играя по двое, они бы получили: $A$~и~$B$~---~36, $A$~и~$C$~---~48, $B$~и~$C$~---~54.
А~вместе они имеют~$72$.


Используя вектор Шепли, определите, как им поделить деньги по справедливости.

\item Вася стреляет по мишени два раза. В первый раз он попадает с вероятностью $0.5$. Во второй раз он попадает с вероятностью $0.7$ из-за того, что пристреливается. 
\begin{enumerate}
\item Нарисуйте дерево возможного развития событий
\item Какова вероятность того, что Вася попадёт ровно один раз?
\item Какова вероятность того, что он попал во второй раз, если известно, что он попал ровно один раз?
\end{enumerate}

\end{enumerate}

Паниковать на контрольной строжайше запрещается :)

\vspace{40pt}

\textbf{Загоночная контрольная!}

\begin{enumerate}


\item Оркестр из трех музыкантов ($A$,~$B$,~$C$) играет в подземном переходе. Поодиночке они
могли бы заработать, соответственно, $6$, $18$ и $30$
рублей в час. Играя по двое, они бы получили: $A$~и~$B$~---~36, $A$~и~$C$~---~48, $B$~и~$C$~---~54.
А~вместе они имеют~$72$.


Используя вектор Шепли, определите, как им поделить деньги по справедливости.

\item Вася стреляет по мишени два раза. В первый раз он попадает с вероятностью $0.5$. Во второй раз он попадает с вероятностью $0.7$ из-за того, что пристреливается. 
\begin{enumerate}
\item Нарисуйте дерево возможного развития событий
\item Какова вероятность того, что Вася попадёт ровно один раз?
\item Какова вероятность того, что он попал во второй раз, если известно, что он попал ровно один раз?
\end{enumerate}

\end{enumerate}

Паниковать на контрольной строжайше запрещается :)


\end{document}