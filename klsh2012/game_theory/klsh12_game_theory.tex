\documentclass[pdftex,12pt,a4paper]{article}

\input{/home/boris/science/tex_general/title_bor_utf8}


\begin{document}
\parindent=0 pt % отступ равен 0

КЛШ-2012, Теория игр.

Кто был?
1 --- все.
2 --- все кроме Валерии Козыркиной
3 --- все кроме Юлии Пакаревой и Валерии Сас
4 --- все кроме Саши Серебрянниковой
5 --- забыл спросить...
6 --- ?
7 --- не было
8 --- все
9 --- ?
10 --- все кроме Светланы Калугиной
11 --- все кроме Светы Калугиной, Арсения Карпычева, Валерии Козыркиной

\section{Встреча 1. Вероятность и условная вероятность}

Теория вероятностей


Пример 1. Множество исходов. Мальчики и девочки. Класс. Выбираем одного человека наугад.


Разница: случайная величина и событие. Событие: наугад выбрали М. Случайные величины: класс в котором учится, возраст, рост, вес. 


Выбрали 10-классника. Событие или случайная величина?


Вопросы: $\P(M)$, $\P(F)$, $\P(F\cup A_{10})$, $\P(F\cap A_9)$, \ldots


Условная вероятность.

\begin{enumerate}
\item Интуитивно, $\P(A|B)$
\item Формальное, $\P(A|B)=\frac{\P(A\cap B)}{P(B)}$
\end{enumerate}


Пример 2. Маше подарят Розы (0.4), Ромашки (0.2), Лилии (0.2) или Гвоздики (0.1). Гвоздики ей не нравятся. Какова вероятность того, что ей понравится подарок?



Пример 3. Те же. Выбираем одного наугад --- и отправляем за маркером. Затем выбираем одного наугад и отправляем второго, $M_1$, $M_2$. Ищем все вероятности комбинированные и условные. 


Заметки:
\begin{enumerate}
\item Обратить внимание, что нужно аккуратно размещать числа: есть вероятности у веточки, есть --- на листиках
\item Нужно доступно объяснить, на примере, почему вероятности на траектории перемножаются
\item Брать маркеры двух цветов
\end{enumerate}


\section{Встреча 2. Решение задач по вероятностям}

Разбор задач. $\P(F_1|$вторым выгнали 10-классника$)$. Задачи про монетки, соц. опросы, успели дерево для Пети и Васи.


Заметки:
\begin{enumerate}
\item Брать маркеры двух цветов
\end{enumerate}

\begin{enumerate}
\item Злобным Вирусом заражен 1\% населения. Имеется тест, который ошибается в 5\% случаев. Если зараженный человек пройдет тест, то он будет признан здоровым с вероятностью 5\%. Если здоровый человек пройдет тест, то он будет признан больным с вероятностью 5\%. Рустам Байбурин прошел тест и судя по тесту он заражен Злобным Вирусом. Какова вероятность того, что он действительно заражен?



\item Два охотника выстрелили в одну утку. Первый попадает с
вероятностью 0,4, второй --- с вероятностью 0,6. В утку попала ровно
одна пуля. Какова вероятность того, что утка была убита первым
охотником?


\item У тети Маши --- двое детей, один старше другого. Предположим, что вероятности рождения мальчика и девочки равны и не зависят от дня недели, а пол первого и второго ребенка независимы. \\
\begin{enumerate}
\item Известно, что у тети Маши есть хотя бы один мальчик. Какова
вероятность того, что у тети Маши есть девочка? 
\item Тетя Маша наугад выбирает одного своего
ребенка и посылает к тете Оле, вернуть учебник по теории
вероятностей. Это оказывается мальчик. Какова вероятность того,
что другой ребенок --- девочка? 
\item Известно, что старший ребенок --- мальчик. Какова вероятность того, что другой ребенок --- девочка? 
\item На вопрос: <<А правда ли тетя Маша, что у вас есть сын, родившийся в пятницу?>>. Она ответила: <<Да>>. Какова вероятность того, что другой ребенок --- девочка?
\end{enumerate}


\item Самолет упал в горах, в степи или в море. Вероятности,
соответственно, равны $0,5$, $0,3$ и $0,2$. Если он упал в горах,
то при поиске его найдут с вероятностью $0,7$. В степи --- $0,8$, на
море --- $0,2$. Самолет искали в горах, в степи и не нашли. Какова
вероятность того, что он упал в море?


\item Четыре шкатулки, в одной из них --- приз. Игрок выбирает одну из шкатулок. Ведущий открывает равновероятно одну из пустых шкатулок, кроме указанной игроком. У игрока есть возможность поменять свой выбор. Ведущий открывает равновероятно еще одну из пустых шкатулок, кроме указанной игроком. У игрока снова есть возможность поменять свой выбор. Какова оптимальная стратегия?


\item Имеется три монетки. Две <<правильных>> и одна "--- с орлами по
обеим сторонам. Петя выбирает одну монетку наугад и подкидывает её
два раза. Оба раза выпадает орёл. Какова вероятность того, что
монетка <<неправильная>>?


\item Предположим, что социологическим опросам доверяют 70\,\% жителей. Те, кто доверяет опросам, всегда отвечают искренне; те, кто не доверяет, отвечают наугад, равновероятно выбирая <<да>> или <<нет>>. Социолог Петя  в анкету очередного опроса включил вопрос: <<Доверяете ли Вы социологическим опросам?>>
\begin{enumerate}
\item Какова вероятность, что случайно выбранный респондент ответит <<Да>>?
\item  Какова вероятность того, что он действительно доверяет, если известно, что он ответил <<Да>>?
\end{enumerate}

\item Петя и Вася независимо друг от друга решают одну и ту же задачу. Каждый из них может решить её с вероятностью 0.7. В тесте к задаче предлагается 5 ответов на выбор, поэтому будем считать, что выбор каждого из пяти ответов равновероятен, если задача решена неправильно.
\begin{enumerate}
\item Какова вероятность несовпадения ответов Пети и Васи?
\item  Какова вероятность того, что Петя ошибся, если ответы совпали?
\item  Каково ожидаемое количество правильных решений, если ответы совпали?
\end{enumerate}

\item В конкурсе участвовало четыре команды: Аз, Буки, Веди и Добро. Силы команд равны, поэтому разумно считать, что призовые места определяются случайным образом. Команды занявшие первое, второе и третье место будут награждены. 
\begin{enumerate}
\item Какова вероятность того, что команду Аз наградят?
\item Капитан команды Аз --- очень любопытный. Он спрашивает судью еще до официального объявления результатов: <<Назовите, пожалуйста, наугад одну любую команду, которую наградят>>. Судья отвечает: <<Буки>>. Какова условная вероятность того, что команду Аз наградят?
\item Капитан команды Аз --- очень любопытный. Он спрашивает судью еще до официального объявления результатов: <<Назовите, пожалуйста, наугад одну любую команду из наших соперников, которую наградят>>. Судья отвечает: <<Буки>>. Какова условная вероятность того, что команду Аз наградят?
\end{enumerate}


\end{enumerate}

\section{Встреча 3. Решение задач по вероятностям, среднее}

Задача про Петю и Васю (сначала находим вероятность правильного ответа), задача про команды.


Определение $E(X)$ в дискретном случае. Смысл --- средний выигрыш за партию при повторяемом эксперименте.

\section{Встреча 4. Описание кооперативной игры}




Описание кооперативной игры включает в себя:
\begin{enumerate}
\item Список игроков
\item Ответ на вопрос, сколько денег может заработать каждая коалиция отдельно от других игроков.
\end{enumerate}

Определение. Коалиция --- подмножество игроков. Большая коалиция --- все игроки как одна коалиция.


Пример. Ботинки: два левых один правый.


Пример. Носки с <<болваном>>.


Супераддитивная игра: игра, в которой $v(S_1 \cup S_2)\geq v(S_1)+v(S_2)$, если в коалициях $S_1$ и $S_2$ нет одних и тех же игроков. 


Вопросы: супераддитивны ли носки? ботинки? 


Здесь неплохо бы: пример несупераддитивной игры. Что-то на тему волк-коза-капуста (?)


В супераддитивной игре логично предположить, что все игроки объединяются, зарабатывают максимум и главный вопрос --- как поделить деньги.


Принцип дележа: на вход сообщаем ему характеристическую функцию, на выходе он выдае конкретный делёж.


По какой формуле делить деньги?


Предложенные варианты: 
\begin{enumerate}
\item отдать своё плюс равную часть <<добавленной стоимости>>, 
\begin{equation}
x_1=v(1)+\frac{1}{3}(v(N)-v(1)-v(2)-v(3))
\end{equation}
\item пропорционально $v(i)$ раздать $v(N)$
\item отдать своё плюс пропорционально <<добавленную стоимость>>, 
\begin{equation}
x_1=v(1)+\frac{v(1)}{v(1)+v(2)+v(3)}(v(N)-v(1)-v(2)-v(3))
\end{equation}
\end{enumerate}

Предложенные варианты не очень хороши: платят бездельнику ноль, неосуществимы, если для всех игроков $v(i)=0$.


Пример. Носки с <<менеджером по продажам>>


Пример. Мамонт с одинаковыми охотниками.


Пример. Мамонт с разными охотниками.


Пример. Мамонтята (все платежи в два раза больше)

Ваши предложения дележа в каждой из ситуаций? Что такое справедливость?


\begin{enumerate}
\item Бесполезные игроки получают ноль
\item Одинаковые игроки получают одинаково
\item Эффективность: вся сумма $v(N)$ распределяется между игроками
\item Линейность-1: при увеличении характеристической функции в $k$ раз мы хотели бы, чтобы платеж возрастал в $k$ раз. Сравнение 
\end{enumerate}

Вывод: лучше было сначала сделать игру, когда школьники находят дележ для каждой игры, а потом вырабатывать принципы.


\section{Встреча 5. Вектор Шепли}


Линейность-2.


Мамонт. Саблезубый тигр. Мамонт и саблезубый тигр. $x(v_1)+x(v_2)=x(v_1+v_2)$


Подвести к этой мысли можно так: предложите справедливый дележ в Мамонте, в Саблезубом тигре. А теперь представим себе комбинированную игру. Утром охотники заметили Мамонта, потом после еды отдохнули и на вечерней охоте заметили Саблезубого тигра. Соответственно, у нас есть $v=v_1+v_2$. И т.д.


Разделить на команды. В каждой команде 3 игрока. Каждая команда для каждой игры приводит свою версию <<справедливого>> распределения: каждый школьник команды играет за своего игрока. Можно выбрать наилучшего первого, второго и третьего игроков. Они могут сыграть между собой (мы так не делали, т.к. не ясно, что делать остальным).


Сенсация! Если вы хотите эффективности, <<бесполезные игроки получают ноль>>, <<одинаковые игроки получают одинаково>>, и линейностей, то существует единственный принцип построения дележа. Вектор Шепли. Shapley value. 


Определение. Вектор Шепли. Shapley value. 
\begin{enumerate}
\item Игроки формируют большую коалицию в случайном порядке
\item Каждый игрок получает при дележе математическое ожидание своего вклада
\end{enumerate}


Примеры расчёта. Сравнение с версиями игроков.



Вывод: лучше было сначала сделать игру, когда школьники находят дележ для каждой игры, а потом вырабатывать принципы.


\section{Встреча 6. Семинар по вектору Шепли}

Дорешиваем старые задачи с вектором Шепли. 


Гадюкино. Предлагаем справедливый делёж интуитивно. Первый --- один из тех, кто пользуется первой участкой дороги, второй --- один из тех, кто пользуется вторым участком дороге и т.д. Этот результат совпадает с вектором Шепли.


Условие задачи про транспортировку нефти.

\section{Встреча 7. Не было :)}




\section{Встреча 8. Недостатки вектора Шепли}



Решаем задачу про транспортировку нефти. 


Утв. Вектор Шепли даёт каждому игроку больше, чем тот может заработать сам.


Вектор Шепли может давать какой-нибудь коалиции меньше денег, чем она может заработать сама.


В некоторых играх (пример ботинок) нет решения удовлетворяющего всем требованиям вектора Шепли и устойчивости.


Играем в повторяемую 10 раз дилемму заключенного. 


\section{Встреча 9. Статические игры и равновесие Нэша }


Определение. В статической игре игроки одновременно независимо друг от друга делают свои ходы. Описание статической игры включает в себя:
\begin{enumerate}
\item Список игроков
\item Список разрешенных ходов для каждого игрока
\item Правило, определяющее, какой выигрыш получает каждый игрок, в зависимости от сделанных ходов
\end{enumerate}


Пример. Дуополия Курно. Две фирмы выпускают абсолютно идентичный товар одного качества. Выпуск первой фирмы обозначим $q_1$, второй --- $q_2$. Суммарный выпуск, $q=q_1+q_2$. Закон спроса (чем больше товара доступно для покупки, тем меньше оказывается его цена), $p=100-q=100-q_1-q_2$. Издержки производства равны $1$. Значит
\begin{equation}
\pi_1=q_1(100-q_1-q_2)-q_1
\end{equation}

Мы хотим ответить на вопрос: какой выпуск выберут фирмы?


Находим прибыли, если известно, что $q_1=10$, $q_2=20$.


Определение. Равновесие Нэша --- ситуация, в которой ни один игрок не захотел бы изменить свой выбор, даже если бы узнал о выборе другого. 


Равновесие ли $q_1=10$, $q_2=20$? Нет.


Что выбрала бы первая фирма, если бы узнала (подслушала), что $q_2=20$?


Рисуем параболу на скорость. 


Что выбрала бы первая фирма, если бы узнала $q_2$?


Человек плохо представляет себе большие числа. Представьте себе песчаный пляж. Сколько примерно на нём песчинок:


$2^5$, $2^{50}$, $2^{500}$, $2^{5000}$?


\section{Встреча 10. Статические игры. Смешанные стратегии}


Упражнение. Газета, 100 читателей. Каждый может отправить смс стоимостью 1 рубль. Если отправляет смс ровно один читатель, то он получает 50 рублей. Иначе никто ничего не выигрывает. Равновесие Нэша в чистых стратегиях?


Упражнение. Аукцион второй цены. Провально\ldots


С конкретными цифрами --- несодержательная игра, с обозначениями $v_1$, \ldots $v_5$, $b_1$, \ldots $b_5$ слишком трудно. 


Два пальца. Считаем ожидаемый выигрыш игроков...


\section{Встреча 11. Статические игры. Смешанные стратегии}


Дорешиваем <<Два пальца>>.


Газета. Смешанное равновесие. Какова средняя прибыль газеты?


Модель Бертрана. Равновесие Нэша.


\section{Встреча 12. Дисконт-фактор}

Модель Бертрана с компенсацией разницы в цене.


Дисконт-фактор, $\delta$, стоимость одного завтрашнего рубля сегодня, $0<\delta<1$.

Упражнения
\begin{enumerate}
\item Годовая процентная ставка равна 10\%. Чему равен годовой дисконт фактор?
\item Дисконт фактор равен $0.9$. Сколько стоят 1000 завтрашних рублей сегодня? А 1000 сегодняшних рублей завтра?
\item Дисконт фактор равен $0.9$. Сколько стоит сегодня один послезавтрашний рубль?
\end{enumerate}


Сумма бесконечно убывающей геометрической прогрессии. Способ через домножение или деление на повторяющийся сомножитель. Графическая иллюстрация суммы $1/4+1/16+1/64+\ldots$. Можно было еще и $1/2+1/4+1/8+\ldots$.

\section{Встреча 13. Бесконечноповторяемые игры}


Повторяемая игра. Примеры стратегий в бесконечноповторяемой игре.


Являются ли равновесием Нэша пары стратегий:
\begin{enumerate}
\item Всегда ходить $D$, Всегда ходить $D$
\item Всегда ходить $C$, Всегда ходить $C$
\item Стратегия переключения, Стратегия переключения
\end{enumerate}


Игра повторяемая два раза. Решение с конца.


Игра повторяемая $n$ раз.


<<Две рулетки>>. Есть три рулетки: на первой равновероятно выпадают числа 2, 4 и 9; на второй --- 1, 6 и 8; на третьей --- 3, 5 и 7. Сначала первый игрок выбирает рулетку себе, затем второй игрок выбирает рулетку себе из двух оставшихся. После этого рулетки, выбранные игроками, запускаются, и случай определяет победителя. Победителем считается тот, чья рулетка покажет большее число. Победитель получает от проигравшего 100 рублей.


Каким игроком лучше быть в этой игре?




\section{Встреча 14. Чай, обсуждение}

Окончание решения игры <<Две рулетки>>. Вероятность выигрыша для второго игрока --- $5/9$. Нетранзитивность.


<<Больший кусок окружности>>. Аня хватается за окружность в любой точке. Дальше мы делаем в случайных местах два разреза веревки. Аня забирает себе тот кусок, за который держится. Боря забирает оставшийся. Выигрыш каждого равен длине полученной им верёвки. В чью пользу эта игра?


Решение с рассмотрением трёх точек как неразличимых. Средние длины интервалов равны по $1/3$. Ане достаётся два интервала. Интуитивная аргументация в пользу Ани: пусть Аня выбирает точку случайно, но после разрезов. Шансов на то, что ей достанется больший кусок больше.


Чаепитие.




\end{document}
