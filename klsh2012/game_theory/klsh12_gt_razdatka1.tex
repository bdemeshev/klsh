\documentclass[pdftex,12pt,a4paper]{article}

\input{/home/boris/science/tex_general/title_bor_utf8}


\begin{document}
\parindent=0 pt % отступ равен 0


\begin{enumerate}

\item Злобным Вирусом заражен 1\% населения. Имеется тест, который ошибается в 5\% случаев. Если зараженный человек пройдет тест, то он будет признан здоровым с вероятностью 5\%. Если здоровый человек пройдет тест, то он будет признан больным с вероятностью 5\%. Рустам Байбурин прошел тест и судя по тесту он заражен Злобным Вирусом. Какова вероятность того, что он действительно заражен?

\item Два охотника выстрелили в одну утку. Первый попадает с
вероятностью 0,4, второй - с вероятностью 0,6. В утку попала ровно
одна пуля. Какова вероятность того, что утка была убита первым
охотником?

\item У тети Маши --- двое детей, один старше другого. Предположим, что вероятности рождения мальчика и девочки равны и не зависят от дня недели, а пол первого и второго ребенка независимы. \\
а) Известно, что у тети Маши есть хотя бы один мальчик. Какова
вероятность того, что у тети Маши есть девочка? \\
б) Тетя Маша наугад выбирает одного своего
ребенка и посылает к тете Оле, вернуть учебник по теории
вероятностей. Это оказывается мальчик. Какова вероятность того,
что другой ребенок - девочка? \\
в) Известно, что старший ребенок - мальчик. Какова вероятность того, что другой ребенок - девочка? \\
г) На вопрос: <<А правда ли тетя Маша, что у вас есть сын, родившийся в пятницу?>>. Она ответила: <<Да>>. Какова вероятность того, что другой ребенок --- девочка?


\item Имеется три монетки. Две <<правильных>> и одна "--- с орлами по
обеим сторонам. Петя выбирает одну монетку наугад и подкидывает её
два раза. Оба раза выпадает орёл. Какова вероятность того, что
монетка <<неправильная>>?


\item Предположим, что социологическим опросам доверяют 70\,\% жителей. Те, кто доверяет опросам, всегда отвечают искренне; те, кто не доверяет, отвечают наугад. Социолог Петя  в анкету очередного опроса включил вопрос: <<Доверяете ли Вы социологическим опросам?>>
\begin{enumerate}
\item Какова вероятность, что случайно выбранный респондент ответит <<Да>>?
\item  Какова вероятность того, что он действительно доверяет, если известно, что он ответил <<Да>>?
\end{enumerate}

\item Петя и Вася независимо друг от друга решают одну и ту же задачу. Каждый из них может решить её с вероятностью 0.7. В тесте к задаче предлагается 5 ответов на выбор, поэтому будем считать, что выбор каждого из пяти ответов равновероятен, если задача решена неправильно.
\begin{enumerate}
\item Какова вероятность несовпадения ответов Пети и Васи?
\item  Какова вероятность того, что Петя ошибся, если ответы совпали?
\item  Каково ожидаемое количество правильных решений, если ответы совпали?
\end{enumerate}

\item В конкурсе участвовало четыре команды: Аз, Буки, Веди и Добро. Силы команд равны, поэтому разумно считать, что призовые места определяются случайным образом. Команды занявшие первое, второе и третье место будут награждены. 
\begin{enumerate}
\item Какова вероятность того, что команду Аз наградят?
\item Капитан команды Аз --- очень любопытный. Он спрашивает судью еще до официального объявления результатов: <<Назовите, пожалуйста, наугад одну любую команду, которую наградят>>. Судья отвечает: <<Буки>>. Какова условная вероятность того, что команду Аз наградят?
\item Капитан команды Аз --- очень любопытный. Он спрашивает судью еще до официального объявления результатов: <<Назовите, пожалуйста, наугад одну любую команду из наших соперников, которую наградят>>. Судья отвечает: <<Буки>>. Какова условная вероятность того, что команду Аз наградят?
\end{enumerate}


\end{enumerate}



\end{document}