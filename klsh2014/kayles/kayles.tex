\documentclass[12pt,a4paper]{article}
\usepackage[utf8]{inputenc}
\usepackage[russian]{babel}

\usepackage{amsmath}
\usepackage{amsfonts}
\usepackage{amssymb}
\usepackage[left=2cm,right=2cm,top=2cm,bottom=2cm]{geometry}
\begin{document}
Игра <<Кегли>>

В ряд с некоторыми пропусками стоят $n$  кеглей.  Игроки по очереди вышибают кегли. За один ход можно вышибить любые одну или две рядом стоящих кегли. Выигрывает тот, кто вышиб последнюю кеглю.

Теория игры.

Каждой конфигурации кеглей $G$ можно сопоставить число в двоичной системе счисления, Ним-значение $N(G)$, по двум принципам:
\begin{enumerate}
\item если ряд кеглей $G$ разделен уже выбитой кеглей (или несколькими) на два участка $G_1$ и $G_2$, то $N(G) = N(G_1)\; XOR \;N(G_2)$
\item если из конфигурации $G$, выбив кегли согласно правилам, можно попасть в конфигурации $G_1$, $G_2$, \ldots, $G_k$, то $N(G)=\min \{ t \mid t \geq 0,  \forall i: t \neq N(G_i) \}$. Если словами, то $N(G)$ равно наименьшему двоичному числу не совпадающему ни с одним $N(G_i)$.
\end{enumerate}

Первый игрок выигрывает если и только если $N(G)>0$.



Табличка $N(G)$ для $n$ кеглей стоящих подряд без пропусков:


\begin{tabular}{cc}
\hline 
$n$ & $N(G)$ \\ 
\hline 
0 & 000 \\ 
1 & 001 \\ 
2 & 010 \\ 
3 & 011 \\ 
4 & 001 \\ 
5 & 100 \\ 
6 & 011 \\ 
7 & 010 \\ 
8 & 001 \\ 
9 & 100 \\ 
10 & 010 \\ 
\hline 
\end{tabular} 


Например, рассмотрим позицию $(6,7,8)$, т.е. 6 кеглей в ряд, пропуск, 7 кеглей в ряд, пропуск, 8 кеглей в ряд. Её Ним-значение равно $011 \; XOR\; 010 \; XOR\; 001=000$, т.е. это позиция проигрышная для ходящего первым. 

Например, рассмотрим $(4,5,6)$. Ним стоимость равна $001\; XOR\; 100 \; XOR \; 011=110$. Это выигрышная позиция для первого. Если перевести $100$ в $010$, то стоимость станет равной нулю, а позиция --- проигрышной для того, кто будет ходить вторым. Значит из ряда в 5 кеглей можно уничтожить 3 крайних, тогда в этом ряду останется две кегли, которые стоят $010$. 

Игра полностью решается, начиная с некоторого $n$ для ряда из $n$ кеглей величина $N(G)$ начинает вести себя периодически.



\end{document}