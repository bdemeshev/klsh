\documentclass[]{article}
\usepackage[utf8]{inputenc}
\usepackage[russian]{babel}
\usepackage{lmodern}
\usepackage{amssymb,amsmath}
\usepackage{ifxetex,ifluatex}
\usepackage{fixltx2e} % provides \textsubscript
% use upquote if available, for straight quotes in verbatim environments
\IfFileExists{upquote.sty}{\usepackage{upquote}}{}
\ifnum 0\ifxetex 1\fi\ifluatex 1\fi=0 % if pdftex
  \usepackage[utf8]{inputenc}
\else % if luatex or xelatex
  \ifxetex
    \usepackage{mathspec}
    \usepackage{xltxtra,xunicode}
  \else
    \usepackage{fontspec}
  \fi
  \defaultfontfeatures{Mapping=tex-text,Scale=MatchLowercase}
  \newcommand{\euro}{€}
\fi
% use microtype if available
\IfFileExists{microtype.sty}{\usepackage{microtype}}{}
\usepackage[margin=1in]{geometry}
\ifxetex
  \usepackage[setpagesize=false, % page size defined by xetex
              unicode=false, % unicode breaks when used with xetex
              xetex]{hyperref}
\else
  \usepackage[unicode=true]{hyperref}
\fi
\hypersetup{breaklinks=true,
            bookmarks=true,
            pdfauthor={},
            pdftitle={4+6},
            colorlinks=true,
            citecolor=blue,
            urlcolor=blue,
            linkcolor=magenta,
            pdfborder={0 0 0}}
\urlstyle{same}  % don't use monospace font for urls
\setlength{\parindent}{0pt}
\setlength{\parskip}{6pt plus 2pt minus 1pt}
\setlength{\emergencystretch}{3em}  % prevent overfull lines
\setcounter{secnumdepth}{0}

\title{4+6}
\author{}
\date{}

\begin{document}

\begin{center}
\huge Крестики-нолики 4+6 \\[0.2cm]
\normalsize
\end{center}


\begin{itemize}
\itemsep1pt\parskip0pt\parsep0pt
\item
  Соедини 4
\end{itemize}

Поле 8 на 8. Одна из сторон поля --- это пол. Ставить крестик или нолик
можно либо на пол, либо на уже стоящий крестик или нолик. ``Повесить''
крестик или нолик ``без опоры'' нельзя! Выигрывает тот, кто поставит
четыре в ряд.

\begin{itemize}
\itemsep1pt\parskip0pt\parsep0pt
\item
  Соедини 6
\end{itemize}

Поле 19 на 19. Первым ходом игрок ставит один крестик. Далее игроки по
очереди ставят по два знака (два нолика - два крестики - два
нолика\ldots{}) Выигрывает тот, кто поставит шесть в ряд.



\begin{center}
\huge Крестики-нолики 4+6 \\[0.2cm]
\normalsize
\end{center}


\begin{itemize}
\itemsep1pt\parskip0pt\parsep0pt
\item
  Соедини 4
\end{itemize}

Поле 8 на 8. Одна из сторон поля --- это пол. Ставить крестик или нолик
можно либо на пол, либо на уже стоящий крестик или нолик. ``Повесить''
крестик или нолик ``без опоры'' нельзя! Выигрывает тот, кто поставит
четыре в ряд.

\begin{itemize}
\itemsep1pt\parskip0pt\parsep0pt
\item
  Соедини 6
\end{itemize}

Поле 19 на 19. Первым ходом игрок ставит один крестик. Далее игроки по
очереди ставят по два знака (два нолика - два крестики - два
нолика\ldots{}) Выигрывает тот, кто поставит шесть в ряд.

\begin{center}
\huge Крестики-нолики 4+6 \\[0.2cm]
\normalsize
\end{center}


\begin{itemize}
\itemsep1pt\parskip0pt\parsep0pt
\item
  Соедини 4
\end{itemize}

Поле 8 на 8. Одна из сторон поля --- это пол. Ставить крестик или нолик
можно либо на пол, либо на уже стоящий крестик или нолик. ``Повесить''
крестик или нолик ``без опоры'' нельзя! Выигрывает тот, кто поставит
четыре в ряд.

\begin{itemize}
\itemsep1pt\parskip0pt\parsep0pt
\item
  Соедини 6
\end{itemize}

Поле 19 на 19. Первым ходом игрок ставит один крестик. Далее игроки по
очереди ставят по два знака (два нолика - два крестики - два
нолика\ldots{}) Выигрывает тот, кто поставит шесть в ряд.




\end{document}
