\documentclass[12pt,a4paper]{article}
\usepackage[utf8]{inputenc}
\usepackage[russian]{babel}

\usepackage{amsmath}
\usepackage{amsfonts}
\usepackage{amssymb}
\usepackage[left=2cm,right=2cm,top=2cm,bottom=2cm]{geometry}
\begin{document}


Три интеллигентных поросёнка, Ниф-Ниф, Наф-Наф и Нуф-Нуф побывали в Летней школе. После школы они понимают русский язык, но отвечать умеют только <<да>> или <<нет>>. Один из поросят отвечает всегда правдиво, другой --- всегда лжёт, третий --- равновероятно лжёт или отвечает правдиво.

Как с помощью трёх вопросов определить, кто из поросят лжёт, кто --- говорит правду, а кто --- отвечает наугад? Разрешено задавать только вопросы, на которые всегда можно ответить <<да>> или <<нет>>.


Решение.

\begin{enumerate}


\item Первым вопросом определим неслучайного поросёнка. Для этого спросим Ниф-Нифа:

<<Верно ли, что из следующих двух утверждений ровно одно является верным:

Ниф-Ниф --- лжец

Наф-Наф --- случайный?>>


Перебором вариантов определяем, что ответ <<да>> гарантирует, что Нуф-Нуф --- неслучайный. А ответ <<нет>> --- гарантирует что Наф-Наф --- неслучайный. 

\item Определяем, кем является неслучайный поросёнок. Для этого спрашиваем его:

<<Если бы я спросил тебя, правда ли, что ты --- лжец, ответил бы ты да?>>

Перебором устанавливаем, что лжец отвечает на данный вопрос <<да>>, а правдивый поросёнок --- <<нет>>.

\item Третий вопрос прост

\end{enumerate}


\end{document}