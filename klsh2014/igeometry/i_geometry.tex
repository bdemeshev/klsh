\documentclass[12pt,a4paper]{article}
\usepackage[utf8]{inputenc}
\usepackage[russian]{babel}

\usepackage{amsmath}
\usepackage{amsfonts}
\usepackage{amssymb}
\usepackage[left=2cm,right=2cm,top=2cm,bottom=2cm]{geometry}
\begin{document}
\section{Занятие 1}



Комплексное число --- вектор  на плоскости. 


Для краткости вместо двух чисел в скобках $(3, 4)$ пишут $3+4i$. Например, $7$, $-2$ --- горизонтальные векторы, а $2i$, $-5i$ --- вертикальные векторы.


Сложение чисел. Геометрическая и арифметическая интерпретация.


Страшные слова:
\begin{enumerate}
\item Длина, модуль, $|z|=\sqrt{a^2+b^2}$
\item Действительная часть, $Re(z)=a$
\item Мнимая часть $Im(z)=b$
\item Аргумент, угол с положительным направлением действительной оси $Arg(z)$
\end{enumerate}


Умножение чисел. Геометрическая интерпретация.

При умножении двух комплексных чисел их длины умножаются, а углы (аргументы) складываются.


Из геометрических соображений находим $(1+i)^2$, $i^2$, (примерно) $(3+4i)\cdot (-2+2i)$.

Арифметическая интерпретация:
\begin{enumerate}
\item раскрывай скобки
\item упрощай $i^2=-1$
\end{enumerate}

Находим $(1+i)^2$ и $(3+4i)\cdot (-2+2i)$ арифметически.

Доказательства одинаковости обоих интерпретаций.

Берём произвольное комплексное число $z$.
\begin{enumerate}
\item  случай умножения на $i$
\item  случай умножения на положительное число $3$
\item умножение на $(3+4i)$
Замечаем, что $(3+4i)*z = 3z + 4(z \text{ повернутое на } \pi/2)$.
Рисуем. О чудо! Углы сложились, а длина домножилась на 5.
\end{enumerate}


Упр. 
\begin{enumerate}
\item Найди $(2+3i)\cdot (1-i)$, $(2+5i)/(1-i)$ и $1+i+i^2+i^3+i^4$
\item Реши уравнение $z^2=-1$
\end{enumerate}

Коммент:
\begin{enumerate}
\item случай умножения на положительное число, пожалуй, лучше было не рассказывать
\item 14 человек = 3 девятых + 6 десятых + 5 одиннадцатых
\end{enumerate}


\section{Лекция 2}


Решаем уравнение $z^3=i$

Составляем табличку возможных аргументов $i$. Делим их на 3 во втором столбике. Рисуем все решения. Выписываем все решения.

Упражнение:
Реши уравнение $z^4=1$,  $w^4=i$

используем $\cos 2\alpha =\cos^2 \alpha - 1$

Решаем квадратные уравнения (отрицательный дискриминант).

Ввели обозначения, $\mathbb{R}$, $\mathbb{C}$

выводим $\cos 2\alpha =\cos^2 \alpha - 1$, 
$\cos ( \alpha + \beta) $ 

Коммент: 

$\cos ( \alpha + \beta) $  --- лучше было отложить на потом :)

\section{Занятие 3}


Сумма $1+1/2+1/4+1/8+\ldots$. 

Геометрический способ нахождения. Арифметический способ нахождения с домножением на $1/2$.

Вечная черепаха. Всю жизнь движется по прямой. В первый час своей жизни движется со скоростью 10 км/ч, затем каждый час её скорость падает на 20\%. Какой путь черепаха пройдет за свою бесконечную жизнь?

Вечная черепаха-2. Стартует в начале координат. Изначально ползёт вправо, затем каждый час поворачивает на 90$^{\circ}$ влево. Где она окажется в конце своего жизненного пути?

Вечная черепаха-3. На 45$^{\circ}$? 


Три формы записи комплексных чисел:

\begin{enumerate}
\item алгебраическая $1+\sqrt{3}i$
\item тригонометрическая $2(\cos \frac{\pi}{3} + \sin \frac{\pi}{3} )$
\item экспоненциальная $2e^{i\frac{\pi}{3}}$.
\end{enumerate}

Задача про три квадрата. Чему равна сумма углов?




Коммент: 

вечная черепаха-3 --- <<плохие>> числа.

экспоненциальная форма слегка подвисла


\section{Занятие 4}

Повторяем три форма записи комплексных чисел.

Упражнение. Запишите во всех формах $-1+\sqrt{3}i$, $-3$, $4e^{i2\pi/3}$, $-7(\cos (-\pi/3)+i\sin(-\pi/3))$

Середина двух точек, $z=\frac{1}{2}z_1+\frac{1}{2}z_2$.

Параллелограмм, середины диагоналей совпадают.

Является ли $1+i$, $3+4i$, $9\sqrt{2}e^{i\pi/4}$, $7+6i$ параллелограммом?

Делим отрезок в произвольном соотношении 4:7.

Медианы треугольника пересекаются в одной точке и делятся в соотношении 2:1. Идея доказательства с помощью комплексных чисел. Пройдем из вершины 2/3 медианы. Пройдем сторону и 2/3 другой медианы. Попадём в одну точку :)


Сопряжение. $z=a+bi$, $\bar{z}=a-bi$.

\section{Занятие 5}


Многозначные функции. Пример --- $arg(z)$, $Arg(z)$, $\sqrt{z}$. 

сопряжение и сумма
сопряжение и произведение
сопряженные корни ходят парами

1.5 Дан положительно ориентированный квадрат ABCD и ком- плексные координаты a и b его вершин A и B. Найдите комплексные координаты вершин C и D

Дан правильный треугольник ABC и комплексная координата a вершины A. Найдите комплексную координату вершины B при положительной и отрицательной ориентациях треугольника ABC, если зa начальную точку принята 1) вершина C, 2) центр треугольника ABC , 3) основание A1 высоты AA1 .

Задача 2. (стр 13)

1.10. Средняя линия четырёхугольника делит его на два че- тырёхугольника. Докажите, что середины диагоналей этих двух че- тырёхугольников являются вершинами параллелограмма, либо лежат на одной прямой
1.11. Докажите, что сумма квадратов медиан треугольника равна 3 суммы квадратов его сторон.
4
1.12. Докажите, что сумма квадратов диагоналей параллелограмма
равна сумме квадратов всех его сторон.
1.13. Докажите, что сумма квадратов диагоналей четырёхугольни-
ка равна удвоенной сумме квадратов его средних линий.

1.16. Точки A и B симметричны относительно центра некоторой окружности. Докажите, что для любой точки M этой окружности значение суммы M A2 + M B2 постоянно.




отрезки AB и CD параллельны тогда и только тогда,
когда число (a−b)/(c-d) является действительным.



Достройте до параллелограмма?


Два параллелограмма и середины сторон


нарисуйте $|z-3+4i|=2$
нарисуйте $|z-3+2i|=i+7$
нарисуйте $|z-i|=|z+i|$

$z=\bar{z}$

преобразование плоскости. 
возведение в квадрат

$(1+\sqrt{3}i)^3/(1-i)^2$
$(1+i)^5/(\sqrt{3}+i)^2$
$(1+\sqrt{3}i)^6$

сопряжение комплексных чисел




Теорема о даме с собачкой? или позже?



$(1+i)^4$


$1/(2-i)$

теорема Абеля в задачах и упражнениях
тривиум нму
другие материалы нму



Последнее занятие --- контрольная :)


Литература:

Понарин, Алгебра комплексных чисел в геометрических задачах

Tristan Needham, Visual complex analysis.

Яглом, Комплексные числа

Ссылка на видео




\end{document}

