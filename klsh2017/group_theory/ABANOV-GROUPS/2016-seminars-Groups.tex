%!TEX TS-program =  LuaLaTeX
%!TEX encoding = UTF-8 Unicode

%\documentstyle[aps,prb,preprint]{revtex}
%\documentstyle[aps,prb]{revtex}

%\documentclass[12pt]{article}
%\documentclass[aps,amsmath,amssymb,nofootinbib,12pt]{revtex4}
\documentclass[amsmath,amssymb,12pt]{revtex4}
%\tolerance = 10000
%\documentclass[11pt,twocolumn]{article}

%\usepackage{multicol}
%\usepackage[dvips]{graphicx}
\usepackage{graphicx}
\usepackage{amsmath}
\usepackage{euler}
\usepackage{amsmath,amsfonts,fullpage,graphicx}
\usepackage{bm}

%\usepackage{fontenc}
\usepackage[T1,T2A]{fontenc}
\usepackage[utf8]{inputenc}
\usepackage[russian,english]{babel}

\usepackage{color}
\def\red{\color{red}}
\def\magenta{\color{magenta}}
\def\blue{\color{blue}}
\def\green{\color{green}}

%\defaultfontfeatures{Scale=MatchLowercase} 
%\setromanfont[Mapping=tex-text]{Geneva CY} 
%\setsansfont[Mapping=tex-text]{Geneva CY} 
%\setmonofont{Geneva CY}
%\setmainfont{RomanCyrillic_Std.ttf}
%%% other fonts: Lucida Grande
 
%\usepackage{euler}
%\usepackage[cm-default]{fontspec}
%\usepackage{xltxtra}

%to hide or show solutions to the problems
\newif\ifimportant
\importanttrue 
% \importantfalse


%%%%%%%%%%%%%%%%%%%%%%%%
\def\Tr{\mbox{Tr}\,}
\newcommand{\la}{\label}
\newcommand{\bbm}{\begin{multline}}
\newcommand{\eem}{\end{multline}}
\newcommand{\be}{\begin{equation}}
\newcommand{\ee}{\end{equation}}
\newcommand{\bea}{\begin{eqnarray}}
\newcommand{\eea}{\end{eqnarray}}
\newcommand{\p}{\partial}
\newcommand{\1}{\frac{1}{2}}
\newcommand{\R} {\mbox{Re}\,}
\newcommand{\I} {\mbox{Im}\,}
\newcommand{\cF} {\mathcal{F}}
\newcommand{\cL} {\mathcal{L}}
\newcommand{\cM} {\mathcal{M}}
\newcommand{\cX} {\mathcal{X}}
\newcommand{\cJ} {\mathcal{J}}
\newcommand{\cE} {\mathcal{E}}
\newcommand{\cP} {\mathcal{P}}
\newcommand{\cB} {\mathcal{B}}
\newcommand{\gzz} {g_{zz}}
\newcommand{\gzb} {g_{z\bar{z}}}
\newcommand{\gbb} {g_{\bar{z}\bar{z}}}
\newcommand{\comment}[1]{}
%%%%%%%%%%%%%%%%%%%%%%%

%
\language=1
\baselineskip = 1.2\baselineskip
\topmargin 0pt
\textheight 35\baselineskip
%\textheight 50\baselineskip % for A4 paper
\textheight 35\baselineskip % for Letter paper
\advance \textheight by \topskip
\marginparwidth 16mm
%\textwidth 15cm
\textwidth 17cm
\evensidemargin 9mm
%\oddsidemargin 9mm
\oddsidemargin 0mm

\renewcommand{\deg}{{\sp\circ}}
\newcommand{\blank}[1]{\underline{\hspace{#1}}}
%%%%%%%% Lists 
\renewcommand{\labelenumi}{{\bf \theenumi.}}
\newcounter{saveenumi}
\newcommand{\intert}[1]{\setcounter{saveenumi}{\value{enumi}}
 \end{enumerate}\par #1
  \begin{enumerate}\itemsep=5pt
  \setcounter{enumi}{\value{saveenumi}}}
\newcommand{\staritem}{\refstepcounter{enumi}\item[\bf *\theenumi.]}
\newcommand{\starsubitem}{\refstepcounter{enumii}\item[*(\theenumii)]
}
\newcommand{\epigraph}[1]{\hfill\parbox{9cm}{\flushright{\sl\small#1}}
}
%\newtheorem{theorem}{Theorem}
%\newtheorem*{theorem*}{Theorem}
%\newtheorem{axiom}{Axiom}
%\theoremstyle{definition}
%\newtheorem{definition}{Definition}                  
%\newtheorem*{definition*}{Definition}                  

\newcommand{\st}{\,|\,}
\newcommand{\re}{\,\mbox{Re}\,}
\newcommand{\im}{\,\mbox{Im}\,}
\newcommand{\fig}[1]
{\raisebox{-0.5\height}%
{\includegraphics{figures/#1}}}


\newcommand{\ph}{\varphi}

\newcommand{\lin}[1]{\stackrel{\longleftrightarrow}{#1}}
\newcommand{\ray}[1]{\stackrel{\longrightarrow}{#1}}
%\renewcommand{\iff}{\leftrightarrow}
\newcommand{\tri}{\triangle}
\newcommand{\ov}[1]{\overline{#1}}
\newcommand{\arc}[1]{\stackrel{\frown}{#1}}

%\newcommand{\_}{\underline{\ }}
\newcommand{\bblank}{\underline{\hspace{2cm}}}

%\pagenumbering{}


%\newtheorem{theorem}{Theorem}
%\newtheorem*{theorem*}{Theorem}
%
%\newtheorem{axiom}{Axiom}
%\theoremstyle{definition}
%\newtheorem{definition}{Definition}                  
%\newtheorem*{definition*}{Definition}                  
%\newcommand{\thref}[1]{Theorem~{\rm\ref{#1}}}
%\newcommand{\axref}[1]{Axiom~{\rm\ref{#1}}}
%%%%%%%%%%%%%%%%%%%%%%%%%%%%%
\renewcommand{\deg}{{\sp\circ}}
%\newcommand{\blank}[1]{\underline{\hspace{#1}}}
\newcommand{\si}{\sigma}
%%%%%%%%%%%%%%%%%
%\newcommand{\fig}[1]
%{\raisebox{-0.5\height}%
%{\includegraphics{figures/#1}}}

%%%%%%%% Lists 
\renewcommand{\labelenumi}{{\bf \theenumi.}}
%\newcounter{saveenumi}
%\newcommand{\intert}[1]{\setcounter{saveenumi}{\value{enumi}}
% \end{enumerate}\par #1
%  \begin{enumerate}\itemsep=5pt
%  \setcounter{enumi}{\value{saveenumi}}}
%\newcommand{\staritem}{\refstepcounter{enumi}\item[\bf *\theenumi.]}
%\newcommand{\starsubitem}{\refstepcounter{enumii}\item[*(\theenumii)]}
%\newcommand{\epigraph}[1]{\hfill\parbox{9cm}{\flushright{\sl\small#1}}}

\begin{document}

\title{Группы и Симметрии}
\author{Саша Абанов\\
{\small Университет Стони Брук, США}}
    %%
     % \\ {\it Department of Physics and Astronomy,
     %       SUNY at Stony Brook,
     %       Stony Brook, NY 11794-3800 }}
     %%

\date{июль-август, 2015}

%\author{E. Bettelheim}
%\affiliation{James Frank Institute, University of Chicago, 5640 S.
%Ellis Ave. Chicago IL 60637.}
%\author{A. G. Abanov}
%\affiliation{Department of Physics and Astronomy,
%Stony Brook University,  Stony Brook, NY 11794-3800.}
%\author{P. Wiegmann}
%\affiliation{James Frank Institute, University of Chicago, 5640 S.
%Ellis Ave. Chicago IL 60637.}
%\affiliation{Also at Landau Institute of Theoretical Physics.}

%В основе современной физики лежит понятие симметрии. Например, мы говорим, что энергия сохраняется потому, что законы физики инвариантны относительно временных сдвигов, а электрический заряд сохраняется потому, что наша Вселенная обладает калибровочной симметрией. Теория групп и их представлений – это математический аппарат, который позволяет работать с симметриями и их следствиями. В этом курсе мы познакомимся с некоторыми основными понятиями теории групп. Основной упор будет делаться на разбор простейших примеров и приложений теории групп. 
%Курс рассчитан на один модуль. Предварительных знаний за пределами школьной программы 9-го класса не требуется. 
%
%
%О лекторе:
%
%Саша Абанов, профессор университета Стони Брук, США
%Специальность: теоретическая физика твердого тела и математическая физика
%
%Участие в КЛШ:
%1983		школьник (после 9-го класса) 
%1984               	зондер 
%1985-1990     	вожатый 
%2006-7          	лектор 
%2010               	лектор 
%2012-15          	лектор     



\begin{abstract}
Данные записки представляют с собой краткое содержание и задачи к семинарам курса прочитанного автором в Красноярской Летней Школе (КЛШ) в 2016 году для школьников старших классов. Курс представлял собой введение в теорию групп. Курс состоял из 3 лекций и 3 семинаров. \\

В основе современной физики лежит понятие симметрии. Например, мы говорим, что энергия сохраняется потому, что законы физики инвариантны относительно временных сдвигов, а электрический заряд сохраняется потому, что наша Вселенная обладает калибровочной симметрией. Теория групп и их представлений – это математический аппарат, который позволяет работать с симметриями и их следствиями. В этом курсе мы познакомимся с некоторыми основными понятиями теории групп. Основной упор будет делаться на разбор простейших примеров и приложений теории групп. 
%Курс рассчитан на один модуль. Предварительных знаний за пределами школьной программы 9-го класса не требуется. 

\end{abstract}


\maketitle

\newpage
\tableofcontents


\newcounter{prn}

%%%%%%%%%%%%%%%%%%%%%%%%%%%%%%%%%%%%%%%%
%%%%%%%%%%%%%%%%%%%%%%%%%%%%%%%%%%%%%%%%
%\newpage 
%\setcounter{page}{1}
%\section*{Содержание}
%\setcounter{prn}{0}
%%%%%%%%%%%%%%%%%%%%%%%%%%%%%%%%%%%%%%%%
%
%\begin{itemize}
%	\item Использованная литература
%	\item Идеи проектов на выставку итоговых проектов  КЛШ
%	\item Лекция 1: Относительность Галилея и Ньютона 
%	\item Семинар 1: Сложение скоростей в механике Ньютона
%	\item Лекция 2: Преобразования Лоренца 
%	\item Семинар 2: Преобразования Лоренца и сложение скоростей
%	\item Лекция 3: Следствия преобразований Лоренца
%	\item Семинар 3: Релятивистские эффекты
%	\item Лекция 4: Геометрия пространства-времени. Пространство Минковского.
%	\item Семинар 4: 
%\end{itemize}
%		

%%%%%%%%%%%%%%%%%%%%%%%%%%%%%%%%%%%%%%%
%%%%%%%%%%%%%%%%%%%%%%%%%%%%%%%%%%%%%%%
\newpage 
%\setcounter{page}{1}
\section*{\large Использованная литература}
\setcounter{prn}{0}
%%%%%%%%%%%%%%%%%%%%%%%%%%%%%%%%%%%%%%%


При подготовке курса автор использовал курс прочитанный Павлом Этингофом для школьников 
\begin{itemize}
	\item P. Etingof, \textit{Groups around us}. 
%	\item C. D. Olds, \textit{Continued Fractions}, Math Association of America, 1963.
%	\item A. M. Rockett, P. Sz\"{u}sz, \textit{Continued Fractions}, World Scientific, 1992.
%	\item В. И. Арнольд, \textit{Цепные дроби}, МЦНМО, 2001.
\end{itemize}

Я также использовал статьи из журнала ``Квант''
\begin{itemize}
	\item А. Колмогоров, \textit{Группы преобразований}, Квант. 
	\item Л. Садовский, М. Аршинов, \textit{Группы}, Квант. 
	\item А. Б. Сосинский, \textit{Конечные группы}, Квант. 
	\item А. И. Кострикин, \textit{Простые группы}, Квант. 
	\item Э. Белага, \textit{Алгебра -- древняя и современная}, Квант.
\end{itemize}
%и задачи из онлайн сборника problems.ru.

%%%%%%%%%%%%%%%%%%%%%%%%%%%%%%%%%%%%%%%%
%%%%%%%%%%%%%%%%%%%%%%%%%%%%%%%%%%%%%%%%
%%\newpage 
%%\setcounter{page}{1}
\section*{\large Идеи проектов на выставку итоговых проектов  КЛШ}
\setcounter{prn}{0}
%%%%%%%%%%%%%%%%%%%%%%%%%%%%%%%%%%%%%%%


\begin{enumerate}
	\item Раскраска многоугольников и малая теорема Ферма.
	\item Раскраска многогранников и теорема Полиа.
	\item *Группа симметрий Судоку.
	%Аня Недорубова
\end{enumerate}
Проекты отмеченные звёздочкой были представлены школьниками КЛШ на выставку итоговых проектов.




%%%%%%%%%%%%%%%%%%%%%%%%%%%%%%%%%%%%%%%
%%%%%%%%%%%%%%%%%%%%%%%%%%%%%%%%%%%%%%%
\newpage
%\setcounter{page}{1}
\section*{\large Лекция 1: Группы преобразований.}
\setcounter{prn}{0}
%%%%%%%%%%%%%%%%%%%%%%%%%%%%%%%%%%%%%%%

\begin{enumerate}
	\item Примеры преобразований и действий. 
		\begin{enumerate}
			\item Композиции преобразований. 
			\item Тождественное преобразование.
			\item Обратимые преобразования.
			\item Обратные преобразования.
		\end{enumerate}
	\item Определение группы преобразований.
	\item Ассоциативность композиции преобразований.
	\item Таблицы умножения для некоторых групп преобразований.
	\item Первое знакомство с изоморфизмом групп. 
	\item Дальнейшие примеры:
		\begin{enumerate}
			\item Группа перестановок и знакопеременная группа.
			\item Группы движений плоскости и пространства.
			\item Группы симметрий геометрических фигур.
		\end{enumerate}
\end{enumerate}



%%%%%%%%%%%%%%%%%%%%%%%%%%%%%%%%%%%%%%%
%%%%%%%%%%%%%%%%%%%%%%%%%%%%%%%%%%%%%%%
\newpage
%\setcounter{page}{1}
\section*{\large Семинар 1: Группы преобразований.}
\setcounter{prn}{0}
\importantfalse % UNCOMMENT TO HIDE SOLUTIONS
%%%%%%%%%%%%%%%%%%%%%%%%%%%%%%%%%%%%%%%

%%%%%%%%%%%%%%%%
%\addtocounter{prn}{1}
%\vspace{0.3cm}
%\paragraph*{Задача \theprn: }
%%
%%%%%%%%%%%%%%%%
%
%Составьте таблицу умножения для преобразований симметрии прямоугольника. Идентифицируйте обратный элемент для каждой из симметрий.
%
%
%\ifimportant
%% only shown if \importanttrue is set
%\medskip
%\noindent
%ОТВЕТ: поворот на 180$^{o}$ вокруг оси, проходящей через середины противоположных ребер куба.  
%\fi

%%%%%%%%%%%%%%%
\addtocounter{prn}{1}
\vspace{0.3cm}
\paragraph*{Задача \theprn: }
%
%%%%%%%%%%%%%%%

Составьте таблицу умножения для преобразований симметрии правильного треугольника. Идентифицируйте обратный элемент для каждой из симметрий.


%\ifimportant
%% only shown if \importanttrue is set
%\medskip
%\noindent
%ОТВЕТ: поворот на 180$^{o}$ вокруг оси, проходящей через середины противоположных ребер куба.  
%\fi




%%%%%%%%%%%%%%%
\addtocounter{prn}{1}
\vspace{0.3cm}
\paragraph*{Задача \theprn: }
%
%%%%%%%%%%%%%%%

Показать, что группа симметрий правильного треугольника изоморфна группе перестановок $S_{3}$. 

\ifimportant
% only shown if \importanttrue is set
\medskip
\noindent
Решение
Сравнить таблицы умножений двух групп.
\fi




%%%%%%%%%%%%%%%
\addtocounter{prn}{1}
\vspace{0.3cm}
\paragraph*{Задача \theprn: }
%Belaga 3
%%%%%%%%%%%%%%%

Рассмотрите поворот куба на 90$^{o}$ вокруг оси, проходящей через середины двух противоположных граней и поворот на 120$^{o}$ вокруг главной диагонали куба. Назовем эти повороты буквами $\alpha$ и $\beta$, соответственно. Найдите симметрию, соответствующую композиции $\beta\circ \alpha$. Совпадает ли эта операция с $\alpha\circ \beta$?


\ifimportant
% only shown if \importanttrue is set
\medskip
\noindent
ОТВЕТ: поворот на 180$^{o}$ вокруг оси, проходящей через середины противоположных ребер куба.  
\fi


%%%%%%%%%%%%%%%
\addtocounter{prn}{1}
\vspace{0.3cm}
\paragraph*{Задача \theprn: }
%Kirillov 10.18.3
%%%%%%%%%%%%%%%

Найдите перестановку обратную следующей:
$$
	\left(
	\begin{array}{cccccc}
		1& 2& 3& 4& 5& 6 \\
		3& 1& 6& 4& 2& 5
	\end{array}
	\right)
$$
Запишите эту перестановку как произведение циклов.

%\ifimportant
%% only shown if \importanttrue is set
%\medskip
%\noindent
%ОТВЕТ: поворот на 180$^{o}$ вокруг оси, проходящей через середины противоположных ребер куба.  
%\fi


%%%%%%%%%%%%%%%
\addtocounter{prn}{1}
\vspace{0.3cm}
\paragraph*{Задача \theprn: }
%Kirillov 10.18.4
%%%%%%%%%%%%%%%

Покажите, что каждая перестановка может быть записана как произведение непересекающихся циклов.


%\ifimportant
%% only shown if \importanttrue is set
%\medskip
%\noindent
%ОТВЕТ: поворот на 180$^{o}$ вокруг оси, проходящей через середины противоположных ребер куба.  
%\fi

%%%%%%%%%%%%%%%
\addtocounter{prn}{1}
\vspace{0.3cm}
\paragraph*{Задача \theprn: }
%Kirillov 10.18.5
%%%%%%%%%%%%%%%

15 школьников сидят в классе с 15 стульями занумерованными числами от 1 до 15. Учитель требует, чтобы школьники пересаживались каждую минуту по следующему правилу:
$$
	\begin{array}{ccccccccccccccc}
	1&2&3&4&5&6&7&8&9&10&11&12&13&14&15 \\
	3&5&10&8&11&14&15&6&13&1&4&9&7&2&12
	\end{array}
$$
(a) Запишите эту перестановку как произведение циклов

\noindent
(b) Через сколько минут все студенты вернуться на свои первоначальные места?
%\ifimportant
%% only shown if \importanttrue is set
%\medskip
%\noindent
%ОТВЕТ: поворот на 180$^{o}$ вокруг оси, проходящей через середины противоположных ребер куба.  
%\fi



%%%%%%%%%%%%%%%
\addtocounter{prn}{1}
\vspace{0.3cm}
\paragraph*{*Задача \theprn: }
%Kirillov 10.18.6
%%%%%%%%%%%%%%%

Порядком перестановки $s$ называется наименьшее число $d$ такое, что $s^{d}=I$, где $I$ - тождественная перестановка. 

\noindent
(a) Найдите порядок цикла длины $n$. 

\noindent
(b) Найдите порядок перестановки $(12)(34795)(6\,10\,11\,12\,13\,14\,15)$.

\noindent
(c) Пусть перестановка $s$ является произведением непересекающихся циклов длин $n_{1},n_{2}, \ldots, n_{k}$ (в этом случае мы пишем, что перестановка имеет \textbf{тип} $\langle n_{1},n_{2},\ldots, n_{k}\rangle$). Каков порядок перестановки $s$?

\noindent
(d) Найдите примеры перестановок длины 9, которые имеют порядок $7, 10, 12, 11$ (если они существуют)?


%\ifimportant
%% only shown if \importanttrue is set
%\medskip
%\noindent
%ОТВЕТ: поворот на 180$^{o}$ вокруг оси, проходящей через середины противоположных ребер куба.  
%\fi


%%%%%%%%%%%%%%%
\addtocounter{prn}{1}
\vspace{0.3cm}
\paragraph*{*Задача \theprn: }
%Belaga 6
%%%%%%%%%%%%%%%

Обозначим как $a$ и $b$ два некоторых поворота икосаэдра на угол 72$^{o}$ вокруг двух соседних вершин. Докажите, что любую симметрию икосаэдра можно записать как композицию некоторого количества операций $a$ и $b$ (например, $a\circ b\circ b\circ a\circ b$). 


%\ifimportant
%% only shown if \importanttrue is set
%\medskip
%\noindent
%ОТВЕТ: поворот на 180$^{o}$ вокруг оси, проходящей через середины противоположных ребер куба.  
%\fi




%%%%%%%%%%%%%%%
\addtocounter{prn}{1}
\vspace{0.3cm}
\paragraph*{Задача \theprn: }
%Belaga 5
%%%%%%%%%%%%%%%

Найдите все симметрии тетраэдра, октаэдра, куба, икосаэдра$^{*}$ и додекаэдра$^{*}$. Какие из них удовлетворяют равенству $a^{2}=e$? Равенству $a^{3}=e$? Равенству $a^{5}=e$? Составьте таблицы умножений для полученных групп симметрий.


%\ifimportant
%% only shown if \importanttrue is set
%\medskip
%\noindent
%ОТВЕТ: поворот на 180$^{o}$ вокруг оси, проходящей через середины противоположных ребер куба.  
%\fi


%%%%%%%%%%%%%%%
\addtocounter{prn}{1}
\vspace{0.3cm}
\paragraph*{Задача \theprn: }
%problems.ru Задача 60374
%%%%%%%%%%%%%%%

Сколько существует ожерелий, составленных из 17 различных бусинок? 
\ifimportant
% only shown if \importanttrue is set
\medskip
\noindent
Решение
17 предметов можно расставить по кругу 16! способами (см. задачу 60373). Но ожерелье можно еще и перевернуть, что сокращает число способов вдвое.

Ответ
16!/2. 
\fi






%%%%%%%%%%%%%%%%%%%%%%%%%%%%%%%%%%%%%%%
%%%%%%%%%%%%%%%%%%%%%%%%%%%%%%%%%%%%%%%
\newpage
%\setcounter{page}{1}
\section*{\large Лекция 2: Понятие абстрактной группы}
%
\setcounter{prn}{0}
%%%%%%%%%%%%%%%%%%%%%%%%%%%%%%%%%%%%%%%


\begin{enumerate}
	\item Бинарные операции
	\item Определение абстрактной группы. Аксиомы группы.
	\item Коммутативные (Абелевые) группы
	\item Гомоморфизм и изоморфизм
	\item Подгруппы
	\item Порядок элемента группы и подгруппы
	\item Теорема Лагранжа
\end{enumerate}


%%%%%%%%%%%%%%%%%%%%%%%%%%%%%%%%%%%%%%%
%%%%%%%%%%%%%%%%%%%%%%%%%%%%%%%%%%%%%%%
\newpage
%\setcounter{page}{1}
\section*{\large Семинар 2: Абстрактные группы}
\setcounter{prn}{0}
%%%%%%%%%%%%%%%%%%%%%%%%%%%%%%%%%%%%%%%

%%%%%%%%%%%%%%%
\addtocounter{prn}{1}
\vspace{0.3cm}
\paragraph*{Задача \theprn: }
% Sadovsky, Arshinov
%%%%%%%%%%%%%%%

Являются ли сложение, вычитание, умножение и деление бинарными операциями на множестве всех нечётных чисел?


\ifimportant
% only shown if \importanttrue is set
\medskip
\noindent

Ответ

\fi

%%%%%%%%%%%%%%%
\addtocounter{prn}{1}
\vspace{0.3cm}
\paragraph*{Задача \theprn: }
% Sadovsky, Arshinov
%%%%%%%%%%%%%%%

Покажите, что положительные рациональные числа с операцией умножения образуют группу. (мультипликативная группа положительных рациональных чисел).


\ifimportant
% only shown if \importanttrue is set
\medskip
\noindent

Ответ

\fi

%%%%%%%%%%%%%%%
\addtocounter{prn}{1}
\vspace{0.3cm}
\paragraph*{Задача \theprn: }
% Sadovsky, Arshinov
%%%%%%%%%%%%%%%

Является ли группой множество рациональных чисел с операцией умножения?


\ifimportant
% only shown if \importanttrue is set
\medskip
\noindent

Ответ

\fi

%%%%%%%%%%%%%%%
\addtocounter{prn}{1}
\vspace{0.3cm}
\paragraph*{Задача \theprn: }
% 
%%%%%%%%%%%%%%%

Рассмотрите множество целых чисел $\{1,2,3,\ldots, n\}$ с операцией сложения по модулю $n$. Является ли это множество группой для $n=7$? любого целого $n$? Можете ли вы придумать геометрическую реализацию этой группы?


\ifimportant
% only shown if \importanttrue is set
\medskip
\noindent

Ответ

\fi

%%%%%%%%%%%%%%%
\addtocounter{prn}{1}
\vspace{0.3cm}
\paragraph*{Задача \theprn: }
%
%%%%%%%%%%%%%%%

Показать, что группа симметрий правильного треугольника изоморфна группе перестановок $S_{3}$. 

\ifimportant
% only shown if \importanttrue is set
\medskip
\noindent
Решение
Сравнить таблицы умножений двух групп.
\fi


%%%%%%%%%%%%%%%
\addtocounter{prn}{1}
\vspace{0.3cm}
\paragraph*{Задача \theprn: }
%
%%%%%%%%%%%%%%%

Установите соответствие между преобразованиями симметрии квадрата и перестановками его вершин. Убедитесь, что не каждой перестановке соответствует преобразование симметрии квадрата. Сравните порядки (число элементов) группы симметрии квадрата и группы перестановок $S_{4}$.

\ifimportant
% only shown if \importanttrue is set
\medskip
\noindent
Решение

\fi



%%%%%%%%%%%%%%%
\addtocounter{prn}{1}
\vspace{0.3cm}
\paragraph*{Задача \theprn: }
%problems.ru Задача 108413
%%%%%%%%%%%%%%%

Фабрика игрушек выпускает проволочные кубики, в вершинах которых расположены маленькие разноцветные шарики. По ГОСТу в каждом кубике должны быть использованы шарики всех восьми цветов (белого и семи цветов радуги). Сколько разных моделей кубиков может выпускать фабрика?

\ifimportant
% only shown if \importanttrue is set
\medskip
\noindent
Решение 1
Если кубик зафиксировать, то поместить 8 разных шариков в его вершины можно 8! способами. Но кубик можно поворачивать: каждую из шести его граней можно сделать нижней и поставить на нее 4 способами. Поэтому каждому кубику соответствуют  6·4 = 24  "раскраски", и общее число моделей равно  8! : 24 = 8·7·6·5 = 1680.

Решение 2
Сначала "наклеим" белый шарик. Повернем кубик так, чтобы белый шарик оказался в левом нижнем переднем углу. Теперь выберем 3 шарика для соседних вершин (это можно сделать $C_{7}^{3}$  способами). "Наклеим" один из них и повернем кубик так, чтобы это шарик оказался в правом нижнем переднем углу (а белый остался на месте). Теперь есть два способа наклеить отбранные два шарика в оставшиеся две вершины, соседние с белой. И в каждом из них еще 4! способов "наклеить" оставшиеся 4 шарика на оставшиеся 4 вершины. Всего $C_{7}^{3}\cdot 2\cdot 4!=7\cdot 6\cdot 5\cdot 2\cdot 4$     моделей.

Ответ
1680 моделей.
\fi

%%%%%%%%%%%%%%%
\addtocounter{prn}{1}
\vspace{0.3cm}
\paragraph*{Задача \theprn: }
%problems.ru Задача 60743
%%%%%%%%%%%%%%%

$p$ – простое число. Сколько существует способов раскрасить вершины правильного $p$-угольника в $a$ цветов? (Раскраски, которые можно совместить поворотом, считаются одинаковыми.)

\ifimportant
% only shown if \importanttrue is set
\medskip
\noindent
Решение
Забудем временно про совмещение раскрасок поворотами. Тогда p вершин можно раскрасить $a^{p}$ способами (см. задачу 60348). Среди этих раскрасок есть a одноцветных. Каждая из оставшихся совмещается с p раскрасками (считая исходную). Поэтому различных неодноцветных раскрасок в p раз меньше:  $\frac{a^{p}-a}{p}$.

Ответ $\frac{a^{p}-a}{p}+a$.
  способов.
Замечания
1. Из этого результата немедленно следует Малая теорема Ферма (см. задачу 60736).
2. Подумайте, почему важна простота числа p.
\fi



%%%%%%%%%%%%%%%
\addtocounter{prn}{1}
\vspace{0.3cm}
\paragraph*{*Задача \theprn: }
%problems.ru Задача 35240
%%%%%%%%%%%%%%%

К кубику Рубика применили последовательность поворотов. Доказать, что применяя ее несколько раз, можно привести кубик в начальное состояние. 

\ifimportant
% only shown if \importanttrue is set
\medskip
\noindent
Подсказка
Число состояний кубика Рубика конечно; для каждого поворота есть обратный. 
Решение
Обозначим начальное состояние кубика Рубика за A. Пусть P=P1P2...Pn - некоторая последовательность поворотов. Обозначим через P(X) результат применения последовательности поворотов P к состоянию X, и через Pm(X) результат m-кратного применения последовательности поворотов P к состоянию X. Рассмотрим последовательность состояний A, P(A), P2(A), P3(A), ... Поскольку имеется лишь конечное число состояний кубика Рубика, то в этой последовательности встретится повторение, т.е. Pk(A)=Pn(A)=B для некоторых k, n, k<n. Для каждого поворота Pi кубика есть обратный поворот P-1i (т.е. такой поворот, что P-1i(Pi) = Pi(P-1i) есть тождественное преобразование). Таким образом, для последовательности поворотов P=P1P2...Pn имеется обратное преобразование P-1, определяемое как последовательное выполнение поворотов P-1n, P-1n-1, ... , P-11. Применяя преобразование P-1 к состоянию B=Pk(A)=Pn(A), мы получаем, что P-1(B)=Pk-1(A)=Pn-1(A). Проводя дальнейшие рассуждения подобным образом, мы получим совпадение состояний Pk-2(A)=Pn-2(A), ... , P1(A)=Pn-k+1(A). Таким образом, начальное состояние повторится после (n-k+1)-кратного выполнения последовательности поворотов P.
\fi











%%%%%%%%%%%%%%%%%%%%%%%%%%%%%%%%%%%%%%%
%%%%%%%%%%%%%%%%%%%%%%%%%%%%%%%%%%%%%%%
\newpage
%\setcounter{page}{1}
\section*{\large Лекция 3: Группы и попытки их классификации}
%OLDS 1.5
\setcounter{prn}{0}
%%%%%%%%%%%%%%%%%%%%%%%%%%%%%%%%%%%%%%%

\begin{enumerate}
	\item Теорема Лагранжа и её применения
	\begin{enumerate}
		\item Порядок групп симметрий многогранников
		\item Делимость некоторых степеней натуральных чисел
	\end{enumerate}
	\item Теорема Кэли
	\item Классификация конечных простых групп
\end{enumerate}

%%%%%%%%%%%%%%%%%%%%%%%%%%%%%%%%%%%%%%%
%%%%%%%%%%%%%%%%%%%%%%%%%%%%%%%%%%%%%%%
\newpage
%\setcounter{page}{1}
\section*{\large Семинар 3: Разные задачи.}
\setcounter{prn}{0}
%%%%%%%%%%%%%%%%%%%%%%%%%%%%%%%%%%%%%%%

%%%%%%%%%%%%%%%
\addtocounter{prn}{1}
\vspace{0.3cm}
\paragraph*{Задача \theprn: }
%
%%%%%%%%%%%%%%%

Найти группу симметрии прямоугольного параллелепипеда вершины которого раскрашены (через одну) в синий и красный цвета. Каков порядок этой группы? Составить таблицу умножения для этой группы. Можете ли вы найти изоморфизм этой группы с подгруппой группы перестановок?


%\ifimportant
%% only shown if \importanttrue is set
%\medskip
%\noindent
%Ответ: 48 
%\fi

%%%%%%%%%%%%%%%
\addtocounter{prn}{1}
\vspace{0.3cm}
\paragraph*{Задача \theprn: }
%
%%%%%%%%%%%%%%%

Разложить на циклы перестановку
$$
	\alpha = \left(
	\begin{array}{ccccccccc}
		1 &  2 & 3 & 4 & 5 & 6 & 7 & 8 & 9 \\
		2 & 9 & 5 & 3 & 1 & 8 & 7 & 4 & 6 
	\end{array}
	\right)
$$
Найти порядок этой  перестановки и обратную перестановку. Найти произведение $\alpha * \beta$, где $\beta = (3,6,7)$.


%\ifimportant
%% only shown if \importanttrue is set
%\medskip
%\noindent
%Ответ: 48 
%\fi




%%%%%%%%%%%%%%%
\addtocounter{prn}{1}
\vspace{0.3cm}
\paragraph*{Задача \theprn: }
%problems.ru 
%%%%%%%%%%%%%%%

Найдите порядок группы симметрий куба и убедитесь в справедливости теоремы Лагранжа, рассматривая различные элементы этой группы.


\ifimportant
% only shown if \importanttrue is set
\medskip
\noindent
Ответ: 48 
\fi


%%%%%%%%%%%%%%%
\addtocounter{prn}{1}
\vspace{0.3cm}
\paragraph*{Задача \theprn: }
%problems.ru Задача 97936
%%%%%%%%%%%%%%%

Авторы: Шнирельман А., Константинов Н.Н.
В некотором городе разрешаются только парные обмены квартир (если две семьи обмениваются квартирами, то в тот же день они не имеют права участвовать в другом обмене). Докажите, что любой сложный обмен квартирами можно осуществить за два дня. 
(Предполагается, что при любых обменах каждая семья как до, так и после обмена занимает одну квартиру, и что семьи при этом сохраняются).


\ifimportant
% only shown if \importanttrue is set
\medskip
\noindent
Решение
Сложный обмен квартир представляет собой цикл. Представим его в виде поворота правильного многоугольника, вершины которого соответствуют участвующим в обмене квартирам. Этот поворот есть композиция двух осевых симметрий (относительно серединных перпендикуляров к двум соседним сторонам). Каждая осевая симметрия задает несколько парных обменам; все их можно осуществить за один день.
Замечания
1. 5 баллов.
2. Задача предлагалась также на 53-й Ленинградской олимпиаде (1987, 8 кл., зад. 4).
\fi





%%%%%%%%%%%%%%%
\addtocounter{prn}{1}
\vspace{0.3cm}
\paragraph*{Задача \theprn: }
%problems.ru Задача 35604
%%%%%%%%%%%%%%%

Некоторый текст зашифровали, поставив в соответствие каждой букве некоторую (возможно, ту же самую букву) букву так, что текст можно однозначно расшифровать. Докажите, что найдется такое число N, что после N-кратного применения шифрования заведомо получится исходный текст. Найдите из всех таких значений N наименьшее, годящееся для всех шифров (при условии, что в алфавите 33 буквы). (Задача с сайта www.cryptography.ru.) 


\ifimportant
% only shown if \importanttrue is set
\medskip
\noindent
Подсказка
Если буква a1 шифруется в букву a2, буква a2 шифруется в букву a3, ... , буква ak шифруется в букву a1, то при k-кратном шифровании мы получим исходный текст. 
Решение
По условию шифрование допускает однозначную расшифровку. Это означает, что шифрование - это просто некоторая перестановка 33 букв алфавита, (т.е. в каждую из букв некоторая буква шифруется). Пусть буква a1 шифруется в букву a2, буква a2 шифруется в букву a3, и т.д. , буква ak шифруется в букву ak+1, ... В последовательности букв a1, a2, ... в некоторый момент возникнет первое повторение, т.е. буква ak+1 совпадет с одной из букв a1, a2, ... , ak, а буквы a1, a2, ... , ak - различны. Но ak+1 не может совпасть с одной из букв a2, ... , ak, поскольку в них шифруются соответственно a1, a2, ... , ak-1, а в ak+1 шифруется ak. Таким образом, ak+1=a1. Итак, мы имеем следующий цикл: a1 шифруется в букву a2, буква a2 шифруется в букву a3, ... , буква ak шифруется в букву a1. При k-кратном шифровании буквы a1, a2, ... , ak будут стоять на тех же местах, как и в исходном тексте. Вся перестановка букв тем самым разбилась на циклы. Длина цикла может быть от 1 до 33. Если взять N, делящееся на каждое из чисел 1, 2, ... , 33, и сделать N-кратное шифрование, то буквы каждого цикла будут стоять на своих местах, т.е. мы получим исходный текст. Наоборот, если N не делится на одно из чисел k от 1 до 33, то в случае, если шифрование содержит цикл длины k, после N-кратного применения шифрования мы не получим исходный текст. Итак, искомое число N - наименьшее общее кратное чисел 1, 2, ... , 33. Это число огромно, оно равно 144403552893600. 
Ответ
наименьшее искомое число - 144403552893600.
\fi

%%%%%%%%%%%%%%%
\addtocounter{prn}{1}
\vspace{0.3cm}
\paragraph*{Задача \theprn: }
%problems.ru Задача 35293
%%%%%%%%%%%%%%%

Комбинация А поворотов кубика Рубика называется порождающей, если среди результатов многократного применения комбинации А встретятся всевозможные состояния, в которые можно перевести кубик Рубика при помощи поворотов. Существует ли порождающая комбинация поворотов? 


\ifimportant
% only shown if \importanttrue is set
\medskip
\noindent
Подсказка
Если бы существовала порождающая комбинация поворотов, то для любых двух поворотов результат их последовательного применения не зависел бы от порядка выполнения этих поворотов. 
Решение
Предположим противное. Обозначим за А порождающую комбинацию, а за X начальное состояние кубика. Тогда в последовательности X, A(X), A(A(X)), ... встретятся все состояния кубика. Возьмем два простых поворота кубика: P - поворот правой грани, Q - поворот верхней грани. Применим поворот P к состоянию X, получим состояние P(X). Согласно нашему предположению оно совпадает с некоторым состоянием вида A(A(...(X)...), т.е. P(X)=Am(X) для некоторого натурального m. Таким же образом, Q(X)=An(X) для некоторого натурального n. Тогда P(Q(X))=Q(P(X))=Am+n(X). Это означает, что результат последовательного выполнения поворотов P и Q не зависит от порядка выполнения этих поворотов. Однако нетрудно видеть, что результат последовательного выполнения поворотов P и Q отличается от результата последовательного выполнения поворотов Q и P. Противоречие. На языке теории групп эту задачу можно сформулировать следующим образом: верно ли, что группа комбинаций поворотов кубика Рубика циклична? В решении показывается, что эта группа не коммутативна, а следовательно, не циклична. 
Ответ
не существует.
\fi


%%%%%%%%%%%%%%%
\addtocounter{prn}{1}
\vspace{0.3cm}
\paragraph*{Задача \theprn: }
%problems.ru Задача 78294
%%%%%%%%%%%%%%%

В окружность вписан неправильный $n$-угольник, который при повороте окружности около центра на некоторый угол $\alpha\neq 2\pi$   совмещается сам с собой. Доказать, что $n$ – число составное.


\ifimportant
% only shown if \importanttrue is set
\medskip
\noindent
Решение
Предположим, что n  – простое число. Рассмотрим орбиту каждой вершины, т. е. множество точек, в которые переходит вершина при поворотах, переводящих n-угольник в себя. Все орбиты состоят из одного количества вершин – их столько, сколько геометрически различных углов поворота, совмещающих n-угольник с собой. Из этого следует, что орбита одна, поскольку n – простое. Но тогда каждые две вершины можно совместить поворотом, а значит, все стороны и все углы многоугольника равны, т. е. n-угольник – правильный. Противоречие.
\fi



%%%%%%%%%%%%%%%
\addtocounter{prn}{1}
\vspace{0.3cm}
\paragraph*{*Задача \theprn: }
%problems.ru Задача 97936
%%%%%%%%%%%%%%%

Автор: Гринберг Н.
На пульте имеется несколько кнопок, с помощью которых осуществляется управление световым табло. После нажатия любой кнопки некоторые лампочки на табло переключаются (для каждой кнопки есть свой набор лампочек, причём наборы могут пересекаться). Доказать, что число состояний, в которых может находиться табло, равно некоторой степени числа 2.


\ifimportant
% only shown if \importanttrue is set
\medskip
\noindent
Решение
   Пусть число лампочек равно N, количество кнопок — n. Обозначим через Ai подмножество лампочек, которое переключает i-я кнопка, то есть узор, который получается из начального состояния B0 — "все лампочки не горят" — при нажатии i-й кнопки; мы будем также говорить об "операции Ai", понимая под этим нажатие i-й кнопки. Приведем несколько способов доказательства утверждения задачи.
   Первый способ. Всего существует 2N различных узоров светового табло, потому что каждая из N лампочек может находиться в двух состояниях — гореть или не гореть. Пусть нажатием кнопок из начального узора B0 можно получить m разных узоров  B0, B1,..., Bm–1.
Тогда из любого начального узора X можно получить такое же количество узоров — это те узоры, которые отличаются от X на множествах
B0, ..., Bm–1.  Разобьем все 2N узоров на несколько классов: отнесем к одному классу те узоры, которые можно получить друг из друга нажатием кнопок. В каждом классе будет по m узоров. Поэтому 2N делится на m; следовательно, m — степень двойки.
   Второй способ. Пусть кнопки занумерованы так, что ни один из s узоров  A1, A2, ..., As  нельзя получить комбинацией предыдущих, а каждый из остальных  $k- s$  узоров  As+1, ..., Ak  можно получить комбинацией некоторых из операций  A1, A2, ..., As.  Тогда общее число узоров m, которое можно получить из начального состояния B0, равно 2s. В самом деле, все комбинации операций  A1, ..., As,  соответствующие 2s различным подмножествам множества  {1, 2,..., s},  дают различные узоры: если бы какие-то два из них совпадали:
Ai1Ai2...Aim = Aj1Aj2...Ajl,
то узор с наибольшим из входящих в это равенство номеров можно было бы получить комбинацией предыдущих. Например, если  AB = CD,  то   CAB = CCD = D   (двойное нажатие на кнопку не меняет состояния табло).
   Третий способ. Посмотрим вначале не на все N лампочек, а на первые L из них. Пусть на этих лампочках из начального состояния B0 можно получить mL различных узоров. Затем посмотрим на (L+1)-ю лампочку. Если состояние (L+1)-й лампочки вполне определяется набором состояний первых L лампочек, то  mL+1 = mL.  Если же хотя бы два узора совпадают на первых L лампочках, но отличаются на (L+1)-й, то можно получить узор, в котором первые L лампочек не горят, а (L+1)-я горит. Тогда из каждого узора наших L лампочек можно получить два различных узора из первых  L + 1  лампочек, то  mL+1 = 2mL.  Поскольку одна (первая) лампочка может находиться в двух состояниях, ясно, что интересующее нас  m = mN  будет степенью двойки.

Также доступны документы в формате TeX 
Замечания
См. также задачу 74200.
Для знатоков. Фактически все решения основаны на следующей общей идее. На множестве P всех подмножеств N-элементного множества M зададим операцию  $A \bigtriangleup B = (A \cup B) \setminus (A \cap B)$  (симметрическая разность), превращающую P в группу. Это следующим образом соответствует нашей задаче: если горят лампочки из B, и мы меняем состояние всех лампочек из A, то получится узор  $A \bigtriangleup B$.  Состояния, в которых может находиться табло, — это элементы подгруппы, порожденной множествами  A1, ..., An.  По теореме Лагранжа порядок группы (а он равен 2N) делится на порядок подгруппы.
\fi






%%%%%%%%%%%%%%%%%%%%%%%%%%%%%%%%%%%%%%
%%%%%%%%%%%%%%%%%%%%%%%%%%%%%%%%%%%%%%
\newpage
\setcounter{page}{1}
\mbox \\
\vspace{-0.5cm}
\section*{\Large Загоночная Контрольная} 
\begin{center}
{\Large по курсу: \\ \vspace{0.1cm}
<<Группы и Симметрии>> \\ \vspace{0.1cm}
У3, лектор: Саша Абанов}
\end{center}
\setcounter{prn}{0}
%%%%%%%%%%%%%%%%%%%%%%%%%%%%%%%%%%%%%%

\mbox{} \\
\vspace{-9cm}

{\bf \Large ИМЯ:  \hspace{7cm} КОМАНДА: }
\vspace{7.0cm}


%%%%%%%%%%%%%%
\addtocounter{prn}{1}
\vspace{0.3cm}
\paragraph*{\large Задача \theprn: }
%%%%%%%%%%%%%%
\mbox{}
\\
а) Найдите группу симметрии правильного шестиугольника вершины которого раскрашены (через одну) в синий и красный цвета.  
\\
б) Каков порядок этой группы?
\\
в) Составьте таблицу умножения для этой группы. 
\\
г) Можете ли вы найти изоморфизм этой группы с подгруппой некоторой группы перестановок?

\begin{center}
{\large Решение: }
\end{center}

\mbox{}
\vspace{8cm}


\newpage
%%%%%%%%%%%%%%%
\addtocounter{prn}{1}
\vspace{0.3cm}
\paragraph*{\large Задача \theprn: }
%%%%%%%%%%%%%%%
Разложить на циклы перестановку
$$
	\alpha = \left(
	\begin{array}{ccccccccc}
		1 &  2 & 3 & 4 & 5 & 6 & 7 & 8 & 9 \\
		3 & 5 & 7 & 8 & 4 & 9 & 1 & 2 & 6 
	\end{array}
	\right)
$$
Найти порядок этой  перестановки и обратную перестановку. Найти произведение $\beta *\alpha $, где $\beta = (2,7,8)$.

\medskip 
\textit{Примечание:} ответ для обратной перестановки и для произведения запишите в виде разложения по циклам.



\begin{center}
{\large Решение: }
\end{center}


\vspace{6cm}
%\newpage


%%%%%%%%%%%%%%%
\addtocounter{prn}{1}
\vspace{0.3cm}
\paragraph*{*\large Задача \theprn $\;$ [задача вне зачёта]: }
%%%%%%%%%%%%%%%

Орнамент, показанный на рисунке, использовался ещё древними Египтянами. Идентифицируйте преобразования симметрии сохраняющие рисунок (предполагается, что рисунок занимает всю бесконечную плоскость).

\centerline{\includegraphics[width=2in]{p4m-5.jpg}}

\begin{center}
{\large Решение: }
\end{center}


\vspace{6cm}
%\newpage



%%%%%%%%%%%%%%%%%%%%%%%%%%
%%%%%%%%%%%%%%%%%%%%%%%%%%
\end{document}
%%%%%%%%%%%%%%%%%%%%%%%%%%
%%%%%%%%%%%%%%%%%%%%%%%%%%


%%%%%%%%%%%%%%%
\addtocounter{prn}{1}
\subsection*{Задача \theprn }
%%%%%%%%%%%%%%%




