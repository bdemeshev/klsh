\documentclass[12pt]{article} % размер шрифта

\usepackage{etex} % боремся с tex capacity exceeded
\usepackage{tikz} % картинки в tikz
\usepackage{microtype} % свешивание пунктуации

\usepackage{array} % для столбцов фиксированной ширины

\usepackage{indentfirst} % отступ в первом параграфе

\usepackage{sectsty} % для центрирования названий частей
\allsectionsfont{\centering} % приказываем центрировать все sections

\usepackage{amsmath} % куча стандартных математических плюшек
\usepackage{amssymb} % куча стандартных математических плюшек

\usepackage[top=1.5cm, left=1.3cm, right=1.3cm, bottom=1.5cm]{geometry} % размер текста на странице

\usepackage{lastpage} % чтобы узнать номер последней страницы

\usepackage{enumitem} % дополнительные плюшки для списков
%  например \begin{enumerate}[resume] позволяет продолжить нумерацию в новом списке
\usepackage{caption} % подписи к картинкам без плавающего окружения figure


\usepackage{fancyhdr} % весёлые колонтитулы
\pagestyle{fancy}
\lhead{Загоночная. Теория групп.}
\chead{}
\rhead{КЛШ-42}
\lfoot{}
\cfoot{}
\rfoot{\thepage/\pageref{LastPage}}
\renewcommand{\headrulewidth}{0.4pt}
\renewcommand{\footrulewidth}{0.4pt}

\usepackage{url} % для вставки гиперссылок

\usepackage{todonotes} % для вставки в документ заметок о том, что осталось сделать
% \todo{Здесь надо коэффициенты исправить}
% \missingfigure{Здесь будет картина Последний день Помпеи}
% команда \listoftodos — печатает все поставленные \todo'шки

\usepackage{booktabs} % красивые таблицы
% заповеди из документации:
% 1. Не используйвертикальные линии
% 2. Не используйдвойные линии
% 3. Единицы измерения помещайв шапку таблицы
% 4. Не сокращай.1 вместо 0.1
% 5. Повторяющееся значение повторяйте, а не говори"то же"

\usepackage{fontspec} % поддержка разных шрифтов
\usepackage{polyglossia} % поддержка разных языков

\setmainlanguage{russian}
\setotherlanguages{english}

\setmainfont{Linux Libertine O} % выбираем шрифт
% можно также попробовать Helvetica, Arial, Cambria и т.Д.

% чтобы использовать шрифт Linux Libertine на личном компе,
% его надо предварительно скачать по ссылке
% http://www.linuxlibertine.org/index.php?id=91&L=1

\newfontfamily{\cyrillicfonttt}{Linux Libertine O}
% пояснение зачем нужно шаманство с \newfontfamily
% http://tex.stackexchange.com/questions/91507/

\AddEnumerateCounter{\asbuk}{\russian@alph}{щ} % для списков с русскими буквами
\setlist[enumerate, 2]{label=\asbuk*),ref=\asbuk*} % списки уровня 2 будут буквами а) б) ...

%% эконометрические и вероятностные сокращения
\DeclareMathOperator{\Cov}{Cov}
\DeclareMathOperator{\Corr}{Corr}
\DeclareMathOperator{\Var}{Var}
\DeclareMathOperator{\E}{E}
\def \hb{\hat{\beta}}
\def \hs{\hat{\sigma}}
\def \htheta{\hat{\theta}}
\def \s{\sigma}
\def \hy{\hat{y}}
\def \hY{\hat{Y}}
\def \v1{\vec{1}}
\def \e{\varepsilon}
\def \he{\hat{\e}}
\def \z{z}
\def \hVar{\widehat{\Var}}
\def \hCorr{\widehat{\Corr}}
\def \hCov{\widehat{\Cov}}
\def \cN{\mathcal{N}}

\def \ZZ{\mathbb{Z}}


\begin{document}

\begin{enumerate}
\item Рассмотрим группу $S_{9}$ (пересадки школьников по стульям). Возведи в 42-ую степень перестановку $(1234)(795)(86)$.

\item Рассмотрим все правильные пирамидки. Каждую \textbf{вершину} пирамидки разрешено красить в один из 7 цветов. Сколько существует принципиально различных (не совмещаемых вращением) пирамидок?

\item Рассмотрим группу $S_9$. Нарисуй граф Кэли подгруппы с образующими $a=(123)$ и $b=(1234)$.

\end{enumerate}

\vspace{1cm}

\begin{enumerate}
\item Рассмотрим группу $S_{9}$ (пересадки школьников по стульям). Возведи в 42-ую степень перестановку $(1234)(795)(86)$.

\item Рассмотрим все правильные пирамидки. Каждую \textbf{вершину} пирамидки разрешено красить в один из 7 цветов. Сколько существует принципиально различных (не совмещаемых вращением) пирамидок?

\item Рассмотрим группу $S_9$. Нарисуй граф Кэли подгруппы с образующими $a=(123)$ и $b=(1234)$.

\end{enumerate}

\vspace{1cm}

\begin{enumerate}
\item Рассмотрим группу $S_{9}$ (пересадки школьников по стульям). Возведи в 42-ую степень перестановку $(1234)(795)(86)$.

\item Рассмотрим все правильные пирамидки. Каждую \textbf{вершину} пирамидки разрешено красить в один из 7 цветов. Сколько существует принципиально различных (не совмещаемых вращением) пирамидок?

\item Рассмотрим группу $S_9$. Нарисуй граф Кэли подгруппы с образующими $a=(123)$ и $b=(1234)$.

\end{enumerate}

\vspace{1cm}


\begin{enumerate}
\item Рассмотрим группу $S_{9}$ (пересадки школьников по стульям). Возведи в 42-ую степень перестановку $(1234)(795)(86)$.

\item Рассмотрим все правильные пирамидки. Каждую \textbf{вершину} пирамидки разрешено красить в один из 7 цветов. Сколько существует принципиально различных (не совмещаемых вращением) пирамидок?

\item Рассмотрим группу $S_9$. Нарисуй граф Кэли подгруппы с образующими $a=(123)$ и $b=(1234)$.

\end{enumerate}

\vspace{1cm}


\begin{enumerate}
\item Рассмотрим группу $S_{9}$ (пересадки школьников по стульям). Возведи в 42-ую степень перестановку $(1234)(795)(86)$.

\item Рассмотрим все правильные пирамидки. Каждую \textbf{вершину} пирамидки разрешено красить в один из 7 цветов. Сколько существует принципиально различных (не совмещаемых вращением) пирамидок?

\item Рассмотрим группу $S_9$. Нарисуй граф Кэли подгруппы с образующими $a=(123)$ и $b=(1234)$.

\end{enumerate}



\end{document}
