\documentclass[a4paper, 12pt]{article}


\usepackage{mathrsfs}

\usepackage{amscd}
\usepackage[paper=a4paper,
top=2cm, bottom=2cm, left=2cm, right=2cm, includefoot]{geometry} % размер листа бумаги


\usepackage{tikz} % картинки в tikz
\usepackage{microtype} % свешивание пунктуации

\usepackage{floatrow} % для выравнивания рисунка и подписи
\usepackage{caption} % для пустых подписей

\usepackage{array} % для столбцов фиксированной ширины

\usepackage{indentfirst} % отступ в первом параграфе

\usepackage{sectsty} % для центрирования названий частей
\allsectionsfont{\centering}

\usepackage{amsmath, amsfonts} % куча стандартных математических плюшек

\usepackage{comment} % для комментариев

\usepackage{multicol} % текст в несколько колонок

\usepackage{lastpage} % чтобы узнать номер последней страницы

\usepackage{enumitem} % дополнительные плюшки для списков
%  например \begin{enumerate}[resume] позволяет продолжить нумерацию в новом списке

\usepackage{booktabs}

\usepackage{url} % для вставки интернет-ссылок

\usepackage{fontspec}
\usepackage{polyglossia}

\setmainlanguage{russian}
\setotherlanguages{english}

% download "Linux Libertine" fonts:
% http://www.linuxlibertine.org/index.php?id=91&L=1
\setmainfont{Linux Libertine O} % or Helvetica, Arial, Cambria
% why do we need \newfontfamily:
% http://tex.stackexchange.com/questions/91507/
\newfontfamily{\cyrillicfonttt}{Linux Libertine O}

\AddEnumerateCounter{\asbuk}{\russian@alph}{щ} % для списков с русскими буквами
\setlist[enumerate, 2]{label=\asbuk*),ref=\asbuk*}

\DeclareMathOperator{\Var}{Var}
\DeclareMathOperator{\E}{\mathbb{E}}

\let\P\relax
\DeclareMathOperator{\P}{\mathbb{P}}
\def\cN{\mathcal{N}}

\usepackage{fancyhdr} % весёлые колонтитулы
\pagestyle{fancy}
\lhead{Теория вероятностей}
\chead{}
\rhead{КЛШ-42}
\lfoot{}
\cfoot{}
\rfoot{\thepage/\pageref{LastPage}}
\renewcommand{\headrulewidth}{0.4pt}
\renewcommand{\footrulewidth}{0.4pt}


\usepackage[bibencoding = auto, backend = biber,
sorting = none]{biblatex}

\addbibresource{../../klsh.bib}




\begin{document}

\section{Анонс курса}

Разработка механизмов

Аукцион какой формы выгоднее для продавца? Какие бывают аукционы?  Какие механизмы используются для распределения школьников по государственным школам, медиков-интернов по больницам и сведения доноров почек с больными, ждущими пересадку? Как правильно поделить расходы на общественное благо? И что значит «правильно»?

С помощью раздела теории игр под названием «разработка механизмов» (mechanism design) мы попытаемся ответить на эти вопросы.

Требование к участникам: уметь строить графики.

Готов к численности: 10-20 человек.

\section{Знакомство с табличками и деревьями}

Включая меня нас 9 человек.

Табличка для 9 участников: пол и любимое мороженое. $P(A)$, $P(A\cap B)$, $P(A\cup B)$.
Подчёркиваю в чём состоит случайный эксперимент: одного человека выбираем наугад.

Понятие условное вероятности. Находим $P(A|B)$.

Двушаговый эксперимент. Те же 9 человек, считая меня. Выбираем двоих по очереди. Сначала записываем букву пола первого человека, а у второго выясняем предпочтение по мороженому. С деревом неожиданно сложно решать без дополнительного деления ветвей. Мы не вводили дополнительное деление, а делили ветви первого уровня по полу, и ветви второго уровня по мороженому :) С помощью таблички дозаполняем дерево. Да и табличкой выходит неочевидно: в табличку заполяются не люди, а варианты выгнать двоих по очереди, $9\cdot 8 = 72$. Находим условную вероятность.

Рассматриваем пример с киданием маркера в доску и выводим, что
\[
P(A|B) = \frac{S_{A\cap B}}{S_B}
\]

Были все. Немного сложный пример для знакомства с деревом. Лучше было у второго тоже пол записывать.


\newpage
\begin{enumerate}

\item  В городе примерно 4\% такси зелёного цвета и остальные жёлтые. Свидетель путает цвет на показаниях в суде с вероятностью 10\%.

\begin{enumerate}
  \item Какова вероятность того, свидетель скажет, что видел зелёное такси?
  \item Какова вероятность того, свидетель ошибётся?
  \item Какова вероятность того, что такси было зелёным, если свидетель говорит, что оно было зелёным?
  \item Какова вероятность того, что такси было жёлтым, если свидетель говорит, что оно было жёлтым?
\end{enumerate}


\item У тети Маши — двое детей, один старше другого. Предположим, что вероятности рождения мальчика и девочки равны и не зависят от дня недели, а пол первого и второго ребенка независимы. Для каждой из ситуаций найдите условную вероятность того, что у тёти Маши есть дети обоих полов.
\begin{enumerate}
  \item Известно, что старший ребенок — мальчик.
  \item Тетя Маша наугад выбирает одного своего
ребенка и посылает к тете Оле, вернуть метлу. Это оказывается мальчик.
  \item На вопрос: «А правда ли тётя Маша, что у Вас есть хотя бы один сын?» тётя Маша ответила: «Да».
  \item На вопрос: «А правда ли тётя Маша, что у Вас есть хотя бы один сын, родившийся в пятницу?» тётя Маша ответила: «Да».
\end{enumerate}

\item Ты смертельно болен. Спасти тебя может только один вид  целебной лягушки. Целебны у этого вида только самцы. Самцы и самки встречаются равновероятно. Ты на дороге и предельно ослаб. Слева в 100 метрах от тебя одна лягушка целебного вида, но не ясно, самец или самка. Справа в 100 метров аж две лягушки целебного вида, но тоже издалека неясно кто. От двух лягушек в твою сторону дует ветер, поэтому ты можешь их слышать.

В какую сторону стоит ползти из последних сил в каждой из  ситуаций?
\begin{enumerate}
  \item Cамцы и самки квакают одинаково, со стороны правых двух лягушек ты слышишь кваканье.
  \item Самки квакают, самцы — нет, со стороны правых двух лягушек ты слышишь кваканье, но не разобрать, одной лягушки или двух.
  \item Самцы и самки квакают по разному, но одинаково часто. Ты слышишь отдельный квак одной из двух лягушек справа и это квак самки.
\end{enumerate}

\item Monty-Hall

Есть три закрытых двери. За двумя из них — по козе, за третьей автомобиль. Ты выбираешь одну из дверей. Допустим, ты выбрал дверь А. Ведущий шоу открывает дверь B и за ней нет автомобиля.
В этот момент ведущий предлагает тебе изменить выбор двери.

Имеет ли смысл изменить выбор в каждой из трёх ситуаций?
\begin{enumerate}
  \item Ведущий выбирал одну из трёх дверей равновероятно.
  \item Ведущий выбирал одну из двух дверей не выбранных тобой равновероятно.
  \item Ведущий выбирал дверь без машины и не совпадающую с твоей.
\end{enumerate}

\end{enumerate}



\newpage
\section{Решаем задачи}

Решили задачу про такси с помощью дерева и таблички. Спросил предполагаемые ответы к задаче про тётю Машу. У части школьников есть путаница между $P(A\cap B)$ и $P(A|B)$.

\section{И снова задачи}

Решили задачу про тётю Машу с помощью дерева. Варианты кроме пятничного мальчика также решали интуитивно с помощью вычёркивания равновероятных вариантов.

\section{КНТ}

Разобрали задачи с КНТ:

Два стрелка делают одновременно по выстрелу. Оба попадают с вероятностью $0.54$, никто не попадает с вероятностью $0.04$. Какова вероятность попадания самого меткого стрелка?

Решали сначала с помощью таблички. В табличке ввели в одну ячейку неизвестную $x$. Получили квадратное уравнение на $x$. Затем с помощью дерева. Составили систему на две меткости, $a$ и $b$.
\[
\begin{cases}
ab=0.54 \\
(1-a)(1-b)=0.04 \\
\end{cases}
\]
Попробовали угадать решение. В таблице умножения 54 раскладывается на сомножители одним способом :)

Монетку подбрасывают 10 раз, $X$ — количество орлов. Найди $P(X=5)$.

Посчитали все случаи и вероятности для $n=2$.

Посчитали все случаи и вероятности для $n=3$.

Посчитали все случаи и вероятности для $n=4$.

Случаи располагали на доске в виде гистограммы.

Здесь бы надо было указать связь между количеством случаев при $n=4$ и $n=3$. Количество случаев при $n=4$ получается как сумма двух количеств случаев при $n=3$. Мы это позже сделали.

Увидели аналогию с разложением $(a+b)^n$.

Показал геометрический смысл $(a+b)^2$ и $(a+b)^3$ — с помощью маленьких кубиков разных цветов выложенных в куб 3 на 3.

Выписали треугольник Паскаля и ручным трудом нашли в нём нужный коэффициент.



\section{Математическое ожидание}

Идея:

Результат одного броска кубика плохо предсказуем.

Средний результат 100 бросков хорошо предсказуем.

Нас 9 человек (8 школьников + я), каждый проводит по 11 опытов.

Задача 1. Отдельный опыт: число очков на одном кубике.

Задание перед опытом: предскажите средний результат по 99 опытам.

Каждый делает 11 опытов с кубиками, считаем среднее. Получили 3.65.

Определили математическое ожидание.

Здесь хорошо было решить простую задачу на ожидание.

Задача 2. Отдельный опыт: число бросков до первого выпадения единицы.

Задание перед опытом: предскажите средний результат по 99 опытам.

Каждый делает 11 опытов с кубиками, считаем среднее. Получили 5.7.

Задача 3. Отдельный опыт: число кубиков выпавших на 2 при 9 подбрасываниях.

Задание перед опытом: предскажите средний результат по 99 опытам.

Каждый делает 11 опытов с кубиками, считаем среднее. Получили 1.51.

Вероятно, лучше было сначала решить задачу 3.

Маша хорошо предсказала последний результат (1.5) и объяснила потом почему.

\section{Непрерывные величины}

Вероятность для равномерного. Условная вероятность для равномерного.

Ожидание от равномерного, условное ожидание от равномерного.

Задача с параметром!

Начали задачу об аукционе двух человек :)

\section{Аукцион первой цены}

Решили задачу максимизации прибыли игрока на аукционе первой цены.

Выводы:

\begin{enumerate}
\item аукционы можно решать!;
\item не выгодно ставить больше ценности товара;
\item для двух игроков занижение ставки от цены составляет 50\%;
\end{enumerate}

\section{Доминируемые стратегии и аукцион второй цены}

Сначала пример с телефонами, один из которых по всем показателям лучшей (дизайн, возможности, цена).

Здесь я приводил пример с троллейбусами, но школьникам трудно было отличить здесь математическую модель от реальной ситуации.

Матрица для аукциона второй цены с конкретными числами. Решили используя доминирование стратегий.

\section{Дележка окружности}


Задача про дележку окружности. E и P.

Ожидаемая прибыль аукциониста на аукционе второй цены.

Задача о встрече.



\section{Прибыль на аукционе первой цены}

Вспоминали задачу про верёвку, так как Вани не было. Нашли прибыль на аукционе первой цены.

На примере дилеммы заключенного ещё раз рассмотрели доминирование стратегий. Разница оптимальность — равновесность.


\section{Марьяжи — мэтчинги}

Школьники предлагали мэтчинги для конкретной ситуации. Потом мы обсудили понятие устойчивости. Мэтчинг называется устойчивым, если нет пары, которая обоюдно хочет быть вместе наплевав на предложенный мэтчинг. Определили, какие мэтчинги являются устойчивыми.

Попробовали разработать алгоритм мэтчинга. Первый алгоритм мог подвисать, второй — нет. Алгоритм у нас вышел такой:

\begin{enumerate}
\item Женихи одновременно делают предложения своим двум лучшим невестам.
\item Каждая невеста выбирает лучшее сделанное ей предложение.
\item Если один жених при этом выбран несколькими невестами, то он выбирает среди них.
\item Сформированные пары отходят в сторону.
\item Начинаем алгоритм заново с меньшим количеством пар.
\end{enumerate}


\section{Алгоритм Гейла-Шепли}


Пробуем поймать предложенный алгоритм на неустойчивости. Первая попытка с тремя парами провалилась. Далее использовали идею «борьба сильных впустую». Пока три сильных жениха борятся за двух идеальных невест, самый плохой жених захватывает вполне хорошую невесту. Сильному жениху достаётся плохая невеста. В результате можно найти блокирующую пару: сильный жених — хорошая невеста.

Сформулировали алгоритм Гейла-Шепли. Пробуем алгоритм Гейла-Шепли на примере, подловившем наш алгоритм.

Проблемы с алгоритмом Гейла-Шепли:

\begin{enumerate}
\item долго;
\item нужно знать полную систему предпочтений;
\item нечестное декларирование (не разбирали).
\end{enumerate}




\section{Загоночная}

\begin{enumerate}
\item Ты играешь в игру. Есть три закрытых двери. За двумя из них — по козе, за третьей — автомобиль. Ведущий игры знает, что находится за дверьми, а  ты — нет. Ты показываешь на одну из дверей. Допустим, ты показал на дверь А. Чтобы поддержать интригу, ведущий открывает дверь с козой и не совпадающую с твоей. Пусть это будет дверь Б.

После открытия двери ведущий предлагает тебе изменить твой выбор. Стоит ли тебе перевыбрать дверь, если ты хочешь выиграть автомобиль?

\item  Есть четыре жениха (a, b, c, d) и четыре невесты (A, B, C, D). Их предпочтения таковы:

невеста A: $a > b > c > d$, жених $a: C>B>A>D$.

невеста B: $b > d > c > a$, жених $b: C>D>A>B$.

невеста C: $c > d > a > b$, жених $c: A>D>B>C$.

невеста D: $c > a > b > d$, жених $d: B>D>C>A$.

К каким парам сойдётся алгоритм Гейла-Шепли, если начинающей стороной являются женихи?


\end{enumerate}


\section{Источники мудрости}

Источники мудрости. Смело направляйте к ним верблюдов своего любопытства!

\nocite{shen2017veroiatnost, shen2007igri, brams2002delim}

\printbibliography

\url{http://econweb.ucsd.edu/~jsobel/172f12/matchingsolutions.pdf}



\end{document}
