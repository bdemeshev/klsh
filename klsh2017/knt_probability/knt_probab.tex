\documentclass[12pt]{article} % размер шрифта

\usepackage{etex} % боремся с tex capacity exceeded
\usepackage{tikz} % картинки в tikz
\usepackage{microtype} % свешивание пунктуации

\usepackage{array} % для столбцов фиксированной ширины

\usepackage{indentfirst} % отступ в первом параграфе

\usepackage{sectsty} % для центрирования названий частей
\allsectionsfont{\centering} % приказываем центрировать все sections

\usepackage{amsmath} % куча стандартных математических плюшек
\usepackage{amssymb} % куча стандартных математических плюшек

\usepackage[top=1.5cm, left=1.3cm, right=1.3cm, bottom=1.5cm]{geometry} % размер текста на странице

\usepackage{lastpage} % чтобы узнать номер последней страницы

\usepackage{enumitem} % дополнительные плюшки для списков
%  например \begin{enumerate}[resume] позволяет продолжить нумерацию в новом списке
\usepackage{caption} % подписи к картинкам без плавающего окружения figure


\usepackage{fancyhdr} % весёлые колонтитулы
\pagestyle{fancy}
\lhead{КНТ. Теория вероятностей.}
\chead{}
\rhead{КЛШ-42}
\lfoot{}
\cfoot{}
\rfoot{\thepage/\pageref{LastPage}}
\renewcommand{\headrulewidth}{0.4pt}
\renewcommand{\footrulewidth}{0.4pt}



\usepackage{todonotes} % для вставки в документ заметок о том, что осталось сделать
% \todo{Здесь надо коэффициенты исправить}
% \missingfigure{Здесь будет картина Последний день Помпеи}
% команда \listoftodos — печатает все поставленные \todo'шки

\usepackage{booktabs} % красивые таблицы
% заповеди из документации:
% 1. Не используйвертикальные линии
% 2. Не используйдвойные линии
% 3. Единицы измерения помещайв шапку таблицы
% 4. Не сокращай.1 вместо 0.1
% 5. Повторяющееся значение повторяйте, а не говори"то же"

\usepackage{fontspec} % поддержка разных шрифтов
\usepackage{polyglossia} % поддержка разных языков

\setmainlanguage{russian}
\setotherlanguages{english}

\setmainfont{Linux Libertine O} % выбираем шрифт
% можно также попробовать Helvetica, Arial, Cambria и т.Д.

% чтобы использовать шрифт Linux Libertine на личном компе,
% его надо предварительно скачать по ссылке
% http://www.linuxlibertine.org/index.php?id=91&L=1

\newfontfamily{\cyrillicfonttt}{Linux Libertine O}
% пояснение зачем нужно шаманство с \newfontfamily
% http://tex.stackexchange.com/questions/91507/

\AddEnumerateCounter{\asbuk}{\russian@alph}{щ} % для списков с русскими буквами
%\setlist[enumerate, 2]{label=\asbuk*),ref=\asbuk*} % списки уровня 2 будут буквами а) б) ...

\renewcommand{\labelenumii}{\theenumii}
\renewcommand{\theenumii}{\theenumi.\arabic{enumii}.}

%% эконометрические и вероятностные сокращения
\DeclareMathOperator{\Cov}{Cov}
\DeclareMathOperator{\Corr}{Corr}
\DeclareMathOperator{\Var}{Var}
\DeclareMathOperator{\E}{E}
\def \hb{\hat{\beta}}
\def \hs{\hat{\sigma}}
\def \htheta{\hat{\theta}}
\def \s{\sigma}
\def \hy{\hat{y}}
\def \hY{\hat{Y}}
\def \v1{\vec{1}}
\def \e{\varepsilon}
\def \he{\hat{\e}}
\def \z{z}
\def \hVar{\widehat{\Var}}
\def \hCorr{\widehat{\Corr}}
\def \hCov{\widehat{\Cov}}
\def \cN{\mathcal{N}}

\def \ZZ{\mathbb{Z}}


\begin{document}

\begin{enumerate}
  \setlength\itemsep{2em}

\item Задачи на 1 балл

\begin{enumerate}
  \setlength\itemsep{2em}

\item Какова вероятность того, что при броске кубика выпадет 1 или 2?    

\item Какова вероятность того, что при двух бросках кубика выпадет одно и то же число?

\item Какова вероятность того, что при двух бросках кубика выпадет 1, а потом 2?

\item Какова вероятность того, что при двух бросках кубика не выпадет ни одной шестерки?

\item Какова вероятность того, что при броске кубика выпадет простое число?

\item Какова вероятность того, что при броске кубика выпадет четное число?

\item Какова вероятность того, что при броске кубика выпадет 5 или 6?

\item Какова вероятность того, что при двух бросках кубика выпадет две шестерки?

\item Какова вероятность того, что при двух бросках монеты выпадет два орла?

\item Какова вероятность того, что при двух бросках монеты выпадет две решки?

\item Какова вероятность того, что при двух бросках монеты выпадет один орел и одна решка?

\item У Пети связка с десятью ключами. Только один из них подходит к замку. Петя пробует их по очереди в случайном порядке. Какой по счёту ключ скорее всего подойдёт?

\item	На клавиатуре телефона 10 цифр, от 0 до 9. Какова вероятность того, что случайно нажатая цифра будет меньше 4?

\item	В случайном эксперименте симметричную монеты бросают четырежды. Найди вероятность того, что решка не выпадет ни разу.

\item	Какова вероятность того, что случайно выбранное натуральное число от 41 до 56 делится на 2?

\newpage
\item	Помещение освещается фонарем с тремя лампами. Вероятность перегорания одной лампы в течение года равна 0,5. Найди вероятность того, что в течение года хотя бы одна лампа не перегорит.

\item	Механические часы с двенадцатичасовым циферблатом в какой-то момент сломались и перестали ходить. Найди вероятность того, что часовая стрелка застыла, достигнув отметки 6, но не дойдя до отметки 9 часов.

\item	В  магазине стоят два платежных автомата.  Каждый из них может быть неисправен с вероятностью 0,1 независимо от другого автомата. Найди вероятность того, что хотя бы один автомат исправен?

\item	На клавиатуре телефона 10 цифр, от 0 до 9. Какова вероятность того, что случайно нажатая цифра будет больше 4?

\item	В случайном эксперименте симметричную монеты бросают четырежды. Найди вероятность того, что решка выпадет ровно 1 раз.


\item	Какова вероятность того, что случайно выбранное натуральное число от 42 до 56  не делится на 2?

\item	Какова вероятность того, что случайно выбранное натуральное число от 21 до 32  делится на 3?

\item	Какова вероятность того, что случайная буква из русского алфавита, включая ё, будет гласной?



\item	Механические часы с двенадцатичасовым циферблатом в какой-то момент сломались и перестали ходить. Найди вероятность того, что часовая стрелка застыла, достигнув отметки 3, но не дойдя до отметки 8 часов.


\item	Механические часы с двенадцатичасовым циферблатом в какой-то момент сломались и перестали ходить. Найди вероятность того, что часовая стрелка застыла, не дойдя до отметки 8 часов.



\newpage



\end{enumerate}

\item Задачи на 2 балла

\begin{enumerate}
  \setlength\itemsep{2em}

  \item В очереди из 10 человек в случайных местах стоят Андрей, Борис и Владимир. Какова вероятность того, что Борис и Владимир стоят позже Андрея?
  \item В банке три красные конфеты, четыре синие конфеты, и пять желтых конфет. Миша вытаскивает из банки две конфеты и по очереди съедает их. Какова вероятность того, что обе были желтыми?
  \item Буквы в слове МИША смешали и затем выложили в случайном порядке. Какова вероятность, что получится то же самое слово?
  \item Буквы в слове МAША смешали и затем выложили в случайном порядке. Какова вероятность, что получится то же самое слово?
  \item Буквы в слове МAМА смешали и затем выложили в случайном порядке. Какова вероятность, что получится то же самое слово?
  \item На день рождения Васи пришли 10 мальчиков и 5 девочек. Они все вместе сели за круглый стол. Какова вероятность того, что слева от именинника будет девочка?
  \item На день рождения Васи пришли 10 мальчиков и 5 девочек. Они все вместе сели за круглый стол. Какова вероятность того, что слева и справа от именинника будут сидеть мальчики?
  \item Среди шахматистов каждый седьмой — музыкант, а среди музыкантов каждый девятый — шахматист. Кого больше, шахматистов или музыкантов, и во сколько раз?
  \item В банке три красные конфеты, четыре синие конфеты, и пять желтых конфет. Миша вытаскивает из банки две конфеты и по очереди съедает их. Какова вероятность того, что он не съел ни одной желтой конфеты?

  \item В банке три красные конфеты, четыре синие конфеты, и пять желтых конфет. Миша вытаскивает из банки две конфеты и по очереди съедает их. Какова вероятность того, что обе были синими?

  \item В банке три красные конфеты, четыре синие конфеты, и пять желтых конфет. Миша вытаскивает из банки две конфеты и по очереди съедает их. Какова вероятность того, что обе были красными?

  \newpage
  \item У Паши 4 ореха. Из них два, не ясно какие, пустые. Паша разбивает первый орех, и затем, не глядя на результат, разбивает второй. Второй разбитый орех — пустой. Вероятность того, что первый разбитый орех был пустым?

  \item Саша Абанов подбрасывает золотой доллар 10 раз. Какова вероятность того, что хотя бы один раз доллар выпадет орлом вверх?

  \item Десять аргонавтов разного роста идут по узкой тропинке друг за другом. Каждый аргонавт видит вперёд не далее спины более высокого аргонавта. Внезапно впереди аргонавтов показалась Медуза Горгона, и все, кто видел её, обратились в камни. Какова вероятность того, что в камень обратился третий с конца аргонавт?

  \item	В  магазине стоят два платежных автомата.  Каждый из них может быть неисправен с вероятностью 0,1 независимо от другого автомата. Найди вероятность того, что оба автомата исправны?

  \item	 Биатлонист 5  раз стреляет по мишеням. Вероятность попадания в мишень при одном выстреле равна 0,5. Найди вероятность того, что биатлонист попадет в мишень все 5 раз.

  \item	Биатлонист 5  раз стреляет по мишеням. Вероятность попадания в мишень при одном выстреле равна 0,5. Найди вероятность того, что биатлонист хотя бы раз попадет в мишень.

\item	Помещение освещается фонарем с тремя лампами. Вероятность перегорания одной лампы в течение года равна 0,5. Найди вероятность того, что в течение года перегорят все лампы.

\end{enumerate}

\item Задачи на 3 балла

\begin{enumerate}
  \setlength\itemsep{2em}
  \item Аня хватается за окружность единичной длины в произвольной
точке. Боря берёт мачете и с завязанными глазами разрубает окружность в двух случайных независимых местах. Аня забирает себе тот кусок, за который держится. Боря забирает оставшийся кусок. Чему равна в среднем  длина куска окружности, доставшегося Ане?
	\item  На шахматную доску из 64 клеток ставят наудачу две ладьи. С какой вероятностью они не будут «бить» друг друга?

	\item Возле десятиместной скамейки стоят 11 членов дирекции. В какой-то момент десять из них случайным образом садятся на скамейку. Найти вероятность того, что Садовский, Кечин и Казакова окажутся рядом.

\newpage
	\item Куб с окрашенными гранями распилен на n = 27 кубиков одинакового размера, которые перемешаны. Извлекаются 3 кубика. Найти вероятность того, что у них будет в сумме ровно 2 окрашенные грани.

	\item Два стрелка стреляют по мишени по одному разу. Вероятность того, что оба стрелка попали в мишень, равна 0,54, а вероятность того, что оба промахнулись – 0,04. Какова вероятность попадания в мишень наиболее метким стрелком при одном выстреле?
	\item В автобусе едут $n$ пассажиров. На следующей остановке каждый из них выходит с вероятностью 1/2. Кроме того, в автобус с вероятностью 1/2 не входит ни один новый пассажир, с вероятностью 1/2 входит один новый пассажир, более одного пассажира войти не может. Найти вероятность того, что, когда автобус снова тронется в путь после следующей остановки, в нем будет по-прежнему $n$ пассажиров.
	\item Два цеха завода производят однотипные детали, которые поступают на сборку в общий контейнер. Известно, что первый цех производит в 2 раза больше деталей, чем второй цех. В первом цехе брак составляет 12\%, во втором – 8\%. Для контроля из контейнера берется одна деталь. Какова вероятность того, что извлечённую бракованную деталь выпустил 2-й цех?
	\item Монета подбрасывается 10 раз. Найти вероятность наивероятнейшего числа появлений орла.
	\item Две грузовые машины могут подойти на погрузку в промежуток времени от 19.00 до 20.30. Погрузка каждой машины длится 10 минут. Какова вероятность того, что одной машине придется ждать окончания погрузки другой?
	\item В одной из популярных в Америке игр игрок бросает монету с достаточно большого расстояния на поверхность стола, разграфленную на однодюймовые квадраты. Если монета (3/4 дюйма в диаметре) попадает полностью внутрь квадрата, то игрок получает награду, в противном случае он теряет свою монету. Каковы шансы выиграть при условии, что монета упала на стол?
	\item Петя и Вова играют в кости на фантики. Ведущий игру Петя выигрывает, если при бросании им двух игральных кубиков сумма выпавших на них очков не превосходит 4 и проигрывает во всех остальных случаях. Проиграв, Петя отдаёт Вове 1 фантик, выиграв – получает от Вовы $k$ фантиков. Найти значение $k$, при котором среднее значение выигрыша каждого игрока равна нулю.
  \item На встречу пришли три пары людей и случайным образом разбились на пары. Какова вероятность того, что получились те же самые пары?
  \newpage
\item Обычная игральная кость имеет на своих гранях числа 1, 2, 3, 4, 5, 6. Ее бросают случайным образом до тех пор, пока сумма выпавших за время бросания очков не превысит числа 12. Какая общая сумма очков будет наиболее вероятной?

	\item Имеется 8 отрезков с длинами 1, 2, 3, 4, 5, 6, 7 и 8. Случайным образом выбираются три отрезка. Какова вероятность, что из них можно составить треугольник?
	\item Из колоды, содержащей 52 карты, случайным образом извлекается пять карт. Какова вероятность фулл хауса (тройка и двойка карт одного достоинства, например, три вальта и две десятки)?
	%\item На шахматную доску из 64 клеток ставят наудачу три ладьи. С какой вероятностью ни одна из них не будет «бить» никакую другую?
%\item7/9×(64-36-2)/36

\end{enumerate}


\end{enumerate}

\begin{enumerate}
  \item
  \begin{enumerate}
    \item ddd
  \end{enumerate}
  \item ss
   \begin{enumerate}
     \item dd
   \end{enumerate}
  \item Ответы на 3 балла
  \begin{enumerate}
    \item $2/3$
    \item $7/9$
    \item $(C_8^7/C_{11}^{10})\cdot ((8\cdot 3!\cdot 7!)/10!)$
    \item $C_6^2/27$
    \item 0,9
    \item $(n+1)\cdot (1/2)(n+1)$
    \item $1/4$
    \item $C_{10}^5\cdot (1/2)10$
    \item $1-(80\cdot 80)/(90\cdot 90)$
    \item 1/16
    \item 5
    \item $6!/(3!2^3)$
    \item $13$
  \item $(C_8^3-(C_7^2+4+3+2+1+2+1))/ C_8^3$
  \item $13\cdot 4\cdot 12\cdot C_4^2/ C_{52}^5$

  \end{enumerate}

\end{enumerate}



\end{document}
