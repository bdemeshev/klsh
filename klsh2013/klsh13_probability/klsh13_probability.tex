\documentclass[pdftex,12pt,a4paper]{article}

\input{/home/boris/science/tex_general/title_bor_utf8}

\newtheorem*{defin}{Определение}
\newtheorem*{theorem}{Теорема}
\newtheorem*{idea}{Идея}


\begin{document}
\parindent=0 pt % отступ равен 0



\section{Встреча 1, 16 июля, 21 человек}


\begin{defin}
Факториал числа $n$,

\begin{equation}
n! = n\cdot (n-1)\cdot \ldots \cdot 3\cdot 2\cdot 1
\end{equation}
\end{defin}


\begin{enumerate}
\item Найди $5!$, $200!/198!$
\item Упрости $2013!/2013$
\item Устно сравни $50!$ и $10^{50}$
\item На факультатив записалось 7 человек. Сколькими способами их можно упорядочить в списке?
\item Сколькими способами можно расставить 6 человек в очередь?
\item В фирме работают 4 человека. Сколькими способами их можно назначить на должности директора, заместителя директора, кассира и уборщика? А на должности директора, заместителя и двух кассиров?
\item В очереди стоят 5 человек из команды $\alpha$ и 6 --- из команды $\Omega$. Сколько есть способов расставить школьников, чтобы сначала шла команда $\alpha$, а затем $\Omega$? А если нужно, чтобы команды чередовались? 
\item У Маши 5 книжек, две она взяла в КЛШ, а три оставила дома. Сколько возможностей выбора было у Маши? Перечисли их все.
\item Лев собрал 100 зверей. Сколькими способами их можно расставить в очередь ко льву?
\item Лев собрал 100 зверей и решил их раскрасить, каждого целиком в один цвет. Лев хочет 20 красных, 30 желтых и 50 зеленых зверей. Сколько существует вариантов раскрасок?
\item В библиотеке Маше выдали 25 книг. Она решила прочесть 4 книги. Сколько вариантов выбора есть у Маши?
\end{enumerate}

\begin{defin}
Число сочетаний из $n$ по $k$, 

\begin{equation}
C_n^k=\left( \begin{array}{c}
n \\ 
k
\end{array}  \right) = \frac{n!}{k!(n-k)!}
\end{equation}

\end{defin}


\begin{theorem}
Если $n$ предметов нужно раскрасить в три цвета: в зелёный --- $a$ предметов, в желтый --- $b$ предметов, в красный --- $c$ предметов, где $a+b+c=n$, то количество способов сделать это равно
\[
\frac{n!}{a!b!c!}
\]
\end{theorem}


\begin{enumerate}[resume]
\item Ответы каких предыдущих задач можно записать с помощью числа сочетаний?
\item У Маши 3 разных кольца. Сколькими способами она может надеть их на левую руку, если ей важен порядок следования колец на пальцах?
\end{enumerate}


\newpage
\section{Встреча 2, 17 июля, 21 человек}

Задачи:
\begin{enumerate}
\item В команде 9 человек. 
\begin{enumerate}
\item Сколькими способами можно сформировать команду на ФМТ из четырех человек?
\item Сколькими способами можно сформировать команду на ГУТ из трех человек?
\item Сколькими способами можно сформировать команду на ФМТ и команду на ГУТ, если в них участвуют разные школьники?
\end{enumerate}
\item Есть $38$ дырочек, расположенных в линию, на расстоянии в 1~см друг от друга. У каждой вилки два штырька на расстоянии в 1 см.
 Сколькими способами можно воткнуть $18$ одинаковых вилок?
\item У Рустама $20$ книжек стоят подряд на полке. Рустам хочет выбрать $5$ книг так, чтобы не было выбрано соседних. Сколькими способами можно это сделать?

\item На длинной лавочке $20$ мест. Сколькими способами можно рассадить на лавочке $13$ человек, чтобы смогли рядом сесть Садовский и Байбурин?

\item Раскрой скобки в выражении $(a+b)^7$ не выполняя умножения.
   
\end{enumerate}

\newpage
\section{Встреча 3, 18 июля, 21 человек}

\begin{enumerate}
\item Генуэзская лотерея (задача Леонарда Эйлера).

Из 90 чисел выбираются 5 наугад. Назовем серией последовательность
из нескольких чисел, идущих подряд. Например, если выпали числа
23, 24, 77, 78 и 79, неважно в каком порядке, то есть две серии
(23-24, 77-78-79). Сколько есть комбинаций с одной серией? С двумя? С тремя? \ldots 

\item На столе стоят 4 отличающихся друг от друга чашки, 4 одинаковых граненых стакана, 10 одинаковых кусков сахара, 7 соломинок разных цветов. Сколькими способами можно полностью разложить: 
\begin{enumerate}
\item сахар по чашкам;
\item сахар по стаканам;
\item соломинки по чашкам;
\item соломинки по стаканам;
\item Как изменятся ответы, если требуется, чтобы пустых емкостей не
оставалось?
\end{enumerate}

\end{enumerate}

\begin{defin}
Пространство элементарных исходов $\Omega$ --- множество всех возможных исходов случайного эксперимента. Случайный эксперимент оканчивается одним исходом из множества $\Omega$.
\end{defin}

\begin{defin}
Событие $A$ --- произвольное подмножество множества $\Omega$. Событие $A$ происходит, если происходит любой исход $w$ из события $A$.
\end{defin}

\begin{defin}
Вероятность, $\P(A)$ --- функция, сопоставляющая каждому событию $A$ число от $0$ до $1$, обладающая свойствами:
\begin{enumerate}
\item $\P(\Omega)=1$, $\P(\emptyset)=0$
\item Если события $A_1$, $A_2$, \ldots не могут произойти одновременно, то 
\[
\P(A_1 \cup A_2 \cup A_3 \cup \ldots )=\P(A_1)+\P(A_2)+\P(A_3)+\ldots
\]
\end{enumerate}

\end{defin}

\begin{theorem}
Если исходов конечное количество и они равновероятны, то 
\[
\P(A)=\frac{\text{количество исходов в событии }A}{\text{количество исходов в множестве }\Omega}
\]
\end{theorem}

\begin{enumerate}[resume]
\item Кубик подбрасывается два раза подряд. Сколько исходов в пространстве $\Omega$? Какова вероятность того, что максимальный результат равен $5$? Сумма двух результатов равна $7$?
\item Из 50 деталей 4 бракованных. Покупатель выбирает наугад 10 деталей. Какова вероятность того, что все купленные детали будут хорошими?
\end{enumerate}


\newpage
\section{Встреча 4, 19 июля, 21 человек}

\begin{enumerate}
\item Монетка подбрасывается три раза подряд. Сколько исходов в пространстве $\Omega$? Какова вероятность того, что будет ровно два <<орла>>?
\item Наугад из четырех тузов разных мастей выбираются два. Как выглядит пространство всех исходов $\Omega$? $\P(\text{тузы будут разного цвета})$?
\item Вероятность $\P(A)=0.3$,  $\P(B)=0.8$. В каких пределах может лежать $\P(A\cap B)$? В каких пределах может лежать $\P(A\cup B)$?
\item Какова вероятность того, что у 24 человек не будет ни одного совпадающего дня рождения?
\item Покер. Выбирается 5 карт из колоды в 52 карты. Джокеров в колоде нет. У карт 4 масти и 13 достоинства: от двойки до туза. Рассчитайте вероятности комбинаций:
\begin{enumerate}
\item Pair (пара) "--- две карты одного достоинства, три остальные "--- разного
\item Two pairs (две пары) "--- две карты одного достоинства и две другого;
\item Three of Kind (тройка) "--- три карты одного достоинства, две остальные "--- разного;
\item Straight (стрит) "--- пять последовательных карт, есть карты разных мастей
\item Flush (масть) "--- все карты одной масти, но не роял-флэш
\item Full House (фул-хаус) "--- три карты одного достоинства и две другого;
\item Four of Kind  (каре) "--- четыре карты одного достоинства;
\item Straight Flush (стрит-флэш) "--- пять последовательных карт одной масти;
\item Royal Flush (роял-флэш) "--- старшие пять последовательных карт одной масти?
\item Суровая Сибирская Задача: рассчитайте указанные вероятности, если в колоде 54 карты из которых 2 джокера. Джокер заменяет любую карту по желанию игрока.
\end{enumerate}

\end{enumerate}

\begin{idea}
Строгое определение равновероятности для случая бесконечного количества исходов не такое простое. Нам будет достаточно интуитивного понимания идеи
\[
\P(A)=\frac{\text{Размер множества }A}{\text{Размер множества }\Omega}
\]
Слово <<Размер>> может означать длину отрезка, площадь фигуры и т.д.
\end{idea}

\begin{enumerate}[resume]
\item На бумаге проведена прямая. На бумагу бросают спичку. Какова
вероятность, что острый угол между прямой и спичкой будет меньше
10 градусов?
\item Вася бегает по кругу длиной 400 метров. В случайный момент
времени он останавливается. Какова вероятность того, что он будет
ближе, чем в 50 м от точки старта? Дальше, чем в 100 м?
\item Треугольник с вершинами $(0;0)$, $(2;0)$ и $(1;1)$. Внутри него случайным образом выбирается точка, $X$ -- абсцисса точки. Чему равные $\P(X>1)$, $\P(X\in [0.5;1])$?
\end{enumerate}

Коммент: часть пунктов про Покер осталась на следующий день


\newpage
\section{Встреча 5, 20 июля, 21 человек}

\begin{enumerate}


\item Внутри квадрата выбирается точка наугад. Какова вероятность того, что она будет ближе к центру, чем к любой из вершин?

\item Маша держит веревочное кольцо одной рукой. Мы разрезаем кольцо в двух случайно выбираемых точках. Кусок, который Маша держит в руке, достается ей, упавший кусок забирает Вовочка. Какова вероятность того, что Машин кусок длиннее?

\item Контрольную писали 32 человека, каждый из четырех вариантов писали 8 человек. Какова вероятность того, что Вася Петькин и Петя Васькин писали один и тот же вариант?

\item На окружности наугад выбираются точки $A$, $B$, $C$, $D$, $E$ и $F$. Какова вероятность того, что треугольники $ABC$ и $DEF$ не пересекаются?

\item Шесть школьников, три из команды $\alpha$ и три из команды $\beta$, стоят в очереди в случайном порядке. Какова вероятность того, что школьники из одной команды ни разу не стоят рядом?

\item На карточках написаны числа от 1 до 100. В левую руку Маша берёт случайно одну карточку, в правую --- 10 карточек. Какова вероятность того, что число на карточке в левой руке окажется больше любого числа на карточках из правой руки?

\item В мешке 50 орехов, из них 5 пустые. Вы выбираете наугад 10 орехов. Какова вероятность того, что ровно два из них будут пустыми?

\item В бридж играют четыре игрока: Юг, Восток, Север, Запад. Перемешанная колода в 52 карты раздаётся игрокам по очереди по одной карте. Юг и Север получили 11 пик. Какова вероятность того, что две оставшиеся пики оказались у одного игрока? Разделились между остальными игроками? Каковы вероятности различных раскладов пик между остальными игроками, если Юг и Север получили 8 пик?

\end{enumerate}

Коммент: дорешали почти все пункты про Покер, не решали 3, 4, 5, 7

\newpage
\section{Встреча 6, 22 июля, 20 человек}

\begin{defin}
\textbf{Случайная величина}, $X$ --- функция сопоставляющая каждому исходу действительное число.
\end{defin}

\begin{defin}
Для случайной величины с конечным множеством значений \textbf{математическое ожидание}, среднее, равно
\[
\E(X) = x_1  \cdot \P(X=x_1)+x_2 \cdot  \P(X=x_2)+x_3 \cdot  \P(X=x_3) +\ldots
\]
\end{defin}

\begin{enumerate}
\item Величина $N$ --- количество 
шестерок выпадающих при двух подбрасываниях кубика. Найди $\P(N=0)$, $\P(N=1)$, $\P(N=2)$, $\P(N=3)$, $\P(N\ge 1)$, $\E(N)$
\item Игральный кубик подбрасывается два раза, $X_1$ --- результат первого подбрасывания, $X_2$ --- второго. Найдите $\E(X_1)$, $\E(\max\{X_1,X_2\})$, $\E(\min\{X_1,X_2\})$, $\E(|X_1-X_2|)$, $\E(\text{НОК}(X_1,X_2))$, $\E(\text{НОД}(X_1,X_2))$

\item Величина  $X$  принимает два значения, 0 или 1. Найди $\E(X)$, если $\P(X=1)=p$.


\end{enumerate}

\begin{idea} Бывает очень полезно нарисовать \textbf{дерево} возможного развития событий. Вероятность конкретной траектории равна произведению вероятностей на веточках этой траектории.
\end{idea}

\begin{enumerate}[resume]
\item Две команды равной силы играют до трёх побед. В отдельно взятой партии вероятность победы каждой команды одинакова, ничья невозможна.  Величина $N$ -- количество сыгранных партий. Составьте табличку возможных значений $N$ с их вероятностями. Найди $\P(N$ --- четное$)$, $\E(N)$ 

\item Примерно $0.4$\% коров заражены <<коровьим бешенством>>.  Имеется тест, который дает ошибочный результат с вероятностью $0.1$ вне зависимости от того, заражено мясо или нет. Какова вероятность того, что случайно выбранная партия будет признана тестом зараженной?

\item В КЛШ-2013 25\% летнешкольников общественного направления, из них 60\% девочек. Среди остальных летнешкольников девочек 55\%. Какова вероятность того, что случайно встреченный летнешкольник окажется мальчиком?

\item Вася наугад выбирает два разных натуральных числа от 1 до 4.
\begin{enumerate}
\item Какова вероятность того, что будет выбрано число 3?
\item Какова вероятность того, что сумма выбранных чисел будет чётная?
\item Каково математическое ожидание суммы выбранных чисел?
\end{enumerate}

\item  В коробке находится четыре внешне одинаковые лампочки. Две
лампочки исправны, две "--- нет. Лампочки извлекают из коробки по
одной до тех пор, пока не будут извлечены обе исправные.
\begin{enumerate}
\item Какова вероятность того, что опыт закончится извлечением трёх
лампочек?
\item  Каково ожидаемое количество извлеченных лампочек?
\end{enumerate}

\item Подбрасывается правильный кубик. Узнав результат, игрок выбирает,
подкидывать ли кубик второй раз. Игрок получает сумму денег, равную
количеству очков при последнем подбрасывании. Каков максимальный ожидаемый выигрыш игрока при оптимальной стратегии?

\item Кубик подбрасывают до первого выпадения шестерки. Случайная величина  $N$ "---
число подбрасываний. Найди $\P(N=6)$, $\P(N=k)$, $\P(N>10)$ и  $\E(N)$.

\item Извлекаются две карты из колоды в 52 карты. Какова вероятность того, что они будут одной масти? Какова вероятность того, что они будут одного достоинства?


\end{enumerate}

% Без Кости, дзета 

\newpage
\section{Встреча 7, 23 июля, 20 человек}

\begin{defin}
Условная вероятность события $A$, если известно, что событие $B$ произошло
\[
\P(A\mid B) = \frac{\P(A\cap B)}{\P(B)}
\]

\end{defin}


\begin{enumerate}
\item Имеется три монетки. Две <<правильных>> и одна "--- с орлами по
обеим сторонам. Петя выбирает одну монетку наугад и подкидывает её
два раза. Какова вероятность того, что
монетка <<неправильная>>, если оба раза выпал орёл?
\item Социологическим опросам доверяют 70\,\% жителей. Те, кто доверяет опросам, всегда отвечают искренне; те, кто не доверяет, отвечают наугад, равновероятно выбирая <<да>> или <<нет>>. Социолог Петя  в анкету очередного опроса включил вопрос: <<Доверяете ли Вы социологическим опросам?>>
\begin{enumerate}
\item Какова вероятность, что случайно выбранный респондент ответит <<Да>>?
\item  Какова вероятность того, что он действительно доверяет, если известно, что он ответил <<Да>>?
\end{enumerate}

\item Ты играешь две партии в шахматы против незнакомца. Равновероятно
незнакомец может оказаться новичком, любителем или профессионалом.
Вероятности вашего выигрыша в отдельной партии, соответственно,
будут равны 0{,}9; 0{,}5; 0{,}3.
\begin{enumerate}
\item Какова вероятность выиграть первую партию?
\item Какова вероятность выиграть вторую партию, если ты выиграл
первую?
\end{enumerate}

\item У тети Маши --- двое детей, один старше другого. Предположим, что вероятности рождения мальчика и девочки равны и не зависят от дня недели, а пол первого и второго ребенка независимы. 
\begin{enumerate}
\item Известно, что хотя бы один ребенок --- мальчик. Какова
вероятность того, что другой ребенок --- девочка?
\item Тетя Маша наугад выбирает одного своего
ребенка и посылает к тете Оле, вернуть учебник по теории
вероятностей. Это оказывается мальчик. Какова вероятность того,
что другой ребенок --- девочка? 
\item Известно, что старший ребенок --- мальчик. Какова вероятность того, что другой ребенок --- девочка? 
\item На вопрос: <<А правда ли тетя Маша, что у вас есть сын, родившийся в пятницу?>>. Она ответила: <<Да>>. Какова вероятность того, что другой ребенок --- девочка?
\end{enumerate}

\item Самолет упал в горах, в степи или в море. Вероятности,
соответственно, равны $0,5$, $0,3$ и $0,2$. Если он упал в горах,
то при поиске его найдут с вероятностью $0,7$. В степи --- $0,8$, на
море --- $0,2$. Какова вероятность того, что самолет упал в море, если его искали в горах, в степи и не нашли?

\item Давайте сыграем в русскую рулетку\ldots Вы привязаны к стулу и не
можете встать. Вот пистолет. Вот его барабан --- в нем шесть гнезд
для патронов, и они все пусты. Смотрите: у меня два патрона. Вы
обратили внимание, что я их вставил в соседние гнезда барабана?
Теперь я ставлю барабан на место и случайно вращаю его. Я подношу пистолет
к вашему виску и нажимаю на спусковой крючок. Щелк! Вы еще живы.
Вам повезло! Сейчас я собираюсь еще раз нажать на крючок. Что вы
предпочитаете: чтобы я снова случайно провернул барабан или чтобы просто
нажал на спусковой крючок?

\end{enumerate}

Коммент: решили всё кроме 2 и 3


\newpage
\section{Встреча 8, 25 июля, 21 человек}

\begin{enumerate}

\item Саша и Маша по очереди подбрасывают кубик. Посуду будет
мыть тот, кто первым выбросит шестерку. Маша бросает первой.
Какова вероятность того, что Маша будет мыть посуду?

\item Злобные варвары поймали Ламзина, Байбурина и Садовского. Всем известно, что двоих из них освободят, а одного --- съедят. Садовский спрашивает у злобного варвара: <<Назови мне, пожалуйста, одного человека, кого освободят, но не меня?>> Какова вероятность того, что Садовского освободят, если варвар ответил <<Ламзин>>?

\item Саша и Маша подкидывают монетку до тех пор, пока не выпадет
последовательность РОО или ОOР. Если игра закончится выпадением
РОО, то выигрывает Саша, если ОOР, то --- Маша. 
\begin{enumerate}
\item У кого какие шансы выиграть? 
\item Сколько в среднем времени ждать до определения победителя?
\end{enumerate}

\end{enumerate}

\begin{theorem}
Для любой последовательности ABC найдётся такая последовательность XYZ, что $\P(\text{XYZ выпадет раньше ABC})>0.5$.
\end{theorem}

Коммент: считали РРО против РОО и прочие, тренировались строить упрощенные деревья.

\begin{enumerate}

\item Сколько раз в среднем надо подбрасывать кубик до выпадения первой шестерки?

\item Сколько раз в среднем нужно подбрасывать монетку до первого выпадения РОО? До РРО?

\item Есть три комнаты. В первой из них лежит сыр. Если мышка
попадает в первую комнату, то она находит сыр через одну минуту.
Если мышка попадает во вторую комнату, то она ищет сыр две минуты
и покидает комнату. Если мышка попадает в третью комнату, то она
ищет сыр три минуты и покидает комнату. Покинув комнату, мышка
выходит в коридор и выбирает новую комнату наугад (мышка может
зайти в одну и ту же). Сейчас мышка в коридоре. Сколько времени ей
в среднем потребуется, чтобы найти сыр?

\item В вершинах треугольника три Зелёных ёжика. Каждую минуту каждый Зелёный ёжик равновероятно движется либо по часовой стрелке, либо против, независимо от остальных ёжиков. Сколько в среднем пройдет времени, прежде чем они встретятся в одной вершине? 

\end{enumerate}



\newpage
\section{Встреча 10, 27 июля, ... человек}

\begin{enumerate}
\item Вася подкидывает кубик. Если выпадает единица, или Вася говорит
<<стоп>>, то игра оканчивается, если нет, то Вася имеет право подбросить кубик второй раз. Васин выигрыш --- последнее выпавшее число. Как выглядит оптимальная
стратегия?

\item Вася подкидывает кубик. Если выпадает единица, или Вася говорит
<<стоп>>, то игра оканчивается, если нет, то игра начинается заново. Васин выигрыш --- последнее выпавшее число.
\begin{enumerate}
\item Как выглядит оптимальная стратегия?
\item Чему равен ожидаемый выигрыш при её использовании?
\item Сколько подбрасываний в среднем будет сделано?  
\end{enumerate}
   

\item В классе было 2 мальчика и сколько-то девочек. Заходит еще кто-то. Ребята решили, что народу стало слишком много, выбрали одного человека жеребьевкой и выгнали. Какова вероятность того, что вошел мальчик, если выгнали мальчика?

\item Два охотника выстрелили в одну утку. Первый попадает с
вероятностью $0{,}4$, второй "--- с вероятностью $0{,}6$. Какова вероятность того, что утка была убита первым
охотником, если в утку попала ровно
одна пуля?

\item Илье Муромцу предстоит дорога к камню. И от камня начинаются ещё три дороги. Каждая из тех дорог снова оканчивается камнем. И от каждого камня начинаются ещё три дороги. И каждые те три дороги оканчиваются камнем\ldots И так далее до бесконечности. На каждой дороге можно встретить живущего на ней трёхголового Змея Горыныча с вероятностью (хм, вы не поверите!) одна третья. Какова вероятность того, что у Ильи Муромца существует возможность пройти свой бесконечный жизненный путь, так ни разу и не встретив Змея Горыныча?

\end{enumerate}

Коммент: про И.М. --- довольно суровая задача для школьников

\newpage
\section{Встреча 11, 28 июля, ... человек}

\begin{enumerate}

\item Цвет глаз кодируется несколькими генами. В целом более темный цвет доминирует более светлый. Ген карих глаз доминирует ген синих. Т.е. у носителя пары bb глаза
синие, а у носителя пар BB и Bb --- карие. У диплоидных организмов
(а мы такие :)) одна аллель наследуется от папы, а одна --- от мамы.
В семье у кареглазых родителей два сына --- кареглазый и синеглазый.
Кареглазый женился на синеглазой девушке. Какова вероятность
рождения у них синеглазого ребенка?

\item В ЕГЭ по информатике в части А на каждый вопрос есть четыре варианта ответа. Ровно один из них верный. Рустам знает 80\% курса по информатике и поэтому вероятность того, что он знает ответ на конкретный вопрос равна $0.8$. Если Рустам не знает ответ, то он выбирает равновероятно один из возможных наугад. 
\begin{enumerate}
\item Какова вероятность того, что Рустам даст верный ответ на первый вопрос?
\item Какова вероятность того, что Рустам знал ответ на первый вопрос, если он верно ответил на него?
\item Какова вероятность того, что Рустам знал ответы на первые два вопроса, если он верно ответил на оба?
\end{enumerate}

\item Пьяница стоит на расстоянии одного шага от края пропасти. Он
шагает случайным образом либо к краю утеса либо от него. На каждом
шагу вероятность отойти от края равна $2/3$, а шаг к краю имеет
вероятность $1/3$. Каковы шансы пьяницы никогда не упасть?

\item Изначально популяция состоит из одной амёбы. Каждую минуту каждая амёба делится на две части с вероятностью $1/4$ или погибает с вероятностью $3/4$. Какова вероятность того, что эта популяция будет жить вечно?

\item Прежнего шамана племени забодало носорогом. На его смену пришел молодой и перспективный шаман. В силу долгой засухи вождь обратился к шаману с просьбой вызвать дождь. 

Чтобы вызвать дождь необходимо произнести заклинание <<АБРА>>. Молодой шаман знает, что в заклинании участвуют всего три буквы: <<А>>, <<Б>> и <<Р>>; и что в заклинании нет двух одинаковых букв подряд. Больше ничего о заклинании молодому шаману неизвестно. Поэтому шаман произносит буквы по одной наугад соблюдая эти два правила. 

Сколько букв услышит вождь от шамана прежде чем пойдет дождь?

\item На кубиках написаны числа от 1 до 100. Кубики свалены в кучу. Вася выбирает наугад из кучи по очереди три кубика.
\begin{enumerate}
\item Какова вероятность, что полученные три числа будут идти в возрастающем порядке?
\item Какова вероятность, что полученные три числа будут идти в возрастающем порядке, если известно, что первое меньше последнего?
\end{enumerate}

\item Сколько подбрасываний кубика в среднем потребуется, чтобы какая-нибудь грань выпала два раза подряд?

\item Паук ползает по вершинам куба, начинает в точке $A$. Противоположную вершину куба обозначим буквой $B$. 
\begin{enumerate}
\item Сколько ходов в среднем потребуется пауку, чтобы добраться до $B$?
\item Сколько ходов в среднем потребуется пауку, чтобы впервые вернуться в $A$?
\item Сколько раз в среднем паук посетит вершину $B$, прежде чем впервые вернется в $A$?
\end{enumerate}


\end{enumerate}



\newpage
\section{Встреча 12, 30 июля, ... человек}

\begin{enumerate}

\item На паужин команде выдали 5 яблок и 6 груш. Сначала Вася взял один фрукт наугад, а затем Петя взял ещё один фрукт наугад. 
\begin{enumerate}
\item Какова вероятность того, что вдвоем они взяли хотя бы одно яблоко?
\item Какова вероятность того, что они на двоих взяли яблоко и грушу?
\item Какова вероятность того, что они взяли груши, если известно, что они взяли один и тот же фрукт?
\item Какова вероятность того, что они взяли разные фрукты, если известно, что они взяли хотя бы одно яблоко?
\end{enumerate}

\item В КЛШ-2013 25\% летнешкольников общественного направления, из них 60\% девушек. Среди остальных летнешкольников девушек 55\%. 
\begin{enumerate}
\item Какова вероятность того, что случайно встреченный летнешкольник окажется девушкой?
\item Какова вероятность того, что случайно встреченный летнешкольник с общественного направления, если известно, что это девушка?
\item Какова вероятность того, что случайно встреченный летнешкольник не с общественного направления, если известно, что это юноша?
\end{enumerate}


\item Сколько подбрасываний кубика в среднем потребуется, чтобы какая-нибудь грань выпала два раза подряд?

\item В городе N ясная погода стоит 70\% времени. Если завтра будет ясная погода, то синоптик Рустам предсказывает ясную погоду с вероятностью $0.9$ и пасмурную --- с вероятностью $0.1$. Если погода завтра будет пасмурная, то Рустам с вероятностью $0.8$ предскажет пасмурную. 
\begin{enumerate}
\item Какова вероятность того, что прогноз будет <<пасмурно>>?
\item Какова вероятность того, что погода будет ясная, если был прогноз <<ясно>>?
\item Какова вероятность того, что погода будет пасмурная, если был прогноз <<пасмурно>>?
\end{enumerate}


\item Есть треугольная пирамида с вершинами $A$, $B$, $C$ и $D$. В вершине $A$ сидит муха, в вершине $B$ лежит банка с вареньем. За каждую минуту муха случайным образом переползает в одну из соседних вершин.
\begin{enumerate}
\item Сколько ходов в среднем нужно мухе, чтобы доползти до вершины $B$?
\item Какова вероятность того, что муха вернётся в $A$ раньше, чем посетить вершину $B$?
\item Сколько раз в среднем муха посетит вершину $C$ прежде чем доползёт до вершины $B$?
\end{enumerate}
\end{enumerate}

\newpage
\section{Встреча 13, ...}


\begin{enumerate}

\item На столе 10 коробок. Мы кладем 7 карандашей по очереди наугад, каждый раз выбирая одну из всех коробок равновероятно. Сколько в среднем коробок окажутся пустыми?

\item Мама в среднем получает 40 т.р. в месяц, папа --- в среднем 50 т.р. Каков среднемесячный доход семьи? Как связаны между собой $\E(X)$, $\E(Y)$ и $\E(X+Y)$?


\item Рустам написал 10 писем и надписал 10 конвертов с нужными адресами. По рассеянности он разложил письма в конверты в случайном порядке. Сколько в среднем адресатов получат <<свои>> письма?

\item Десять охотников вспугнули стаю из двадцати уток. Каждый охотник независимо от других случайно выбирает себе цель и стреляет. Каждый охотник попадает с вероятностью $0.7$.
\begin{enumerate}
\item Сколько в среднем уцелеет уток?
\item Сколько в среднем будет попавших охотников?
\end{enumerate}


\item У Маши семь пар туфель. Собака Боня утащила без разбору на правые и левые случайным образом 5 туфель <<поиграть>>. 
\begin{enumerate}
\item Сколько полных пар осталось у Маши?
\item Сколько полных пар досталось Боне?
\end{enumerate}


\end{enumerate}


\begin{theorem} 
Если $\E(X)$ и $\E(Y)$ существуют, то $\E(X+Y)=\E(X)+\E(Y)$.
\end{theorem}



Коммент: <<уцелевших>> идёт лучше чем <<убитых>>, т.к. проще.

\end{document}
