\documentclass[pdftex,12pt,a4paper]{article}

\input{/home/boris/science/tex_general/title_bor_utf8}


\begin{document}
\parindent=0 pt % отступ равен 0

Маша и Саша играют в быстрые шахматы. У них одинаковый класс игры и оба предпочитают
играть белыми, т.е. вероятность выигрыша белых равна $3/4$. Партии играются до 10 побед, не обязательно подряд. Первую партию Маша играет белыми. Она считает, что в следующей партии белыми должен играть тот, кто выиграл предыдущую партию. Саша считает, что ходить белыми нужно по очереди. 

Во сколько раз вероятность победы Маши по машиным правилам больше вероятности победы Маши по сашиным правилам?





Решение:
Игра обязательно заканчивается за 19 партий, иногда раньше. Обяжем игроков сыграть
недостающие до 19 партий, даже если победитель уже определился. При этом всегда
можно добиться того, что Маша сыграла бы белыми 10 раз. Т.е. мы получаем, что при
любом варианте правил, Маша играет 10 партий белыми, а Саша — 9 черными, а победитель --- тот, кто выиграл 10 партий. Значит, от варианта правил ничего не зависит.


\newpage
Обозначим карты $1$, $2$, $3$. Обозначим игроков $A$, $B$ и $C$.

Перестановка --- взаимно однозначная функция из $\{1,2,3\}$ в $\{1,2,3\}$.

Игрок С будет особым, <<тасующим>>. Число $x$ --- это будет <<тасованный>> или <<шифрованный>> номер карты, а число $\pi_c(x)$ --- настоящий номер карты.

Тасовка виртуальной колоды:
\begin{enumerate}
\item Каждый игрок придумывает себе перестановку. Назовём перестановки соответственно буквами $\pi_a$, $\pi_b$ и $\pi_c$.
\item Игрок А сообщает игроку С перестановку $\pi_a$
\item Игрок Б сообщает игроку С перестановку $\pi_b$
\item Игрок С сообщает игроку А перестановку $\pi_b\circ \pi_c$
\item Игрок С сообщает игроку Б перестановку $\pi_a\circ \pi_c$
\end{enumerate}

Игрок А <<выдает>> карту игроку С:
\begin{enumerate}[resume]
\item Игрок А выбирает любое число $x\in\{1,2,3\}$. Это будет <<перетасованный>> номер карты.
\item Игрок А сообщает число $x$ всем остальным.
\item Игрок С узнает свою карту, $\pi_c(x)$.
\end{enumerate}

Игрок Б <<выдает>> карту игроку А:
\begin{enumerate}[resume]
\item Игрок Б выбирает еще не выбранное $x\in\{1,2,3\}$. Это будет <<перетасованный>> номер карты.
\item Игрок Б сообщает игроку А: число $x$, число $\pi_a\circ \pi_c(x)$.
\item Игрок А узнает свою карту, $\pi_c(x)$.
\end{enumerate}

Игрок А <<выдает>> карту игроку Б:
\begin{enumerate}[resume]
\item Игрок А определяет последнее невыбранное $x\in\{1,2,3\}$. Это будет <<перетасованный>> номер карты.
\item Игрок А сообщает игроку Б: число $\pi_b\circ \pi_c(x)$
\item Игрок Б узнает свою карту, $\pi_c(x)$.
\end{enumerate}

Данная задача по английски называется <<mental poker>>


\end{document}
