\documentclass[pdftex,12pt,a4paper]{article}

\input{/home/boris/Dropbox/Public/tex_general/title_bor_utf8}

%\usepackage{showkeys} % показывать метки

\input{/home/boris/Dropbox/Public/tex_general/prob_and_sol_utf8}

\title{Задачи для тигров из КЛШ}
%\author{Составитель: Борис Демешев, boris.demeshev@gmail.com}
%\date{\today}

\begin{document}

%\pagestyle{myheadings} \markboth{ТВИМС-задачник. Демешев Борис. roah@yandex.ru }{ТВИМС-задачник. Демешев Борис. roah@yandex.ru }
\maketitle
%\tableofcontents{}

\parindent=0 pt % отступ равен 0



Аня и Боря, интеллигентные люди, хотят отдохнуть.
Вот их предпочтения в порядке убывания ценности:\par
Аня: театр, казино, бассейн, футбол. Боря: футбол, бассейн, казино, театр.

Они по очереди, в порядке Аня-Боря-Аня, вычеркивают нежелательную альтернативу, до тех пор, пока не останется только одна. 

а) Нарисуйте дерево игры

б) Найдите равновесия по Нэшу, совершенные в подыграх\par

\vspace{10pt}
Морали особой нет, пусть просто поучатся деревья рисовать и обратно-индукционный исход находить. На контрольной та же задача, только с порядком Боря-Аня-Боря.


\vspace{10pt}
Волшебная шкатулка-2 \par 
Количество денег в волшебной шкатулке постоянно увеличивается! В первый день в ней лежит 2 рубля. Во второй - 4 рубля, в третий - 6 рублей и т.д. Каждый из двух игроков решает, когда ему потребовать деньги. Тот кто потребует деньги первым - получает сумму полностью, тот, кто потребует вторым - не получает ничего. Если требования поступают одновременно, то игроки делят сумму в шкатулке поровну. Если никто не потребует деньги к сотому дню, то в 101-ый день деньги сгорают. Найдите равновесия по Нэшу, совершенные в подыграх.\par

\vspace{10pt}
Решаем с конца. Если мы дошли до момента $t=100$, то перед нами одновременная игра. Рисуем ее матрицу, получаем что в ней есть единственное равновесие по Нэшу - это (требовать, требовать). Рассмотрим $t=99$. Для следующего шага игра уже решена, поэтому перед нами снова матрица два на два. И снова оптимальное поведение для обоих игроков - требовать деньги. Продолжаем до первого момента времени. Получаем, что единственное SPNE - это следующая пара стратегий (требовать деньги в каждый момент времени, требовать деньги в каждый момент времени).

Мораль: равновесие по Нэшу не оптимально.

\vspace{10pt}
Волшебная шкатулка-2 и тупой игрок

Количество денег в волшебной шкатулке постоянно увеличивается! В первый день в ней лежит 2 рубля. Во второй - 4 рубля, в третий - 6 рублей и т.д. Каждый из двух игроков решает, когда ему потребовать деньги. Тот кто потребует деньги первым - получает сумму полностью, тот, кто потребует вторым - не получает ничего. Если требования поступают одновременно, то игроки делят сумму в шкатулке поровну. Если никто не потребует деньги к сотому дню, то в 101-ый день деньги сгорают. 
 Первый игрок - обычный, т.е. он хочет выиграть побольше денег. Как будут развиваться события в игре, если:

а) Второй игрок - тупой, т.е. он при своем ходе подбрасывает монетку: если она падает орлом - требует деньги, если решкой - ждет. Первый игрок знает, что второй - тупой.

б) Первый игрок думает, что второй игрок - тупой. Но на самом деле второй игрок - не тупой.

в) Первый игрок думает, что второй игрок - обычный. Но на самом деле второй игрок - тупой.

\vspace{10pt}
Решение:

Начнем с пункта <<в>>. За тупого игрока думать не надо - он монетку перед ходом бросает. Думаем за первого. Рассмотрим момент $t=100$. Первый рассуждает за себя: мне лучше требовать. Первый рассуждает за второго: второй - обычный, значит он хочет побольше денег, значит он будет требовать. Рассмотрим $t=99$. С точки зрения первого ничего не поменялось по сравнению со случаем, когда все - умные. Значит получаем итог: первый игрок в любой момент времени требует деньги. Второй ходит согласно подброшенной монетке, игра обязательно оканчивается на первом ходе: либо все деньги достаются первому, либо - пополам.

Теперь <<a>>. За тупого игрока думать не надо - он монетку перед ходом бросает. Думаем за первого. Рассмотрим момент $t=100$. Первый рассуждает за себя: мне лучше требовать. Теперь смотрим $t=99$. Первый рассуждает так: если я буду ждать, то получу равновероятно 200 или 100. Если я буду требовать деньги, то я получу равновероятно 198 или 99. Значит лучше ждать. Аналогично для $t=98$,... $t=1$. Игра идет так: первый ждет до $t=100$, а второй побрасывает свою монетку. Игра оканчивается когда второй выбросит <<орла>> или когда время дойдет до $t=100$.

Наконец, <<б>>. В пункте <<а>> мы выяснили, что первый будет ждать при $t=1...99$ и требовать деньги на $t=100$. Второй игрок догадается до этого, т.к. он умный. Значит второй игрок прождет до $t=99$ и потребует деньги. Игра окончится при $t=99$ тем, что второй получит $198$.

Мораль: хорошо быть умным, но казаться дураком. Впрочем иногда лучше быть и казаться тупым, чем быть и казаться умным. Можно увидеть в этой простой задаче и такую глубокую мысль: эмоции способствуют выживанию человека как вида.

\vspace{10pt}
Тигры и волшебная антилопа \par
На острове живут 99 тигров и одна вкусная волшебная антилопа. \par
Если тигр съест волшебную антилопу, то он сам превратится в волшебную антилопу. Мясо волшебной антилопы настолько вкусно, что любой тигр готов ради его вкуса на превращение в антилопу. Но ни один тигр не готов полностью расстаться с жизнью ради мяса антилопы. Тигры охотяться только в одиночку.\par
Что будет происходить на этом острове? \par

\vspace{10pt}
Решение:

 Один тигр и одна антилопа -> тигр съедает антилопу. Два тигра и одна антилопа -> тигры не едят антилопу, т.к. иначе за вкус придется заплатить жизнью. По индукции - при четном числе тигров, тигры воздерживаются от трапезы; при нечетном числе тигров один из тигров ест антилопу, а далее тигры воздерживаются от трапезы. 

\vspace{10pt}
Пираты  \par
Полный золотого песка торговый корабль был захвачен  $33$-мя  абсолютно рациональными пиратами! \par
У пиратов есть строгая иерархия: капитан, первый помощник капитана, второй помощник и т.д. Пираты делят золото так: сначала капитан предлагает свой вариант дележа, затем пираты голосуют за или против, если дележ одобрен не менее чем половиной пиратов (включая предложившего дележ), то он принимается, если нет, то капитана убивают, и дележ предлагает первый помощник...\par
Каждый пират хочет остаться в живых и получить побольше золота. При одинаковых выгодах для себя пират голосует за тот вариант, где в живых остается больше сотоварищей.\par

Какой дележ будет реализован?

\vspace{10pt}
Решаем с конца. Для удобства занумеруем пиратов начиная с самого младшего (номер 1 - зеленый юнга, ..., номер $33$ - капитан). Если в живых остался один пират, то он предлагает все себе и сам же одобряет этот дележ. Если в живых осталось два пирата ($n=2$), то предлагающий дележ забирает все себе. Он сам составляет ровно 50\%. Если в живых осталось три пирата, то пират номер 3 предлагает все себе. Сам он одобряет этот дележ, юнга одобряет (ему все равно ничего не достанется), только пират номер 2 против. Дележ одобрен. Если осталось $n>3$ пиратов, то пират номер $n$ предлагает все себе. Все пираты кроме пирата номер $(n-1)$ одобряют этот дележ.


\vspace{10pt}
Студенты и экзамен \par
Петя и Вася прогуляли экзамен... Они знали, что профессор очень любит путешествия, и придумали для него историю про то, как они отправились в автомобильное путешествие по России и очень хотели вернуться в день экзамена, но по дороге обратно у них сломалось колесо... Профессор согласился дать им отдельный экзамен. Он посадил их по разным аудиториям и задал один и тот же вопрос: <<Какое колесо сломалось?>>\par
а)	Составьте матрицу для этой игры;\par
б)	Сколько равновесий по Нэшу существует в данной игре?\par

\vspace{10pt}
Матрица - 4 на 4 с равновесиями по Нэшу по диагонали. Можно, например, считать, что если выбор игроков совпал, то они получают по единичке, а если не совпал - то ноль. Мораль: равновесий по Нэшу бывает много. Когда их много - непонятно, какое будет сыграно.

\vspace{10pt}
Limbo

До весны 2007 года в Швеции существовала необычная лотерея <<Limbo>>. Правила выглядят следующим образом. Вы можете выбрать любое натуральное число. Победителем объявляется тот, кто назвал самое маленькое число, никем более не названное. Например, если игроки назвали числа 1, 3, 1, 2, 4, то победителем будет тот, кто назвал число 2. Если наименьшего никем более не названного нет, то приз остается у организаторов. \par
Допустим, что в игру играет 100 игроков. Найдите несколько равновесий по Нэшу. 

\vspace{10pt}
Например: один игрок называет 1, остальные все что угодно больше 1. Или: один игрок называет 3, два игрока называют 1, 7 игроков называют 2, остальные называют все что угодно больше 3.




\vspace{10pt}
Модель Курно. Аня и Борис нашли родник с прекрасной водой. Они независимо друг от друга выбирают какое количество воды продавать. Набирание воды из родника не связано с издержками. Если Аня продает $a$ литров воды, а Борис - $b$ литров, то на воду установится цена $1-a-b$ и вся вода будет продана.

а) Найдите прибыль Ани и прибыль Бориса (это функции от $a$ и $b$)

б) Предположим, что после окончания этой игры Борис узнал, что Аня продавала $a$ литров воды. Какое $b$ было бы наилучшим выбором при заданном $a$, т.е. не вызвало бы у Бориса сожаления о своих действиях?

в) Предположим, что после окончания этой игры Аня узнала, что Борис продавал $b$ литров воды. Какое $a$ было бы наилучшим выбором при заданном $b$, т.е. не вызвало бы у Ани сожаления о своих действиях?

г) Воспользовавшись уравнениями из <<б>> и <<в>> найдите равновесие по Нэшу в этой игре.

\vspace{10pt}
а) Прибыли: $\pi_{a}=a(1-a-b)$, $\pi_{b}=b(1-a-b)$. 

б) Заметим, что относительно переменной $b$ функция $\pi_{b}=b(1-a-b)$ - это парабола. Корни у нее - $b=0$ и $b=1-a$. Ветви направлены вниз, значит максимум посередине, т.е. наилучшее (не вызывающее сожаление) $b$ при ставшем известном $a$ - это $b=(1-a)/2$.

в) Аналогично, получаем второе уравнение системы $a=(1-b)/2$




\vspace{10pt}
Модель Бертрана. Аня и Борис нашли родник с прекрасной водой. Они независимо друг от друга выбирают цену, по которой они продают. Если они назвали одинаковую цену $p$, то каждый продаст $(1-p)/2$ литров воды. Если цены отличаются, то тот, кто установил наименьшую цену $p$, продаст $(1-p)$ литров воды. Набирание воды из родника не связано с издержками. Найдите равновесие по Нэшу.

\vspace{10pt}
Решаем методом проб и ошибок. Пробуем (0.1,0.3) - не равновесие. Пробуем (0.2,0.2) - не равновесие. Кто-то, думаю, догадается, что равновесие - (0,0).

Мораль: разные модели одной и той же реальной ситуации дают разные равновесия. Какая наиболее приближена к реальности - это вопрос спорный.


\vspace{10pt}
Модель Бертрана с компенсацией. Аня и Борис нашли родник с прекрасной водой. Они независимо друг от друга выбирают цену, по которой они продают. Оба они взяли на себя обязательства компенсировать покупателю разницу в цене, если конкурент продает дешевле. Например, если они назначили цены 0.1 и 0.2, то покупателям все равно, где покупать, потому что у Ани вода продается за 0.1, а у Бориса за 0.2, но 0.1 он компенсирует. Найдите равновесия по Нэшу.


\vspace{10pt}
Решаем методом проб и ошибок. Пробуем (0.2,0.2) - равновесие! Т.к. при любых ценниках фактические цены продажи оказываются одинаковы и у Ани, и у Бориса покупают одинаковое количество воды. Единственное ограничение: если рассмотреть одного продавца, продающего по цене $p$, то его прибыль равна $p(1-p)$ и вершина этой параболы - $p=1/2$. Значит цены типа $(0.7,0.8)$ - не равновесие, каждый игрок сожалеет о том, что не снизил цены до 0.5. Цены типа $(0.2,0.1)$ - также не равновесны, т.к. игроку с наименьшей ценой выгодно поднять цены. А вот все цены типа $(x,x)$, где $x\in[0;1]$ - равновесны.

Мораль: до обязательства компенсировать разницу - равновесие было одно $(0,0)$. Теперь появились более хорошие для продавцов равновесия.



\bibliography{/home/boris/Dropbox/tex_general/opit}
\printindex % печать предметного указателя здесь

\end{document}