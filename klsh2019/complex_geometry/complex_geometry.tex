\documentclass[12pt]{article}

\usepackage{hyperref} % гиперссылки

\usepackage{tikz} % картинки в tikz
\usetikzlibrary{arrows.meta} % tikz-прибамбас для рисовки стрелочек подлиннее

\usepackage{microtype} % свешивание пунктуации

\usepackage{array} % для столбцов фиксированной ширины

\usepackage{indentfirst} % отступ в первом параграфе

\usepackage{sectsty} % для центрирования названий частей
\allsectionsfont{\centering}

\usepackage{amsmath} % куча стандартных математических плюшек
\usepackage{amssymb} % символы
\usepackage{amsthm} % теоремки

\usepackage{comment} % добавление длинных комментариев

\usepackage[top=2cm, left=1.2cm, right=1.2cm, bottom=2cm]{geometry} % размер текста на странице

\usepackage{lastpage} % чтобы узнать номер последней страницы

\usepackage{enumitem} % дополнительные плюшки для списков
%  например \begin{enumerate}[resume] позволяет продолжить нумерацию в новом списке

\usepackage{caption} % что-то делает с подписями рисунков :)

\usepackage{qcircuit} % для рисовки квантовых диаграмм
\usepackage{physics} % бракеты

\usepackage{answers} % разделение условий и ответов в упражнениях
\usepackage{multicol}

\usepackage{fancyhdr} % весёлые колонтитулы
\pagestyle{fancy}
\lhead{Комплексные числа и геометрия}
\chead{}
\rhead{КЛШ-2019}
\lfoot{}
\cfoot{}
\rfoot{\thepage/\pageref{LastPage}}
\renewcommand{\headrulewidth}{0.4pt}
\renewcommand{\footrulewidth}{0.4pt}



\usepackage{todonotes} % для вставки в документ заметок о том, что осталось сделать
% \todo{Здесь надо коэффициенты исправить}
% \missingfigure{Здесь будет Последний день Помпеи}
% \listoftodos — печатает все поставленные \todo'шки



\usepackage{booktabs} % красивые таблицы
% заповеди из докупентации:
% 1. Не используйте вертикальные линни
% 2. Не используйте двойные линии
% 3. Единицы измерения - в шапку таблицы
% 4. Не сокращайте .1 вместо 0.1
% 5. Повторяющееся значение повторяйте, а не говорите "то же"



\usepackage{fontspec} % что-то про шрифты?
\usepackage{polyglossia} % русификация xelatex

\setmainlanguage{russian}
\setotherlanguages{english}

% download "Linux Libertine" fonts:
% http://www.linuxlibertine.org/index.php?id=91&L=1
\setmainfont{Linux Libertine O} % or Helvetica, Arial, Cambria
% why do we need \newfontfamily:
% http://tex.stackexchange.com/questions/91507/
\newfontfamily{\cyrillicfonttt}{Linux Libertine O}

\AddEnumerateCounter{\asbuk}{\russian@alph}{щ} % для списков с русскими буквами
\setlist[enumerate, 2]{label=\asbuk*),ref=\asbuk*}

%% эконометрические сокращения
\DeclareMathOperator{\Cov}{Cov}
\DeclareMathOperator{\Arg}{Arg}
\let\arg\relax
\DeclareMathOperator{\arg}{arg}
\DeclareMathOperator{\Corr}{Corr}
\DeclareMathOperator{\Var}{Var}
\DeclareMathOperator{\E}{\mathbb{E}}
\def \hb{\hat{\beta}}
\def \hs{\hat{\sigma}}
\def \htheta{\hat{\theta}}
\def \s{\sigma}
\def \hy{\hat{y}}
\def \hY{\hat{Y}}
\def \v1{\vec{1}}
\def \e{\varepsilon}
\def \he{\hat{\e}}
\def \z{z}
\def \RR{\mathbb{R}}
\def \hVar{\widehat{\Var}}
\def \hCorr{\widehat{\Corr}}
\def \hCov{\widehat{\Cov}}
\def \cN{\mathcal{N}}
\let\P\relax
\DeclareMathOperator{\P}{\mathbb{P}}



\usepackage[bibencoding = auto,
backend = biber,
sorting = none,
style=alphabetic]{biblatex}

\addbibresource{em1_pset_v2.bib}



% делаем короче интервал в списках
\setlength{\itemsep}{0pt}
\setlength{\parskip}{0pt}
\setlength{\parsep}{0pt}




\Newassociation{sol}{solution}{solution_file}
% sol --- имя окружения внутри задач
% solution --- имя окружения внутри solution_file
% solution_file --- имя файла в который будет идти запись решений
% можно изменить далее по ходу
\Opensolutionfile{solution_file}[all_solutions]
% в квадратных скобках фактическое имя файла

% магия для автоматических гиперссылок задача-решение
\newlist{myenum}{enumerate}{3}
% \newcounter{problem}[chapter] % нумерация задач внутри глав
\newcounter{problem}[section]

\newenvironment{problem}%
{%
\refstepcounter{problem}%
%  hyperlink to solution
     \hypertarget{problem:{\thesection.\theproblem}}{} % нумерация внутри глав
     % \hypertarget{problem:{\theproblem}}{}
     \Writetofile{solution_file}{\protect\hypertarget{soln:\thesection.\theproblem}{}}
     %\Writetofile{solution_file}{\protect\hypertarget{soln:\theproblem}{}}
     \begin{myenum}[label=\bfseries\protect\hyperlink{soln:\thesection.\theproblem}{\thesection.\theproblem},ref=\thesection.\theproblem]
     % \begin{myenum}[label=\bfseries\protect\hyperlink{soln:\theproblem}{\theproblem},ref=\theproblem]
     \item%
    }%
    {%
    \end{myenum}}
% для гиперссылок обратно надо переопределять окружение
% это происходит непосредственно перед подключением файла с решениями



\theoremstyle{definition}
\newtheorem{definition}{Определение}



\begin{document}

\tableofcontents{}

\section*{Цель}

Рассказать про комплексный числа, преобразование Мёбиуса, кватернионы, вращения, роторы, октонионы, гиперболическую и проективную геометрию.

\begin{itemize}
  \item ччч
\end{itemize}


\newpage
\section{Комплексные числа. Определение}

\begin{definition}
  Комплексное число — это вектор на плоскости.
\end{definition}
  
  \begin{enumerate}
    \item Длина вектора — модуль комплексного числа, $|z|$.
    \item Угол между вектором и горизонатльной осью — аргумента комплексного числа, $\arg z$.
    \item Горизонтальная составляющая вектора — действительная часть, $\Re z$.
    \item Вертикальная составляющая вектора — мнимая часть, действительное число, $\Im z$.
  \end{enumerate}
  
  \begin{problem}
  Поехали.
  \begin{enumerate}
    \item Для комплексных чисел $1+i$ и $3+4i$ найди $|z|$, $\arg z$, $\Re z$, $\Im z$. 
    \item Нарисуй числа $1+i$, $3+4i$, $3-i$, $-3i$.
  \end{enumerate}
  
  \begin{sol}
  \end{sol}
  \end{problem}
  
  
  Действия:
  \begin{enumerate}
    \item Сложение комплексных чисел — сложение векторов.
    \item Умножение комплексных чисел — длины векторов умножаются, аргументы складываются.
    \item Сопряжение $z^*$ комплексного числа — отражение относительно горизонтальной оси.
  \end{enumerate}
  
  
  \begin{problem}
  Базируясь на геометрическом определении умножения, ответь на вопросы:
  
  \begin{enumerate}
    \item Чему равняется $(1+i)^2$? $(1+i)^{43}$?
    \item Почему $i^2=-1$?
    \item Чему равняется произведение $z=6 + 3i$ на $i$?
  \end{enumerate}
  \begin{sol}
  \end{sol}
  \end{problem}
 
  Наивное умножение комплексных чисел. Раскрываем скобки и упрощаем по принципу $i^2=-1$.

  \begin{problem}
  Нарисуй процесс умножение произвольного $z$ на $3+4i$. 
  А именно, нарисуй $3z$, $4iz$, $(3+4i)z$ и по рисунку объясни, почему $(3+4i)z = 3z + 4iz$.
  \begin{sol}
  \end{sol}
  \end{problem}
  
  
  
  \begin{problem}
  
  \begin{enumerate}
    \item У комплексного числа $w = \sqrt{11} + 5i$ найди $|w|$, $|w|^2$, $\Arg w$, $\Re w$, $\Im w$, $w^*$, $ww^*$.
    \item Найди $(3+5i) \cdot (3+3i)$, $(1+i)/(1-i)$,
    \item Найди $(\sqrt{3}+i)^{43}$, $(1-i)^{2018}$;
    \item Найди $(\cos (20^{\circ}) + i \sin (20^{\circ})) \cdot (\cos (10^{\circ}) + i \sin (10^{\circ}))$;
    \item Найди $(\cos (20^{\circ}) + i \sin (20^{\circ})) / (\cos (10^{\circ}) + i \sin (10^{\circ}))$;
  \end{enumerate}
  
  \begin{sol}
  \end{sol}
  \end{problem}
  
  
  \begin{problem}
  Реши уравнения $z^2 = -1$, $z^2 + 6z + 10 = 0$, $z^6 = 64$, $(z-1) / (z + 1) = 1 + 3i$.
  \begin{sol}
  \end{sol}
  \end{problem}
  
  
  \begin{problem}
  Найди суммы $1 + i + i^2 + i^3 + i^4 + \ldots + i^{2019}$,
  $(1+i) + (1+i)^2 + (1+i)^3 + (1+i)^4 + \ldots + (1+i)^{2020}$.
  \begin{sol}
  \end{sol}
  \end{problem}
  
  
  \begin{problem}
    Бесконечно живущая черепаха за первый день проходит 10 км на север.
    Затем каждый день она поворачивает на $90^{\circ}$ налево и
    снижает скорость на 20\%. К какой точке она приближается?
  
    К какой точке стремится черепах, если она поворачивает на $60^{\circ}$?
  \begin{sol}
  \end{sol}
  \end{problem}
  
  
  
  
  \begin{problem}
    Найди сумму углов между векторами и горизонтальной осью.
  
  
    \begin{minipage}{0.8\textwidth}
    \begin{center}
    \begin{tikzpicture}
    \draw (0,0) -- (4,0);
    \draw (0,1) -- (4,1);
    \draw (0,2) -- (4,2);
    \draw (0,0) -- (0,2);
    \draw (1,0) -- (1,2);
    \draw (2,0) -- (2,2);
    \draw (3,0) -- (3,2);
    \draw (4,0) -- (4,2);
    \draw[-{Latex[length=5mm, width=2mm]}] (0,0) -- (4,2);
    \draw[-{Latex[length=5mm, width=2mm]}] (0,0) -- (3,1);
    \end{tikzpicture}
    \end{center}
    \end{minipage}
  \begin{sol}
    $(4+2i)(3+i) = 10 + 10i$, $\pi/4$.
  \end{sol}
  \end{problem}
  
  
  
  \begin{problem}
  На плоскости нарисована кошечка. Что прозойдет с кошечкой,
  если каждую точку кошечки домножить на комплексное число $1/\sqrt{2} + i/\sqrt{2}$?
  \begin{sol}
    Кошка повернётся на $\pi/4$ против часовой стрелки относительно начала координат
  \end{sol}
  \end{problem}
  

\newpage
\setcounter{section}{0}
\section{Комплексные числа. Определение}

\begin{definition}
  Комплексное число — это вектор на плоскости.
\end{definition}
  
\begin{enumerate}
  \item Длина вектора — модуль комплексного числа, $|z|$.
  \item Угол между вектором и горизонатльной осью — аргумента комплексного числа, $\arg z$.
  \item Горизонтальная составляющая вектора — действительная часть, $\Re z$.
  \item Вертикальная составляющая вектора — мнимая часть, действительное число, $\Im z$.
\end{enumerate}

\begin{problem}
Поехали.
\begin{enumerate}
  \item Для комплексных чисел $1+i$ и $3+4i$ найди $|z|$, $\arg z$, $\Re z$, $\Im z$. 
  \item Нарисуй числа $1+i$, $3+4i$, $3-i$, $-3i$.
\end{enumerate}

\begin{sol}
\end{sol}
\end{problem}


Действия:
\begin{enumerate}
  \item Сложение комплексных чисел — сложение векторов.
  \item Умножение комплексных чисел — длины векторов умножаются, аргументы складываются.
  \item Сопряжение $z^*$ комплексного числа — отражение относительно горизонтальной оси.
\end{enumerate}


\begin{problem}
Базируясь на геометрическом определении умножения, ответь на вопросы:

\begin{enumerate}
  \item Чему равняется $(1+i)^2$? $(1+i)^{43}$?
  \item Почему $i^2=-1$?
  \item Чему равняется произведение $z=6 + 3i$ на $i$?
\end{enumerate}
\begin{sol}
\end{sol}
\end{problem}

Наивное умножение комплексных чисел. Раскрываем скобки и упрощаем по принципу $i^2=-1$.

\begin{problem}
Нарисуй процесс умножение произвольного $z$ на $3+4i$. 
А именно, нарисуй $3z$, $4iz$, $(3+4i)z$ и по рисунку объясни, почему $(3+4i)z = 3z + 4iz$.
\begin{sol}
\end{sol}
\end{problem}



\begin{problem}

\begin{enumerate}
  \item У комплексного числа $w = \sqrt{11} + 5i$ найди $|w|$, $|w|^2$, $\Arg w$, $\Re w$, $\Im w$, $w^*$, $ww^*$.
  \item Найди $(3+5i) \cdot (3+3i)$, $(1+i)/(1-i)$,
  \item Найди $(\sqrt{3}+i)^{43}$, $(1-i)^{2018}$;
  \item Найди $(\cos (20^{\circ}) + i \sin (20^{\circ})) \cdot (\cos (10^{\circ}) + i \sin (10^{\circ}))$;
  \item Найди $(\cos (20^{\circ}) + i \sin (20^{\circ})) / (\cos (10^{\circ}) + i \sin (10^{\circ}))$;
\end{enumerate}

\begin{sol}
\end{sol}
\end{problem}


\begin{problem}
Реши уравнения $z^2 = -1$, $z^2 + 6z + 10 = 0$, $z^6 = 64$, $(z-1) / (z + 1) = 1 + 3i$.
\begin{sol}
\end{sol}
\end{problem}


\begin{problem}
Найди суммы $1 + i + i^2 + i^3 + i^4 + \ldots + i^{2019}$,
$(1+i) + (1+i)^2 + (1+i)^3 + (1+i)^4 + \ldots + (1+i)^{2020}$.
\begin{sol}
\end{sol}
\end{problem}


\begin{problem}
  Бесконечно живущая черепаха за первый день проходит 10 км на север.
  Затем каждый день она поворачивает на $90^{\circ}$ налево и
  снижает скорость на 20\%. К какой точке она приближается?

  К какой точке стремится черепах, если она поворачивает на $60^{\circ}$?
\begin{sol}
\end{sol}
\end{problem}




\begin{problem}
  Найди сумму углов между векторами и горизонтальной осью.


  \begin{minipage}{0.8\textwidth}
  \begin{center}
  \begin{tikzpicture}
  \draw (0,0) -- (4,0);
  \draw (0,1) -- (4,1);
  \draw (0,2) -- (4,2);
  \draw (0,0) -- (0,2);
  \draw (1,0) -- (1,2);
  \draw (2,0) -- (2,2);
  \draw (3,0) -- (3,2);
  \draw (4,0) -- (4,2);
  \draw[-{Latex[length=5mm, width=2mm]}] (0,0) -- (4,2);
  \draw[-{Latex[length=5mm, width=2mm]}] (0,0) -- (3,1);
  \end{tikzpicture}
  \end{center}
  \end{minipage}
\begin{sol}
  $(4+2i)(3+i) = 10 + 10i$, $\pi/4$.
\end{sol}
\end{problem}



\begin{problem}
На плоскости нарисована кошечка. Что прозойдет с кошечкой,
если каждую точку кошечки домножить на комплексное число $1/\sqrt{2} + i/\sqrt{2}$?
\begin{sol}
  Кошка повернётся на $\pi/4$ против часовой стрелки относительно начала координат
\end{sol}
\end{problem}
    

\newpage
\setcounter{section}{1}
\section{Комплексные числа. Геометрия и картинки}


\begin{problem}
Рассмотрим произвольный четырёхугольник. Снаружи каждой стороны четырёхугольника построим квадрат.
Назовём отрезки, соединяющие центры противоположных квадратов, $MN$ и $KL$. 

\begin{enumerate}
  \item Найди угол между $MN$ и $KL$.
  \item Найди отношение длин $MN$ и $KL$.
\end{enumerate}

\begin{sol}
\end{sol}
\end{problem}



\begin{problem}
Нарисуй на комплексной плоскости множества

\begin{multicols}{2}
\begin{enumerate}
  \item $|z| = 4$;
  \item $|z - 2 + 3i| > 5$;
  \item $\Re z = 3$;
  \item $\Im z < 6$;
  \item $1 < |2z - 6| < 2$;
  \item $|z - 1|^2 + |z + 1|^2 < 8$;
  \item $|z - 1| + |z + 1| \leq 2$;
  \item $|\Re z | < |z|$;
  \item $|z - i| = |z - (3 + 2i)|$;
  \item $\Re ((1+i)z) > 2$; 
  \item $\Re \left( \frac{z - 1 - i}{z + 1 + i} \right) = 0$;
  \item $\Im \left( \frac{z - 1 - i}{z + 1 + i} \right) = 0$;
\end{enumerate}
\end{multicols}
\begin{sol}
\end{sol}
\end{problem}
  


\begin{problem}
Нарисуй на комплексной плоскости траектории, $t \to z(t)$, для $t \in \RR$, отметив стрелкой направление:

\begin{enumerate}
  \item $t \to 6 + it$;
  \item $t \to t+2 + 7i$;
  \item $t \to t+2 + it$;
  \item $t \to t + it^2$;
  \item $t \to \cos t + i \sin t$;
  \item $t \to t \cdot (\cos t + i \sin t)$;
  \item $t \to t \cdot (\cos t - i \sin t)$;
\end{enumerate}

\begin{sol}
\end{sol}
\end{problem}
  

\begin{problem}
Нарисуй комплексные числа $z_1$ и $z_2$ с единичной длиной и аргументами $\pi/4$ и $\pi/2$. 
\begin{enumerate}
  \item   Запиши $z_1$, $z_2$ и $z_1 + z_2$ в виде $a + bi$. 
  \item Найди $\tan 3\pi/8$;
\end{enumerate}



\begin{sol}
\end{sol}
\end{problem}

\newpage
\setcounter{section}{1}
\section{Комплексные числа. Геометрия и картинки}


\begin{problem}
Рассмотрим произвольный четырёхугольник. Снаружи каждой стороны четырёхугольника построим квадрат.
Назовём отрезки, соединяющие центры противоположных квадратов, $MN$ и $KL$. 

\begin{enumerate}
  \item Найди угол между $MN$ и $KL$.
  \item Найди отношение длин $MN$ и $KL$.
\end{enumerate}

\begin{sol}
\end{sol}
\end{problem}



\begin{problem}
Нарисуй на комплексной плоскости множества
\begin{multicols}{2}
\begin{enumerate}
  \item $|z| = 4$;
  \item $|z - 2 + 3i| > 5$;
  \item $\Re z = 3$;
  \item $\Im z < 6$;
  \item $1 < |2z - 6| < 2$;
  \item $|z - 1|^2 + |z + 1|^2 < 8$;
  \item $|z - 1| + |z + 1| \leq 2$;
  \item $|\Re z | < |z|$;
  \item $|z - i| = |z - (3 + 2i)|$;
  \item $\Re ((1+i)z) > 2$; 
  \item $\Re \left( \frac{z - 1 - i}{z + 1 + i} \right) = 0$;
  \item $\Im \left( \frac{z - 1 - i}{z + 1 + i} \right) = 0$;
\end{enumerate}
\end{multicols}
\begin{sol}
\end{sol}
\end{problem}
  


\begin{problem}
Нарисуй на комплексной плоскости траектории, $t \to z(t)$, для $t \in \RR$, отметив стрелкой направление:

\begin{enumerate}
  \item $t \to 6 + it$;
  \item $t \to t+2 + 7i$;
  \item $t \to t+2 + it$;
  \item $t \to t + it^2$;
  \item $t \to \cos t + i \sin t$;
  \item $t \to t \cdot (\cos t + i \sin t)$;
  \item $t \to t \cdot (\cos t - i \sin t)$;
\end{enumerate}

\begin{sol}
\end{sol}
\end{problem}
  

\begin{problem}
Нарисуй комплексные числа $z_1$ и $z_2$ с единичной длиной и аргументами $\pi/4$ и $\pi/2$. 
\begin{enumerate}
  \item   Запиши $z_1$, $z_2$ и $z_1 + z_2$ в виде $a + bi$. 
  \item Найди $\tan 3\pi/8$;
\end{enumerate}



\begin{sol}
\end{sol}
\end{problem}
  

\newpage
\setcounter{section}{2}
\section{Поле направления и экспонента}


\begin{definition}
  Если $z(t)$ — положение точки в момент $t$, то $\dot z(t)$ или $z'(t)$ — мгновенная скорость точки (вектор).
\end{definition}

\begin{definition}
  Поле направления — в каждой точки плоскости нарисован вектор скорости движения точки.
\end{definition}


\begin{problem}
 Нарисуй поле направления для каждого случая:
 \begin{multicols}{3}
 \begin{enumerate}
   \item $\dot z (t) = 1$;
   \item $\dot z (t) = i$;
   \item $\dot z (t) = z(t)$;
   \item $\dot z (t) = -z(t)$;
   \item $\dot z (t) = iz(t)$;
   \item $\dot z (t) = -iz(t)$;
   \item $\dot z (t) = 2 - z(t)$;
   \item $\dot z (t) = 2 - iz(t)$;
 \end{enumerate}
\end{multicols}
\begin{sol}
\end{sol}
\end{problem}
  

\begin{definition}
  Экспонента $\exp(t)$ — функция $z(t)$ со свойствами $z(0) = 1$, $\dot z(t) = z(t)$.

  Экспонента $\exp(it)$ — функция $z(t)$ со свойствами $z(0) = 1$, $\dot z(t) = i z(t)$.
\end{definition}


\begin{problem}
Докажи, что 
\begin{enumerate}
  \item $\exp(1) \approx 1.01^{100}$;
  \item $\exp(2) = \exp(1) \cdot \exp(1)$;
  \item $\exp(3) = \exp(1) \cdot \exp(1) \cdot \exp(1)$;
\end{enumerate}
\begin{sol}
\end{sol}
\end{problem}

\begin{problem}
Найди
\begin{multicols}{3}
\begin{enumerate}
  \item $\exp(i \pi/3)$;
  \item $\exp(i \pi/2)$;
  \item Формула Эйлера! $\exp(i \pi)$;
  \item $\exp(i t)$;
  \item $\exp(i \pi/3) \cdot \exp(i \pi /2)$;
\end{enumerate}
\end{multicols}
\begin{sol}
\end{sol}
\end{problem}
  
\begin{problem}
Запиши комплексные числа с помощью экспоненты
\begin{enumerate}
  \item $1 + i$;
  \item $\sqrt{3} + i$;
  \item $\sqrt{3} - i$;
  \item $6i$;
\end{enumerate}
\begin{sol}
\end{sol}
\end{problem}
  



\begin{problem}
Реши уравнения
\begin{multicols}{3}
\begin{enumerate}
  \item $z^2 = 6$;
  \item $z^2 = -9$;
  \item $z^2 = 4i$;
  \item $z^3 = -27$;
  \item $z^2 = -4i$;
  \item $z^2 + 4z + 13 = 0$;
  \item $\frac{z + i + 2}{z - i - 3} = 4i$;
  \item $z^3 + z^2 + z - 3 = 0$;
  \item $z^5 = 32$;
  \item $z^6 = i$;
  \item $z^7 = 1 - i$;
\end{enumerate}
\end{multicols}
\begin{sol}
\end{sol}
\end{problem}
 

\newpage
\setcounter{section}{2}
\section{Поле направления и экспонента}


\begin{definition}
  Если $z(t)$ — положение точки в момент $t$, то $\dot z(t)$ или $z'(t)$ — мгновенная скорость точки (вектор).
\end{definition}

\begin{definition}
  Поле направления — в каждой точки плоскости нарисован вектор скорости движения точки.
\end{definition}


\begin{problem}
 Нарисуй поле направления для каждого случая:
 \begin{multicols}{3}
 \begin{enumerate}
   \item $\dot z (t) = 1$;
   \item $\dot z (t) = i$;
   \item $\dot z (t) = z(t)$;
   \item $\dot z (t) = -z(t)$;
   \item $\dot z (t) = iz(t)$;
   \item $\dot z (t) = -iz(t)$;
   \item $\dot z (t) = 2 - z(t)$;
   \item $\dot z (t) = 2 - iz(t)$;
 \end{enumerate}
\end{multicols}
\begin{sol}
\end{sol}
\end{problem}
  

\begin{definition}
  Экспонента $\exp(t)$ — функция $z(t)$ со свойствами $z(0) = 1$, $\dot z(t) = z(t)$.

  Экспонента $\exp(it)$ — функция $z(t)$ со свойствами $z(0) = 1$, $\dot z(t) = i z(t)$.
\end{definition}


\begin{problem}
Докажи, что 
\begin{enumerate}
  \item $\exp(1) \approx 1.01^{100}$;
  \item $\exp(2) = \exp(1) \cdot \exp(1)$;
  \item $\exp(3) = \exp(1) \cdot \exp(1) \cdot \exp(1)$;
\end{enumerate}
\begin{sol}
\end{sol}
\end{problem}

\begin{problem}
Найди
\begin{multicols}{3}
\begin{enumerate}
  \item $\exp(i \pi/3)$;
  \item $\exp(i \pi/2)$;
  \item Формула Эйлера! $\exp(i \pi)$;
  \item $\exp(i t)$;
  \item $\exp(i \pi/3) \cdot \exp(i \pi /2)$;
\end{enumerate}
\end{multicols}
\begin{sol}
\end{sol}
\end{problem}
  
\begin{problem}
Запиши комплексные числа с помощью экспоненты
\begin{enumerate}
  \item $1 + i$;
  \item $\sqrt{3} + i$;
  \item $\sqrt{3} - i$;
  \item $6i$;
\end{enumerate}
\begin{sol}
\end{sol}
\end{problem}
  



\begin{problem}
Реши уравнения
\begin{multicols}{3}
\begin{enumerate}
  \item $z^2 = 6$;
  \item $z^2 = -9$;
  \item $z^2 = 4i$;
  \item $z^3 = -27$;
  \item $z^2 = -4i$;
  \item $z^2 + 4z + 13 = 0$;
  \item $\frac{z + i + 2}{z - i - 3} = 4i$;
  \item $z^3 + z^2 + z - 3 = 0$;
  \item $z^5 = 32$;
  \item $z^6 = i$;
  \item $z^7 = 1 - i$;
\end{enumerate}
\end{multicols}
\begin{sol}
\end{sol}
\end{problem}



\newpage
\setcounter{section}{3}
\section{Кубические уравнения}

\begin{problem}
Найди все значения многозначной функции
\begin{multicols}{2}
\begin{enumerate}
  \item $8^{1/3}$;
  \item $i^{1/3}$;
  \item $(1+i)^{1/3}$;
  \item $(\sqrt{3} + i)^{1/3}$;
\end{enumerate}
\end{multicols}
\begin{sol}
\end{sol}
\end{problem}
  
\begin{problem}
Реши системы 
\begin{multicols}{2}
\begin{enumerate}
\item 
$
\begin{cases}
x + y + xy = 5 \\
x^2 + y^2 = 5 \\
\end{cases}
$
\item   
$
\begin{cases}
x y (x + y) = 30 \\
x^3 + y^3 = 35 \\
\end{cases}
$

\item 
$
\begin{cases}
x^2 + 3xy + y^2  = 79 \\
xy + y + x = 23 \\
\end{cases}
$;
\item %%% Исправить на хорошие коэффициенты!!!!!!!!!!!!!!!!
$
\begin{cases}
x^3 + y^3  = 10 \\
y \cdot x = 2 \\
\end{cases}
$;
\end{enumerate}
\end{multicols}
\begin{sol}
\end{sol}
\end{problem}

\begin{problem}
Реши кубическое уравнение
\begin{multicols}{3}
\begin{enumerate}
  \item $z^3 - 15z - 4  = 0$;
  \item $z^3 - 15z - 10 = 0$;
  \item $z^3 - 6z - 6 = 0$;
\end{enumerate}
\end{multicols}
\begin{sol}
\end{sol}
\end{problem}
  


 

\begin{problem}
Подбери число $t$ так, чтобы при замене $z = w + t$ в записи исчезло слагаемое $w^2$:
\begin{multicols}{3}
\begin{enumerate}
  \item $z^3 + 21z^2$;
  \item $z^3 - 9z^2$;
  \item $z^3 + 6z^2$;
\end{enumerate}
\end{multicols}
\begin{sol}
\end{sol}
\end{problem}

\begin{definition}
  Экспонента $\exp(a + bi) = \exp(a) \cdot \exp(b)$ — функция $z(t)$ со свойствами $z(0) = 1$, $\dot z(t) = (a+bi) \cdot z(t)$.
\end{definition}

\newpage
\setcounter{section}{4}
\section{Преобразования плоскости}
Нарисуй исходное множество $A$ и его образ $f(A)$ для случаев
\begin{problem}
\begin{multicols}{2}
\begin{enumerate}
  \item $A = \{ |z - 1| = 1\}$, $f(z) = z^2$;
  \item $A = \{ \Re z = 1 \}$, $f(z) = z^2$;
  \item $A = \{ \Im z = 1 \}$, $f(z) = z^2$;
  \item $A = \{ \Im z = (\Re z)^2 \}$, $f(z)=z^2$;
  \item $A = \{ \Re z = 4 \}$, $f(z) = (1+i) z$;
  \item $A = \{ \Re z = 4 \}$, $f(z) = \exp(z)$;
  \item $A = \{ \Im z = 4 \}$, $f(z) = \exp(z)$;
  \item $A = \{ \Re z = 4 \}$, $f(z) = 1/\bar z$;
  \item $A = \{ \Im z = 4 \}$, $f(z) = 1/\bar z$;
  \item $A = \{ \Im z = 0 \}$, $f(z) = 1/\bar z$;
  \item $A = \{ |z| = 2\}$, $f(z) = 1/\bar z$;
\end{enumerate}
\end{multicols}
\begin{sol}
\end{sol}
\end{problem}
  

\begin{definition}
  Комплексная инверсия $f : z \to 1/z$; 
  
  Геометрическая инверсия (просто инверсия): $f: z \to 1/\bar z$ и обобщение. 
\end{definition}

\begin{problem}
Нарисуй окружность с центром $Q$ и радиусом $r$. Нарисуй точки $A$ и $B$ внутри окружности и их образы $\tilde A$ и $\tilde B$ после инверсии. 
\begin{enumerate}
\item Найди подобные треугольники. 
\item Найди длину $\tilde A\tilde B$, если $QA=4$, $QB=6$, $r = 10$, $AB = 5$. 
\end{enumerate}
\begin{sol}
\end{sol}
\end{problem}
    
\begin{problem}
Свойства инверсии:
\begin{enumerate}
\item Что получится, если инверсию применить два раза?
\item Во что переходит сама окружность при инверсии?
\end{enumerate}
\begin{sol}
\end{sol}
\end{problem}
  

\begin{problem}
Нарисуй окружность с центром $Q$ и радиусом $r$. Во что перейдёт при инверсии:
\begin{enumerate}
\item Прямая $\ell$, проходящая через центр окружности $Q$.
\item Прямая $\ell$, не проходящая через центр окружности $Q$.
\end{enumerate}
\begin{sol}
\end{sol}
\end{problem}

\begin{problem}
Миша С. выполняет инверсию точки $A$ относительно окружности радиуса $m$ с центром в точке $Q$ и получает 
точку $\tilde A_1$.
Серёжа Л. выполняет инверсию той же точки $A$ относительно окружности радиуса $s$ с центром в точке $Q$ и
получает точку $\tilde A_2$. 

\begin{enumerate}
  \item Как будут соотносится длины отрезков $Q\tilde A_1$ и $Q \tilde A_2$? Как зависит это отношение от выбора точки $A$?
  \item Объясни содержательную разницу между инверсией Миши С. и Серёжи Л.
\end{enumerate}
\begin{sol}
\end{sol}
\end{problem}
    



\begin{problem}
Нарисуй окружность с центром $Q$ и радиусом $r$. Во что перейдёт при инверсии:
\begin{enumerate}
\item Окружность $w$, проходящая через центр исходной окружности $Q$.
\item Окружность $w$, не проходящая через центр иходной окружности $Q$. 
\end{enumerate}
\begin{sol}
\end{sol}
\end{problem}
  

  

\newpage
\setcounter{section}{4}
\section{Преобразования плоскости}
Нарисуй исходное множество $A$ и его образ $f(A)$ для случаев
\begin{problem}
\begin{multicols}{2}
\begin{enumerate}
  \item $A = \{ |z - 1| = 1\}$, $f(z) = z^2$;
  \item $A = \{ \Re z = 1 \}$, $f(z) = z^2$;
  \item $A = \{ \Im z = 1 \}$, $f(z) = z^2$;
  \item $A = \{ \Im z = (\Re z)^2 \}$, $f(z)=z^2$;
  \item $A = \{ \Re z = 4 \}$, $f(z) = (1+i) z$;
  \item $A = \{ \Re z = 4 \}$, $f(z) = \exp(z)$;
  \item $A = \{ \Im z = 4 \}$, $f(z) = \exp(z)$;
  \item $A = \{ \Re z = 4 \}$, $f(z) = 1/\bar z$;
  \item $A = \{ \Im z = 4 \}$, $f(z) = 1/\bar z$;
  \item $A = \{ \Im z = 0 \}$, $f(z) = 1/\bar z$;
  \item $A = \{ |z| = 2\}$, $f(z) = 1/\bar z$;
\end{enumerate}
\end{multicols}
\begin{sol}
\end{sol}
\end{problem}
  

\begin{definition}
  Комплексная инверсия $f : z \to 1/z$; 
  
  Геометрическая инверсия (просто инверсия): $f: z \to 1/\bar z$ и обобщение. 
\end{definition}

\begin{problem}
Нарисуй окружность с центром $Q$ и радиусом $r$. Нарисуй точки $A$ и $B$ внутри окружности и их образы $\tilde A$ и $\tilde B$ после инверсии. 
\begin{enumerate}
\item Найди подобные треугольники. 
\item Найди длину $\tilde A\tilde B$, если $QA=4$, $QB=6$, $r = 10$, $AB = 5$. 
\end{enumerate}
\begin{sol}
\end{sol}
\end{problem}
    
\begin{problem}
Свойства инверсии:
\begin{enumerate}
\item Что получится, если инверсию применить два раза?
\item Во что переходит сама окружность при инверсии?
\end{enumerate}
\begin{sol}
\end{sol}
\end{problem}
  

\begin{problem}
Нарисуй окружность с центром $Q$ и радиусом $r$. Во что перейдёт при инверсии:
\begin{enumerate}
\item Прямая $\ell$, проходящая через центр окружности $Q$.
\item Прямая $\ell$, не проходящая через центр окружности $Q$.
\end{enumerate}
\begin{sol}
\end{sol}
\end{problem}

\begin{problem}
Миша С. выполняет инверсию точки $A$ относительно окружности радиуса $m$ с центром в точке $Q$ и получает 
точку $\tilde A_1$.
Серёжа Л. выполняет инверсию той же точки $A$ относительно окружности радиуса $s$ с центром в точке $Q$ и
получает точку $\tilde A_2$. 

\begin{enumerate}
  \item Как будут соотносится длины отрезков $Q\tilde A_1$ и $Q \tilde A_2$? Как зависит это отношение от выбора точки $A$?
  \item Объясни содержательную разницу между инверсией Миши С. и Серёжи Л.
\end{enumerate}
\begin{sol}
\end{sol}
\end{problem}
    



\begin{problem}
Нарисуй окружность с центром $Q$ и радиусом $r$. Во что перейдёт при инверсии:
\begin{enumerate}
\item Окружность $w$, проходящая через центр исходной окружности $Q$.
\item Окружность $w$, не проходящая через центр иходной окружности $Q$. 
\end{enumerate}
\begin{sol}
\end{sol}
\end{problem}
  



\newpage
\section{Геометрия Фано}


Количество точек и прямых на проективной плоскости порядка такого-то?




\section{Лог. КЛШ-2019}

\begin{enumerate}
  \item Было 29 школьников, от 8-го до 10-го класса и одна храбрая семиклассница. Комплексное число — вектор на плоскости.
  Сложение и вычитание. Изобразите $3+4i$, $5i$, $-6 + i$, $-8$. Длина и аргумент. Многозначная функция. 
  Геометрическое умножение. Находим $(1+i)^{44}$. Геометрически считаем $i \cdot i$, $(5 + 6i) \cdot i$. Наивное умножение.
  Геометрически интерпретируем наивное умножение $z \cdot (3 + 4i)$.
  Рисуем число $\cos 40^{\circ} + i \sin 40^{\circ}$. Делим через домножение на сопряжённое.
  Делим геометрически. Находим сумму конечной геометрической прогрессии комплексных чисел. 
  \item Повторили основные мысли. Два варианта записи чисел. Явно $z=a + bi$, через длину и угол с косинусом и синусом.
  Решили задачу про сумму углов. Разобрал окружность с центром не в нуле. Далее школьники решали и сдавали номера.
  \item Решили задачу про сумму квадратов через явное представление $z=a+bi$.
  Дальше пообсуждали, что разумно сделать после решения задачи. 
  Придумать более простой метод. Придумать более универсальный метод. 
  Проверить, работает ли старый метод, если пошевилить задачу.
  Пошевелили нашу задачу и пришли к выводу, что геометрическо множество точек $Z$ таких, что
  $AZ^2 + BZ^2 = const$ — это окружность. 
  Влад, решивший дома задачу по геометрии с произвольным четырёхугольником, начал излагать её. 
  Чтобы ускорить процесс, я изложил за него. Затем кратко рассказал про кривые. 
  И школьники рисовали кривые. 
  \item Рисовали поле направлений. Хороший образ: нарисовать стрелочки ветра и куда несёт парашутиста. 
  Определили две экспоненты: $\exp(t)$ и $\exp(it)$. Посчитали вместе примерно $\exp(1)$, $\exp(2)$.
  Перевели запись $\exp(it)$ для хороших $t$ в координатную форму. Подытожили три формы записи комплексных чисел. 
  \item Повторили три формы записи комплексных чисел. Эффективнее всего решать уравнения табличкой. Хотя до этого процесса мы 
  дошли только в конце. 
  Берём исходное число записываем его в виде $27 \exp(120^{\circ} + 360^{\circ}k)$. Пишем длину, угол. 
  Далее табличкой пишем то же самое для нескольких $k$. Затем в общей записи и в примерах делим угол на три, а из длины извлекаем 
  кубический корень. Изображаем четыре кандидата, замечаем, что три кандидата совпадают. Записываем каждого кандидата в координатной форме записи.
  После этого школьники решали сами задачи на нахождение корней. Из-за не оптимального рассказа после перерыва ещё раз изложил алгоритм решения. 
  \item Вспомнили экспоненту для действительных чисел, $\exp(t)$, экспоненту от чисто мнимых, $\exp(it)$. 
  И определили экспоненту от комплексных чисел $\exp(a + bi)$. Доказали через выделение полного квадрата, что дискриминант работает 
  для квадратного уравнения. Многие школьники немного удивлённо узнали, 
  что дискриминант — это то, что остаётся в правой части после умножения уравнения на 4 и выделения полного квадрата. 
  Далее я рассказал про симметричную замену. И пример, где симметричная замена в системе не работает, но создаёт мостик до кубического уравнения. 
  Начали решать кубическое уравнение. Схематично обозначил окончание. Надо было взять хорошие коэффициенты. Плохие коэффициенты от фонаря — резкое препятствие!
  \item Преобразования плоскости. 
  Реакция на критику: нет парт — увы. Немного темно — увы. Мотивационное: квантовые вычисления и преобразование Лоренца. 
  Свойства инверсии. 



\end{enumerate}

\subsection{Плакат}





\Closesolutionfile{solution_file}

% для гиперссылок на условия
% http://tex.stackexchange.com/questions/45415
\renewenvironment{solution}[1]{%
         % add some glue
         \vskip .5cm plus 2cm minus 0.1cm%
         {\bfseries \hyperlink{problem:#1}{#1.}}%
}%
{%
}%



\section{Решения}
\protect \hypertarget {soln:1.1}{}
\begin{solution}{{1.1}}
  $\P(X=1)=3/5$, $\P(X=2)=3/10$, $\P(X=3)=1/10$, $\E(X)=1.5$
\end{solution}
\protect \hypertarget {soln:1.2}{}
\begin{solution}{{1.2}}
\end{solution}
\protect \hypertarget {soln:1.3}{}
\begin{solution}{{1.3}}
\end{solution}
\protect \hypertarget {soln:1.4}{}
\begin{solution}{{1.4}}
   N 3 4 5

  2/8 3/8 3/8
\end{solution}
\protect \hypertarget {soln:2.1}{}
\protect \hypertarget {soln:2.2}{}
\protect \hypertarget {soln:2.3}{}
\protect \hypertarget {soln:2.4}{}
\protect \hypertarget {soln:3.1}{}
\begin{solution}{{3.1}}
\end{solution}
\protect \hypertarget {soln:3.2}{}
\begin{solution}{{3.2}}
\end{solution}
\protect \hypertarget {soln:10.1}{}
\begin{solution}{{10.1}}
  
\end{solution}
\protect \hypertarget {soln:11.1}{}
\begin{solution}{{11.1}}
  
\end{solution}
\protect \hypertarget {soln:11.2}{}
\begin{solution}{{11.2}}
  Например, $CNOT = \ketbra{00}{00} + \ketbra{01}{01} + \ketbra{10}{11} + \ketbra{11}{10}$.
\end{solution}



\section{Источники мудрости}

\todo[inline]{передалать потом в bib-файл}

\begin{enumerate}
\item Кратко про геометрию Фано, \url{https://www.youtube.com/watch?v=CRqso5-uLfI}
\item How to build hyperbolic soccer ball, \url{http://theiff.org/images/IFF_HypSoccerBall.pdf}
\item Chaim Goodman-Strauss, Compass and Straightedge in the Poincaré Disk
\item Mann, DIY hyperbolic course, \url{https://math.berkeley.edu/~kpmann/DIY%20hyperbolic%20course.pdf}
\item 3blue1brown, Quaternions visualized, \url{https://www.youtube.com/watch?v=d4EgbgTm0Bg}
\item Grant Sanderson, Visualizing quaternions, \url{https://eater.net/quaternions}
\item \url{https://www.quantamagazine.org/the-octonion-math-that-could-underpin-physics-20180720/}, есть pdf-ка
с картинками умножения на кватернионов и октонионов. 
\item Hanson, Visualizing quaternions, примеры про ремень, мячик, Apollo
\item \url{https://brilliant.org/wiki/complex-numbers-in-geometry/}
\item Прасолов, Геометрия Лобачевского
\item Slerp, wiki, \url{https://en.wikipedia.org/wiki/Slerp}
\item Wiki, 3d rotation, \url{https://en.wikipedia.org/wiki/Rotation_formalisms_in_three_dimensions}
\item Fano plane, \url{https://blogs.scientificamerican.com/roots-of-unity/a-few-of-my-favorite-spaces-the-fano-plane/}
\item Lam, Search finite Fano plane of order 10, \url{https://www.maa.org/sites/default/files/pdf/upload_library/22/Ford/Lam305-318.pdf},
связка с латинскими квадратами
\item \url{http://kahrstrom.com/mathematics/documents/OnProjectivePlanes.pdf}, геометрия как точки и линии

\end{enumerate}

\printbibliography[heading=none]


\end{document}
