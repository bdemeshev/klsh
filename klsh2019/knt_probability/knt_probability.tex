\documentclass[a4paper, 12pt]{article}


\usepackage{mathrsfs}

\usepackage{amscd}
\usepackage[paper=a4paper,
top=2cm, bottom=2cm, left=1cm, right=1cm, includefoot]{geometry} % размер листа бумаги


\usepackage{tikz} % картинки в tikz
\usepackage{microtype} % свешивание пунктуации

\usepackage{floatrow} % для выравнивания рисунка и подписи
\usepackage{caption} % для пустых подписей

\usepackage{array} % для столбцов фиксированной ширины

\usepackage{indentfirst} % отступ в первом параграфе

\usepackage{sectsty} % для центрирования названий частей
\allsectionsfont{\centering}

\usepackage{amsmath, amsfonts} % куча стандартных математических плюшек

\usepackage{comment} % для комментариев

\usepackage{multicol} % текст в несколько колонок

\usepackage{lastpage} % чтобы узнать номер последней страницы

\usepackage{enumitem} % дополнительные плюшки для списков
%  например \begin{enumerate}[resume] позволяет продолжить нумерацию в новом списке

\usepackage{booktabs}

\usepackage{url} % для вставки интернет-ссылок

\usepackage{fontspec}
\usepackage{polyglossia}

\setmainlanguage{russian}
\setotherlanguages{english}

% download "Linux Libertine" fonts:
% http://www.linuxlibertine.org/index.php?id=91&L=1
\setmainfont{Linux Libertine O} % or Helvetica, Arial, Cambria
% why do we need \newfontfamily:
% http://tex.stackexchange.com/questions/91507/
\newfontfamily{\cyrillicfonttt}{Linux Libertine O}

\AddEnumerateCounter{\asbuk}{\russian@alph}{щ} % для списков с русскими буквами
\setlist[enumerate, 2]{label=\asbuk*),ref=\asbuk*}

\DeclareMathOperator{\Var}{Var}
\DeclareMathOperator{\E}{\mathbb{E}}

\let\P\relax
\DeclareMathOperator{\P}{\mathbb{P}}
\def\cN{\mathcal{N}}

\usepackage{fancyhdr} % весёлые колонтитулы
\pagestyle{fancy}
\lhead{Теория вероятностей}
\chead{КНТ}
\rhead{КЛШ-2019}
\lfoot{}
\cfoot{}
\rfoot{\thepage/\pageref{LastPage}}
\renewcommand{\headrulewidth}{0.4pt}
\renewcommand{\footrulewidth}{0.4pt}


\begin{document}

% источник задач:
% * группа vk Синяя птица (вероятность в школе)
% Шень
% Канеман

% Илья Панов, Борис Дураков, Дима Постников — на цифроглот


\begin{enumerate}
\item (1б) Сережа Л. и Миша С. одновременно подбрасывают по одному игральному кубику. У кого выпало больше, тот забирает у другого рыбные котлеты. 
Если выпало поровну, то объявляется ничья. Какова вероятность того, что будет ничья? 

\item (1б) В игре бросают кубик; выигрышем считается выпадение пятёрки или шестёрки. Сколько примерно выигрышей будет в длинной серии из 666 игр?


\item (1б) Маша идёт на день рождения, где будут десять мальчиков и десять девочек помимо Маши. Они садятся за круглый стол в случайном порядке. 
Какова вероятность, что справа от Маши будет сидеть мальчик?

\item (1б) Маша идёт на день рождения, где будут пять мальчиков и пять девочек помимо Маши. Они садятся за круглый стол в случайном порядке. 
Какова вероятность, что справа от Маши будет сидеть мальчик?

\item (1б) Одно из десяти чисел увеличили на 1. Как изменилось от этого среднее арифметическое этих чисел?

\item (1б) В автобусе собралась футбольная команда из 11 человек и волейбольная команда из 6 человек. Средний возраст футболистов — 37 лет, 
средний возраст волейболистов — 20 лет. Каков средний возраст пассажиров автобуса?

\item (1б) В автобусе собралась футбольная команда из 11 человек и волейбольная команда из 6 человек. Средний возраст футболистов — 26 лет, 
средний возраст волейболистов — 43 года. Каков средний возраст пассажиров автобуса?

\item (1б) Летнешкольник подбрасывает симметричную монету 3 раза. С какой вероятностью выпадет ровно 2 орла?

\item (1б) Летнешкольник подбрасывает симметричную монету 3 раза. С какой вероятностью выпадет не более 2 орлов?

\item (1б) Летнешкольник подбрасывает симметричную монету 3 раза. С какой вероятностью выпадет ровно 1 решка?

\item (1б) Ты бросил две шестигранных игральных кости. Какова вероятность, что у тебя выпал дубль?

\item (1б) На ёлку пришли: Вася с братом Мишей, Аня с сестрой Настей, Петя с братом Костей и Катя с братом Васей. Дед Мороз устроил лотерею. 
Какова вероятность того, что главный приз достался мальчику?

\item (1б) Какова вероятность того, что при броске кубика выпадет простое число?

\item (1б) На клавиатуре телефона 10 цифр, от 0 до 9. Какова вероятность того, что случайно нажатая цифра будет меньше 4?

\item (1б) На клавиатуре телефона 10 цифр, от 0 до 9. Какова вероятность того, что случайно нажатая цифра будет четная?

\item (1б) Какова вероятность того, что случайно выбранное натуральное число от 41 до 56 делится на 2?

\item (1б) Механические часы с двенадцатичасовым циферблатом в какой-то момент сломались и перестали ходить. Найди вероятность того, 
что часовая стрелка застыла, достигнув отметки 6, но не дойдя до отметки 9 часов.

\item (1б) Механические часы с двенадцатичасовым циферблатом в какой-то момент сломались и перестали ходить. Найди вероятность того, 
что часовая стрелка застыла, достигнув отметки 3, но не дойдя до отметки 8 часов.

\item (1б) Какова вероятность того, что случайная буква из русского алфавита, включая ё, будет гласной?

\item (1б) В третьем подъезде дома квартиры с 41 по 60 включительно. Гость набрал на домофоне номер одной из этих квартир. 
Найдите вероятность того, что он позвонил в квартиру с четным номером. 

\item (1б) На клавиатуре телефона 10 цифр, от 0 до 9. Какова вероятность того, что случайно нажатая цифра будет больше 2, но меньше 7? 

\item (1б) Из слова «МАТЕМАТИКА» случайным образом выбирается одна буква. Найдите вероятность того, что эта буква окажется согласной.

\item (1б) На тусовке есть только инопланетяне и жирафы, причем инопланетян в 7 раз больше, чем жираф. Найдите вероятность того, 
что случайно выбранная на этой ферме тварь окажется жирафой.

\item (1б) В городе N частота события «гарантийный ремонт гироскутера модели А» составила 0.07. 
Сколько из 200 проданных гироскутеров модели А попало в ремонт в этом городе по гарантии?

\item (1б) Школьников рассаживают по трем аудиториям. В первых двух по 120 человек, оставшихся проводят в запасную аудиторию в другом корпусе. 
При подсчете выяснилось, что всего было 250 участников. Найдите вероятность того, что случайно выбранный участник писал олимпиаду в запасной аудитории.

\item (1б) На день рождения Васи пришли 10 инопланетян и 5 жираф. Они все сели вместе за круглый стол. 
Какова вероятность, что справа от именинника будет сидеть жирафа? 


% \item (1б) Петя и Вася одновременно подбрасывают по одному игральному кубику. У кого выпало больше, тот и выиграл. Если выпало поровну, то объявляется ничья. Какова вероятность того, что будет ничья? $1/6$
% \item (1б) На ёлку пришли: Вася с братом Мишей, Аня с сестрой Настей, Петя с братом Костей и Катя с братом Васей. Дед Мороз устроил лотерею. Какова вероятность того, что главный приз достался мальчику? $5/8$
% \item (1б) Игральный кубик подбрасывают два раза. Сколько в среднем очков выпадает в сумме на двух кубиках? $7$
% \item (1б) В игре бросают кубик; выигрышем считается выпадение пятёрки или шестёрки. Сколько (примерно) выигрышей будет в длинной серии из 666 игр? $666/3=222$
% \item (1б) В мешке лежат бумажки с цифрами 1, 2, \ldots ,9. Из мешка наудачу вытаскивают одну из бумажек наугад. Какова вероятность того, что будет вытащено чётное число? $4/9$
% \item (1б) В мешке лежат бумажки с цифрами 1, 2, \ldots, 9. Из мешка наудачу вытаскивают одну из бумажек наугад. Какова вероятность того, что будет вытащено число, делящееся на 3? $3/9=1/3$
% \item (1б) В мешке лежат бумажки с цифрами 1, 2, \ldots, 9. Из мешка наудачу вытаскивают одну из бумажек наугад. Какова вероятность того, что будет вытащено число, не делящееся на 3? $6/9=2/3$
% \item (1б) В мешке лежат бумажки с цифрами 1, 2, \ldots, 9. Из мешка наудачу вытаскивают одну из бумажек наугад. Какова вероятность того, что будет вытащено число, не делящееся ни на 2, ни на 3? $3/9=1/3$
% \item (1б) Маша идёт на день рождения, где будут десять мальчиков и десять девочек (кроме Маши). Они садятся за круглый стол в случайном порядке. Какова вероятность, что справа от Маши будет сидеть мальчик? $10/20=1/2$
% \item (1б) Маша идёт на день рождения, где будут пять мальчиков и пять девочек (кроме Маши). Они садятся за круглый стол в случайном порядке. Какова вероятность, что справа от Маши будет сидеть мальчик? $5/10=1/2$
% \item (1б) Обычную рублевую монетку подбрасывают четыре раза. Первые три раза она выпала орлом. Какова вероятность того, что она выпадет орлом в четвертый раз? $1/2$
% \item (1б) У Пети связка из 10 ключей. Один из них подходит к замку. Петя не знает, какой ключ подходит к замку и перебирает их по очереди. У какого ключа выше шансы подойти? У всех одинаковы
% \item (1б) Одно из десяти чисел увеличили на 1. Как изменилось от этого среднее арифметическое этих чисел? На $0.1$
% \item (1б) В автобусе собралась футбольная команда из 11 человек и волейбольная команда из 6 человек. Средний возраст футболистов — 37 лет, средний возраст волейболистов — 20 год. Каков средний возраст пассажиров автобуса? $(37 \cdot 11 + 6 \cdot 20)/17=31$
\end{enumerate}

\newpage
\begin{enumerate}
\item (2б) В чемодане у школьника 4 красных и 6 зеленых доширака. В абсолютной темноте ночи он решил достать доширак сначала себе, а потом другу. 
Какова вероятность, что у друга окажется красный доширак? 
\item (2б) Найди разницу вероятностей того, что случайная буква будет гласной для английского алфавита и русского алфавита с буквой ё. $35/429$.
\item (2б) Помещение освещается фонарем с тремя лампами. Вероятность перегорания одной лампы в течение года равна 0,5. Найди вероятность того, 
что в течение года хотя бы одна лампа не перегорит. 
\item (2б) Вероятность того, что батарейка бракованная, равна 0,06. Покупатель в магазине выбирает случайную упаковку, в которой две таких батарейки. 
Найдите вероятность того, что обе батарейки окажутся исправными. 
\item (2б) Комната освещвется двумя идентичными лампами. Вероятность поломки отдельной лампы за год равна 0.8. Какова вероятность того, 
что через год обе лампы будут работать?
\item (2б) Ты бросил три шестигранных игральных кости. Какова вероятность, что у тебя в сумме выпало 5 очков? 
\item (2б) Вожатому Саше нужно спаять трубу для самовара. Для этого ему нужны 3 паяльника. У Саши есть 13 паяльников, из которых 6 не рабочих.  
Саша наугад вытаскивает 3 паяльника наугад. С какой вероятностью они будут рабочими? 
\item (2б) Сотрудник биохима Яша хочет покормить тараканов. У него есть 10 кусков хлеба, из которых 5 отравленные. Чтобы накормить тараканов, 
Яше нужно 3 куска хлеба. С какой вероятностью Яша не отравит тараканов?  

\item (2б) Фамилию КАЗАКОВА нарезали на буквы и сложили в мешок. Мальчик Ваня наугад вытаскивает 2 буквы. 
Какова вероятность, что он не вытащит букву К? 


\item (2б) В классе 26 человек, среди них два близнеца — Андрей и Сергей. Класс случайным образом делят на две группы по 13 человек в каждой. 
Найдите вероятность того, что Андрей и Сергей окажутся в одной группе. 
\item (2б) В классе 21 учащийся, среди них два друга — Вадим и Олег. Класс случайным образом разбивают на 3 равные группы. Найдите вероятность того, 
что Вадим и Олег окажутся в одной группе. 
\item (2б) В классе учится 21 человек. Среди них две подруги: Аня и Нина. Класс случайным образом делят на 7 групп, по 3 человека в каждой. 
Найти вероятность того. что Аня и Нина окажутся в одной группе. 
\item (2б) На день рождения Васи пришли 10 инопланетян и 5 жираф. Они все сели вместе за круглый стол. 
Какова вероятность, что справа и слева от именинника будут сидеть инопланетяне? 
\item (2б) Десять аргонавтов разного роста в случайном порядке идут по узкой тропинке друг за другом. 
Каждый аргонавт видит вперёд не далее спины более высокого аргонавта. 
Внезапно впереди аргонавтов показалась Медуза Горгона, и все, кто её видел, обратились в камни. 
Какова вероятность того, что последний аргонавт обратился в камень? 
\item (2б) Какова вероятность того, что случайно выбранное натуральное число от 10 до 19 делится на три? 
\item (2б) Какова вероятность, что случайно выбранное двузначное число делится на 5? 
\item (2б) Из множества натуральных чисел от 28 до 47 наудачу выбирают одно число. Какова вероятность того, что оно делится на 3? 
\item (2б) В случайном эксперименте бросают две игральные кости. Найдите вероятность того, что разница большего и меньшего результатов равна 1 или 2. 
\item (2б) В случайном эксперименте бросают 2 игральные кости. Найдите вероятность того, что произведение выпавших очков делится на 5, 
но не делится на 30. 
\item (2б) Андрей отправляет СМС другу. Связь не очень устойчивая, поэтому каждая попытка отправить СМС имеет вероятность успеха 0,8. 
Найдите вероятность того, что СМС будет отправлена с третьей попытки. 
\item (2б) Андрей отправляет СМС другу. Связь не очень устойчивая, поэтому каждая попытка отправить СМС имеет вероятность успеха 0,8. 
Найдите вероятность того, что СМС будет отправлена не раньше, чем с третьей попытки. 
\item (2б) Монету бросают 5 раз. Во сколько раз событие «орел выпадет ровно два раза» более вероятно, что событие «орел выпадет ровно один раз». 

\item (2б) Монету бросают 5 раз. Во сколько раз событие «орел выпадет ровно три раза» более вероятно, что событие «орел выпадет ровно четыре раза». 
\item (2б) Ты бросил три шестигранных игральных кости. Какова вероятность, что у тебя в сумме выпало 13 очков? 
\item (2б) У Паши 4 ореха. Из них два, не ясно какие, пустые. Паша разбивает первый орех, и затем, не глядя на результат, разбивает второй. Второй разбитый орех — пустой. Вероятность того, что первый разбитый орех был пустым?

\item (2б) Джон Сильвер подбрасывает игральную кость до тех пор, пока не выпадет 6 три раза, не обязательно подряд. Сколько в среднем бросков ему потребуется?
\item (2б) Джон Сильвер подбрасывает игральную кость до тех пор, пока не выпадет 6 два раза, не обязательно подряд. Сколько в среднем бросков ему потребуется?

\item (2б) В пакетике 6 оранжевых и $n$ жёлтых конфет. Аня достаёт одну наугад и съедает. Затем достаёт ещё одну и снова съедает. Вероятность того, что Аня съела две оранжевых равна $1/3$. Сколько конфет было в пакетике?


%   \item (2б) Студент Мгамба Унь сдаёт экзамен по английском языку в России. Ему дали 5 карточек с английскими словами и 5 карточек с их переводами на русский. Мгамба не знает ни английского, ни русского и сопоставляет карточки наугад. Какова вероятность того, что он угадает все переводы? $1/5!=1/120$.
%   \item (2б) На ёлку пришли: Вася с братом Мишей, Аня с сестрой Настей, Петя с братом Костей и Катя с братом Васей. Дед Мороз устроил лотерею и главный приз достался мальчику. Какова вероятность того, что тот пришёл с братом? $4/5$.
%   \item (2б) В классе 20 человек. Один заболел, ещё один — опоздал. Какова вероятность того, что их фамилии идут подряд в классном журнале? $19/C_{20}^2=0.1$
%   \item (2б) Вероятность рождения двойняшек в Урляндии равна $1/10$. А тройняшки и больше чем тройняшки в Урляндии не рождаются. Какова вероятность того, что первый встречный урляндец один из двойняшек? $2/11$.
%   \item (2б) Петя и Вася играют в дурака до 6 побед. Сейчас счёт 5:4 в пользу Пети. На кону 36 рублей, и тут внезапно начался ураган. Петя и Вася вынуждены прервать игры. Как им поделить деньги по справедливости?

%   В пропорции $3:1$, то есть $27:9$.

%   \item (2б) Петя и Вася играют в дурака до 4 побед. Сейчас счёт 3:2 в пользу Пети. На кону 20 рублей, и тут внезапно начался ураган. Петя и Вася вынуждены прервать игры. Как им поделить деньги по справедливости?

%   В пропорции $3:1$, то есть $15:5$.
%   \item (2б) Вася подбрасывает два кубика. Какая сумма очков на кубиках наиболее вероятна? $7$.
%   \item (2б) Маша переставляет буквы в слове МАША в случайном порядке. Какова вероятность того, что снова получится слово МАША? $2/4!=1/12$
%   \item (2б) Маша переставляет буквы в слове МАМА в случайном порядке. Какова вероятность того, что снова получится слово МАМА? $4/4!=1/6$
%   \item (2б) Среди учеников школы 15\% знают французский язык и 20\% знают немецкий язык. Доля учеников, знающих оба этих языка, составляет 5\%. Какова доля учеников, знающих французский язык, среди учеников, знающих немецкий язык? $5/20=1/4$
%   \item (2б) Среди учеников школы 15\% знают французский язык и 20\% знают немецкий язык. Доля учеников, знающих оба этих языка, составляет 5\%. Какова доля учеников, знающих французский язык, среди учеников, не знающих немецкий язык? $10/80=1/8$
%   \item (2б) Среди учеников школы 15\% знают французский язык и 20\% знают немецкий язык. Доля учеников, знающих оба этих языка, составляет 5\%. Какова доля учеников, знающих немецкий язык, среди учеников, знающих французский язык? $5/15=1/3$
%   \item (2б) Среди учеников школы 15\% знают французский язык и 20\% знают немецкий язык. Доля учеников, знающих оба этих языка, составляет 5\%. Какова доля учеников, знающих немецкий язык, среди учеников, не знающих французский язык? $15/85=3/17$
%   \item (2б) Среди шахматистов каждый седьмой — музыкант, а среди музыкантов каждый девятый — шахматист. Кого больше, шахматистов или музыкантов и во сколько раз? Музыкантов в $9/7$
\end{enumerate}
    




\newpage
\begin{enumerate}
\item (3б) В классе не более 40 человек, среди них есть те, кого зовут Коля. Вероятность того, что случайно выбранный ученик выше всех Коль, равна $2/5$. Вероятность того, что случайно выбранный ученик класса ниже всех Коль, равна $3/7$. Сколько Коль может быть в классе?
\item (3б) Джон Сильвер и Билли Бонс играют в кости. У них есть одна игральная кость и они по очереди её бросают неограниченное количество раз. Кто первый выбросит шестёрку, тот и выиграл. Начинает Джон Сильвер. Какова вероятность того, что победит Билли Бонс?
\item (3б) Ровно половина жителей острова Невезения зайцы, а остальные — кролики. Зайцы врут в половине своих фраз, а кролики — в двух третях. Вышел однажды житель острова, сел на пенёк и сказал: «Я не заяц». Какова условная вероятность того, что он действительно не заяц?
\item (3б) Ровно половина жителей острова Невезения зайцы, а остальные — кролики. Зайцы врут в половине своих фраз, а кролики — в двух третях. Вышел однажды житель острова, сел на пенёк и сказал: «Я заяц». Какова условная вероятность того, что он действительно заяц?
\item (3б) Ровно половина жителей острова Невезения зайцы, а остальные — кролики. Зайцы врут в половине своих фраз, а кролики — в двух третях. Вышел однажды житель острова, сел на пенёк и сказал: «Я кролик». Какова условная вероятность того, что он действительно кролик?
\item (3б) Ровно половина жителей острова Невезения зайцы, а остальные — кролики. Зайцы врут в половине своих фраз, а кролики — в двух третях. Вышел однажды житель острова, сел на пенёк и сказал: «Я не кролик». Какова условная вероятность того, что он действительно не кролик?
%\item (3б) Ровно половина жителей острова Невезения зайцы, а остальные — кролики. Зайцы врут в половине своих фраз, а кролики — в двух третях. Вышел однажды житель острова, сел на пенёк и сказал: «Я не заяц». А потом помолчал и добавил: «Я не кролик». Какова условная вероятность того, что он всё же заяц?
\item (3б) Редкой болезнью болеет 1\% населения. Существующий тест ошибается в 10\% случаев. У первого встречного берут тест. Судя по тесту, человек болен. Какова вероятность того, что он действительно болен?
\item (3б) Вероятность того, что за полчаса по шоссе проедет хотя бы одна машина равна $0.95$. 
Какова вероятность того, что хотя бы одна машина проедет за 10 минут? 
\item (3б) Вероятность того, что летнешкольница Маша не поймает ни одной многоножки, обойдя три гектара леса равна $0.8$. 
Какова вероятность того, что летнешкольница Маша поймает хотя бы одну многоножку обойдя два гектара леса? 


% ответ угадывается по неправильной логике — это неприятно :)
%\item (3б) Илон Маск подбросил правильную монетку 3 раза. Записал результаты бросков на 3 бумажках и положил бумажки в шляпку. 
%Затем Билл Гейтс достал две бумажки из шляпы наугад. Какова вероятность того, что у Илона Маска все три раза выпал орёл,
%если Билл Гейтс достал две бумажки с орлами? 
%\item (3б) Илон Маск подбросил правильную монетку 3 раза. Записал результаты бросков на 3 бумажках и положил бумажки в шляпку. 
%Затем Билл Гейтс достал одну бумажку из шляпы наугад. Какова вероятность того, что у Илона Маска все три раза выпал орёл,
%если Билл Гейтс достал бумажку с орлом? 
\
item (3б) В убийстве равновероятно виноват либо Джон, либо Билл. На месте убийства найдена кровь убийцы, совпадающая с группой крови Джона. 
Такой группой крови обладает 10\% населения. Группа крови Билла неизвестна. Какова вероятность, что у Билла окажется та же группа крови?
\item (3б) В убийстве равновероятно виноват либо Джон, либо Билл. На месте убийства найдена кровь убийцы, совпадающая с группой крови Джона. 
Такой группой крови обладает 10\% населения. Группа крови Билла неизвестна. Какова вероятность, что Билл — убийца? 
\item (3б) В классе было два школьника и сколько-то школьниц. Равновероятно в класс заходит новый школьник или новая школьница. 
Затем все собравшиеся выбрали одного человека жеребьевкой и выгнали. Какова вероятность того, что вошел школьник, если выгнали школьника?
\item (3б) Судья подкидывает монетку неограниченное количество раз. Если последовательность орёл-орёл-решка выпадает раньше
последовательности решка-орёл-орёл, то выигрывает Оля. Иначе выигрывает Рома. Какова вероятность того, что Рома выиграет?
\end{enumerate}

\newpage
Ответы
\begin{enumerate}
\item (1б) Сережа Л. и Миша С. одновременно подбрасывают по одному игральному кубику. У кого выпало больше, тот забирает у другого рыбные котлеты. 
Если выпало поровну, то объявляется ничья. Какова вероятность того, что будет ничья? $1/6$

\item (1б) В игре бросают кубик; выигрышем считается выпадение пятёрки или шестёрки. Сколько примерно выигрышей будет в длинной серии из 666 игр?
$222$

\item (1б) Маша идёт на день рождения, где будут десять мальчиков и десять девочек помимо Маши. Они садятся за круглый стол в случайном порядке. 
Какова вероятность, что справа от Маши будет сидеть мальчик?
$1/2$

\item (1б) Маша идёт на день рождения, где будут пять мальчиков и пять девочек помимо Маши. Они садятся за круглый стол в случайном порядке. 
Какова вероятность, что справа от Маши будет сидеть мальчик?
$1/2$

\item (1б) Одно из десяти чисел увеличили на 1. Как изменилось от этого среднее арифметическое этих чисел?
выросло на $0.1$

\item (1б) В автобусе собралась футбольная команда из 11 человек и волейбольная команда из 6 человек. Средний возраст футболистов — 37 лет, 
средний возраст волейболистов — 20 лет. Каков средний возраст пассажиров автобуса?
$31$

\item (1б) В автобусе собралась футбольная команда из 11 человек и волейбольная команда из 6 человек. Средний возраст футболистов — 26 лет, 
средний возраст волейболистов — 43 года. Каков средний возраст пассажиров автобуса?
$32$

\item (1б) Летнешкольник подбрасывает симметричную монету 3 раза. С какой вероятностью выпадет ровно 2 орла?
$3/8 = 0.375$

\item (1б) Летнешкольник подбрасывает симметричную монету 3 раза. С какой вероятностью выпадет не более 2 орлов?
$7/8=0.875$

\item (1б) Летнешкольник подбрасывает симметричную монету 3 раза. С какой вероятностью выпадет ровно 1 решка?
$3/8=0.375$

\item (1б) Ты бросил две шестигранных игральных кости. Какова вероятность, что у тебя выпал дубль?
$1/6$

\item (1б) На ёлку пришли: Вася с братом Мишей, Аня с сестрой Настей, Петя с братом Костей и Катя с братом Васей. Дед Мороз устроил лотерею. 
Какова вероятность того, что главный приз достался мальчику?
$5/8$

\item (1б) Какова вероятность того, что при броске кубика выпадет простое число?
$1/2=3/6$

\item (1б) На клавиатуре телефона 10 цифр, от 0 до 9. Какова вероятность того, что случайно нажатая цифра будет меньше 4?
$4/10$

\item (1б) На клавиатуре телефона 10 цифр, от 0 до 9. Какова вероятность того, что случайно нажатая цифра будет четная?
$5/10$

\item (1б) Какова вероятность того, что случайно выбранное натуральное число от 41 до 56 делится на 2?
$1/2$

\item (1б) Механические часы с двенадцатичасовым циферблатом в какой-то момент сломались и перестали ходить. Найди вероятность того, 
что часовая стрелка застыла, достигнув отметки 6, но не дойдя до отметки 9 часов.
$1/4$

\item (1б) Механические часы с двенадцатичасовым циферблатом в какой-то момент сломались и перестали ходить. Найди вероятность того, 
что часовая стрелка застыла, достигнув отметки 3, но не дойдя до отметки 8 часов.
$5/12$

\item (1б) Какова вероятность того, что случайная буква из русского алфавита, включая ё, будет гласной?
$10/33$

\item (1б) В третьем подъезде дома квартиры с 41 по 60 включительно. Гость набрал на домофоне номер одной из этих квартир. 
Найдите вероятность того, что он позвонил в квартиру с четным номером. 
$1/2$

\item (1б) На клавиатуре телефона 10 цифр, от 0 до 9. Какова вероятность того, что случайно нажатая цифра будет больше 2, но меньше 7? 
$0.4$

\item (1б) Из слова «МАТЕМАТИКА» случайным образом выбирается одна буква. Найдите вероятность того, что эта буква окажется согласной.
$1/2$

\item (1б) На тусовке есть только инопланетяне и жирафы, причем инопланетян в 7 раз больше, чем жираф. Найдите вероятность того, 
что случайно выбранная на этой ферме тварь окажется жирафой.
$1/8=0.125$

\item (1б) В городе N частота события «гарантийный ремонт гироскутера модели А» составила 0.07. 
Сколько из 200 проданных гироскутеров модели А попало в ремонт в этом городе по гарантии?
$14$

\item (1б) Школьников рассаживают по трем аудиториям. В первых двух по 120 человек, оставшихся проводят в запасную аудиторию в другом корпусе. 
При подсчете выяснилось, что всего было 250 участников. Найдите вероятность того, что случайно выбранный участник писал олимпиаду в запасной аудитории.
$1/25$


\item (1б) На день рождения Васи пришли 10 инопланетян и 5 жираф. Они все сели вместе за круглый стол. 
Какова вероятность, что справа от именинника будет сидеть жирафа? $5/15=1/3$.

        

% \item (1б) Петя и Вася одновременно подбрасывают по одному игральному кубику. У кого выпало больше, тот и выиграл. Если выпало поровну, то объявляется ничья. Какова вероятность того, что будет ничья? $1/6$
% \item (1б) На ёлку пришли: Вася с братом Мишей, Аня с сестрой Настей, Петя с братом Костей и Катя с братом Васей. Дед Мороз устроил лотерею. Какова вероятность того, что главный приз достался мальчику? $5/8$
% \item (1б) Игральный кубик подбрасывают два раза. Сколько в среднем очков выпадает в сумме на двух кубиках? $7$
% \item (1б) В игре бросают кубик; выигрышем считается выпадение пятёрки или шестёрки. Сколько (примерно) выигрышей будет в длинной серии из 666 игр? $666/3=222$
% \item (1б) В мешке лежат бумажки с цифрами 1, 2, \ldots ,9. Из мешка наудачу вытаскивают одну из бумажек наугад. Какова вероятность того, что будет вытащено чётное число? $4/9$
% \item (1б) В мешке лежат бумажки с цифрами 1, 2, \ldots, 9. Из мешка наудачу вытаскивают одну из бумажек наугад. Какова вероятность того, что будет вытащено число, делящееся на 3? $3/9=1/3$
% \item (1б) В мешке лежат бумажки с цифрами 1, 2, \ldots, 9. Из мешка наудачу вытаскивают одну из бумажек наугад. Какова вероятность того, что будет вытащено число, не делящееся на 3? $6/9=2/3$
% \item (1б) В мешке лежат бумажки с цифрами 1, 2, \ldots, 9. Из мешка наудачу вытаскивают одну из бумажек наугад. Какова вероятность того, что будет вытащено число, не делящееся ни на 2, ни на 3? $3/9=1/3$
% \item (1б) Маша идёт на день рождения, где будут десять мальчиков и десять девочек (кроме Маши). Они садятся за круглый стол в случайном порядке. Какова вероятность, что справа от Маши будет сидеть мальчик? $10/20=1/2$
% \item (1б) Маша идёт на день рождения, где будут пять мальчиков и пять девочек (кроме Маши). Они садятся за круглый стол в случайном порядке. Какова вероятность, что справа от Маши будет сидеть мальчик? $5/10=1/2$
% \item (1б) Обычную рублевую монетку подбрасывают четыре раза. Первые три раза она выпала орлом. Какова вероятность того, что она выпадет орлом в четвертый раз? $1/2$
% \item (1б) У Пети связка из 10 ключей. Один из них подходит к замку. Петя не знает, какой ключ подходит к замку и перебирает их по очереди. У какого ключа выше шансы подойти? У всех одинаковы
% \item (1б) Одно из десяти чисел увеличили на 1. Как изменилось от этого среднее арифметическое этих чисел? На $0.1$
% \item (1б) В автобусе собралась футбольная команда из 11 человек и волейбольная команда из 6 человек. Средний возраст футболистов — 37 лет, средний возраст волейболистов — 20 год. Каков средний возраст пассажиров автобуса? $(37 \cdot 11 + 6 \cdot 20)/17=31$
\end{enumerate}

\newpage
\begin{enumerate}
\item (2б) В чемодане у школьника 4 красных и 6 зеленых доширака. В абсолютной темноте ночи он решил достать доширак сначала себе, а потом другу. 
Какова вероятность, что у друга окажется красный доширак? $4/10$
\item (2б) Найди разницу вероятностей того, что случайная буква будет гласной для английского алфавита и русского алфавита с буквой ё. $35/429$.
\item (2б) Помещение освещается фонарем с тремя лампами. Вероятность перегорания одной лампы в течение года равна 0,5. Найди вероятность того, 
что в течение года хотя бы одна лампа не перегорит. $1-0.5^3 = 0.875$
\item (2б) Вероятность того, что батарейка бракованная, равна 0,06. Покупатель в магазине выбирает случайную упаковку, в которой две таких батарейки. 
Найдите вероятность того, что обе батарейки окажутся исправными. $(1-0.06)^2 = 0.94^2 = 0.8836$
\item (2б) Комната освещвется двумя идентичными лампами. Вероятность поломки отдельной лампы за год равна 0.8. Какова вероятность того, 
что через год обе лампы будут работать? $0.2^2=0.04$ 
\item (2б) Ты бросил три шестигранных игральных кости. Какова вероятность, что у тебя в сумме выпало 5 очков? $6/216=1/36$
\item (2б) Вожатому Саше нужно спаять трубу для самовара. Для этого ему нужны 3 паяльника. У Саши есть 13 паяльников, из которых 6 не рабочих.  
Саша наугад вытаскивает 3 паяльника наугад. С какой вероятностью они будут рабочими? $C_{7}^3/C_{13}^3 = 35/286$
\item (2б) Сотрудник биохима Яша хочет покормить тараканов. У него есть 10 кусков хлеба, из которых 5 отравленные. Чтобы накормить тараканов, 
Яше нужно 3 куска хлеба. С какой вероятностью Яша не отравит тараканов?  $C_5^3/C_{10}^3 = 1/12$

\item (2б) Фамилию КАЗАКОВА нарезали на буквы и сложили в мешок. Мальчик Ваня наугад вытаскивает 2 буквы. 
Какова вероятность, что он не вытащит букву К? $6\cdot 5/(7\cdot 8) = 15/28$.


\item (2б) В классе 26 человек, среди них два близнеца — Андрей и Сергей. Класс случайным образом делят на две группы по 13 человек в каждой. 
Найдите вероятность того, что Андрей и Сергей окажутся в одной группе. $12/25 = 0.48$
\item (2б) В классе 21 учащийся, среди них два друга — Вадим и Олег. Класс случайным образом разбивают на 3 равные группы. Найдите вероятность того, 
что Вадим и Олег окажутся в одной группе. $6/20=0.3$
\item (2б) В классе учится 21 человек. Среди них две подруги: Аня и Нина. Класс случайным образом делят на 7 групп, по 3 человека в каждой. 
Найти вероятность того. что Аня и Нина окажутся в одной группе. $2/20=0.1$
\item (2б) На день рождения Васи пришли 10 инопланетян и 5 жираф. Они все сели вместе за круглый стол. 
Какова вероятность, что справа и слева от именинника будут сидеть инопланетяне? $10 \cdot 9 / 15\cdot 14 = 3/7$
\item (2б) Десять аргонавтов разного роста в случайном порядке идут по узкой тропинке друг за другом. 
Каждый аргонавт видит вперёд не далее спины более высокого аргонавта. 
Внезапно впереди аргонавтов показалась Медуза Горгона, и все, кто её видел, обратились в камни. 
Какова вероятность того, что последний аргонавт обратился в камень? $1/10$
\item (2б) Какова вероятность того, что случайно выбранное натуральное число от 10 до 19 делится на три? $3/10$
\item (2б) Какова вероятность, что случайно выбранное двузначное число делится на 5? $18/90$
\item (2б) Из множества натуральных чисел от 28 до 47 наудачу выбирают одно число. Какова вероятность того, что оно делится на 3? $6/20=0.3$
\item (2б) В случайном эксперименте бросают две игральные кости. Найдите вероятность того, что разница большего и меньшего результатов равна 1 или 2. $18/36=1/2$
\item (2б) В случайном эксперименте бросают 2 игральные кости. Найдите вероятность того, что произведение выпавших очков делится на 5, 
но не делится на 30. $0.25$
\item (2б) Андрей отправляет СМС другу. Связь не очень устойчивая, поэтому каждая попытка отправить СМС имеет вероятность успеха 0,8. 
Найдите вероятность того, что СМС будет отправлена с третьей попытки. $0.2 \cdot 0.2 \cdot 0.8 = 0.032$
\item (2б) Андрей отправляет СМС другу. Связь не очень устойчивая, поэтому каждая попытка отправить СМС имеет вероятность успеха 0,8. 
Найдите вероятность того, что СМС будет отправлена не раньше, чем с третьей попытки. $0.2^2 = 0.04$
\item (2б) Монету бросают 5 раз. Во сколько раз событие «орел выпадет ровно два раза» более вероятно, что событие «орел выпадет ровно один раз». $2$

\item (2б) Монету бросают 5 раз. Во сколько раз событие «орел выпадет ровно три раза» более вероятно, что событие «орел выпадет ровно четыре раза». $2$
\item (2б) Ты бросил три шестигранных игральных кости. Какова вероятность, что у тебя в сумме выпало 13 очков? $21/216=7/36$
\item (2б) У Паши 4 ореха. Из них два, не ясно какие, пустые. Паша разбивает первый орех, и затем, не глядя на результат, разбивает второй. Второй разбитый орех — пустой. Вероятность того, что первый разбитый орех был пустым? $1/3$.

\item (2б) Джон Сильвер подбрасывает игральную кость до тех пор, пока не выпадет 6 три раза, не обязательно подряд. Сколько в среднем бросков ему потребуется? $6+6+6=18$ бросков.
\item (2б) Джон Сильвер подбрасывает игральную кость до тех пор, пока не выпадет 6 два раза, не обязательно подряд. Сколько в среднем бросков ему потребуется? $6+6=12$.

\item (2б) В пакетике 6 оранжевых и $n$ жёлтых конфет. Аня достаёт одну наугад и съедает. Затем достаёт ещё одну и снова съедает. Вероятность того, что Аня съела две оранжевых равна $1/3$. Сколько конфет было в пакетике? $\frac{6}{6+n}\frac{5}{5+n}=\frac{1}{3}$, $6+n=10$.


    %   \item (2б) Студент Мгамба Унь сдаёт экзамен по английском языку в России. Ему дали 5 карточек с английскими словами и 5 карточек с их переводами на русский. Мгамба не знает ни английского, ни русского и сопоставляет карточки наугад. Какова вероятность того, что он угадает все переводы? $1/5!=1/120$.
%   \item (2б) На ёлку пришли: Вася с братом Мишей, Аня с сестрой Настей, Петя с братом Костей и Катя с братом Васей. Дед Мороз устроил лотерею и главный приз достался мальчику. Какова вероятность того, что тот пришёл с братом? $4/5$.
%   \item (2б) В классе 20 человек. Один заболел, ещё один — опоздал. Какова вероятность того, что их фамилии идут подряд в классном журнале? $19/C_{20}^2=0.1$
%   \item (2б) Вероятность рождения двойняшек в Урляндии равна $1/10$. А тройняшки и больше чем тройняшки в Урляндии не рождаются. Какова вероятность того, что первый встречный урляндец один из двойняшек? $2/11$.
%   \item (2б) Петя и Вася играют в дурака до 6 побед. Сейчас счёт 5:4 в пользу Пети. На кону 36 рублей, и тут внезапно начался ураган. Петя и Вася вынуждены прервать игры. Как им поделить деньги по справедливости?

%   В пропорции $3:1$, то есть $27:9$.

%   \item (2б) Петя и Вася играют в дурака до 4 побед. Сейчас счёт 3:2 в пользу Пети. На кону 20 рублей, и тут внезапно начался ураган. Петя и Вася вынуждены прервать игры. Как им поделить деньги по справедливости?

%   В пропорции $3:1$, то есть $15:5$.
%   \item (2б) Вася подбрасывает два кубика. Какая сумма очков на кубиках наиболее вероятна? $7$.
%   \item (2б) Маша переставляет буквы в слове МАША в случайном порядке. Какова вероятность того, что снова получится слово МАША? $2/4!=1/12$
%   \item (2б) Маша переставляет буквы в слове МАМА в случайном порядке. Какова вероятность того, что снова получится слово МАМА? $4/4!=1/6$
%   \item (2б) Среди учеников школы 15\% знают французский язык и 20\% знают немецкий язык. Доля учеников, знающих оба этих языка, составляет 5\%. Какова доля учеников, знающих французский язык, среди учеников, знающих немецкий язык? $5/20=1/4$
%   \item (2б) Среди учеников школы 15\% знают французский язык и 20\% знают немецкий язык. Доля учеников, знающих оба этих языка, составляет 5\%. Какова доля учеников, знающих французский язык, среди учеников, не знающих немецкий язык? $10/80=1/8$
%   \item (2б) Среди учеников школы 15\% знают французский язык и 20\% знают немецкий язык. Доля учеников, знающих оба этих языка, составляет 5\%. Какова доля учеников, знающих немецкий язык, среди учеников, знающих французский язык? $5/15=1/3$
%   \item (2б) Среди учеников школы 15\% знают французский язык и 20\% знают немецкий язык. Доля учеников, знающих оба этих языка, составляет 5\%. Какова доля учеников, знающих немецкий язык, среди учеников, не знающих французский язык? $15/85=3/17$
%   \item (2б) Среди шахматистов каждый седьмой — музыкант, а среди музыкантов каждый девятый — шахматист. Кого больше, шахматистов или музыкантов и во сколько раз? Музыкантов в $9/7$
\end{enumerate}

\newpage
\begin{enumerate}
\item (3б) В классе не более 40 человек, среди них есть те, кого зовут Коля. Вероятность того, что случайно выбранный ученик выше всех Коль, равна $2/5$. Вероятность того, что случайно выбранный ученик класса ниже всех Коль, равна $3/7$. Сколько Коль может быть в классе?

В классе $5\cdot 7=35$ человек. Выше всех Коль 14 человек, ниже всех Коль 15 человек. Значит Коль от 1 до 16.
\item (3б) Джон Сильвер и Билли Бонс играют в кости. У них есть одна игральная кость и они по очереди её бросают неограниченное количество раз. Кто первый выбросит шестёрку, тот и выиграл. Начинает Джон Сильвер. Какова вероятность того, что победит Билли Бонс? $p=5/11$.
\item (3б) Ровно половина жителей острова Невезения зайцы, а остальные — кролики. Зайцы врут в половине своих фраз, а кролики — в двух третях. Вышел однажды житель острова, сел на пенёк и сказал: «Я не заяц». Какова условная вероятность того, что он действительно не заяц? $10/25=0.4$
\item (3б) Ровно половина жителей острова Невезения зайцы, а остальные — кролики. Зайцы врут в половине своих фраз, а кролики — в двух третях. Вышел однажды житель острова, сел на пенёк и сказал: «Я заяц». Какова условная вероятность того, что он действительно заяц? $15/35=3/7$
\item (3б) Ровно половина жителей острова Невезения зайцы, а остальные — кролики. Зайцы врут в половине своих фраз, а кролики — в двух третях. Вышел однажды житель острова, сел на пенёк и сказал: «Я кролик». Какова условная вероятность того, что он действительно кролик? $10/25=0.4$
\item (3б) Ровно половина жителей острова Невезения зайцы, а остальные — кролики. Зайцы врут в половине своих фраз, а кролики — в двух третях. Вышел однажды житель острова, сел на пенёк и сказал: «Я не кролик». Какова условная вероятность того, что он действительно не кролик? $15/35=3/7$
%\item (3б) Ровно половина жителей острова Невезения зайцы, а остальные — кролики. Зайцы врут в половине своих фраз, а кролики — в двух третях. Вышел однажды житель острова, сел на пенёк и сказал: «Я не заяц». А потом помолчал и добавил: «Я не кролик». Какова условная вероятность того, что он всё же заяц?
\item (3б) Редкой болезнью болеет 1\% населения. Существующий тест ошибается в 10\% случаев. У первого встречного берут тест. Судя по тесту, человек болен. Какова вероятность того, что он действительно болен? $0.01\cdot 0.9/(0.01 \cdot 0.9 + 0.99 \cdot 0.1)=1/12$
\item (3б) Вероятность того, что за полчаса по шоссе проедет хотя бы одна машина равна $0.95$. 
Какова вероятность того, что хотя бы одна машина проедет за 10 минут? $1 - \sqrt[3]{0.05}$.
\item (3б) Вероятность того, что летнешкольница Маша не поймает ни одной многоножки, обойдя три гектара леса равна $0.8$. 
Какова вероятность того, что летнешкольница Маша поймает хотя бы одну многоножку обойдя два гектара леса? $1 - \sqrt[3]{0.8^2}$. 

% ответ угадывается по неправильной логике — это неприятно :)
%\item (3б) Илон Маск подбросил правильную монетку 3 раза. Записал результаты бросков на 3 бумажках и положил бумажки в шляпку. 
%Затем Билл Гейтс достал две бумажки из шляпы наугад. Какова вероятность того, что у Илона Маска все три раза выпал орёл,
%если Билл Гейтс достал две бумажки с орлами? $0.5^3 / (0.5^3 + 3 \cdot 0.5^3 /3 ) = 1/2$.
%\item (3б) Илон Маск подбросил правильную монетку 3 раза. Записал результаты бросков на 3 бумажках и положил бумажки в шляпку. 
%Затем Билл Гейтс достал одну бумажку из шляпы наугад. Какова вероятность того, что у Илона Маска все три раза выпал орёл,
%если Билл Гейтс достал бумажку с орлом? $0.5^3 / (0.5^3 + 3 \cdot 0.5^3 \cdot 2/3 + 3\cdot 0.5^3 / 3) = 1/4$.

\item (3б) В убийстве равновероятно виноват либо Джон, либо Билл. На месте убийства найдена кровь убийцы, совпадающая с группой крови Джона. 
Такой группой крови обладает 10\% населения. Группа крови Билла неизвестна. Какова вероятность, что у Билла окажется та же группа крови?
$0.1 \cdot 0.1 /(0.5 \cdot 0.1 \cdot 0.1 + 0.5 \cdot 0.1) = 10/55=2/11$.
\item (3б) В убийстве равновероятно виноват либо Джон, либо Билл. На месте убийства найдена кровь убийцы, совпадающая с группой крови Джона. 
Такой группой крови обладает 10\% населения. Группа крови Билла неизвестна. Какова вероятность, что Билл — убийца? $0.5 \cdot 0.1 \cdot 0.1 / 
(0.5 \cdot 0.1 \cdot 0.1 + 0.5 \cdot 0.1) = 1/11$.
\item (3б) В классе было два школьника и сколько-то школьниц. Равновероятно в класс заходит новый школьник или новая школьница. 
Затем все собравшиеся выбрали одного человека жеребьевкой и выгнали. Какова вероятность того, что вошел школьник, если выгнали школьника?
$(0.5 \cdot 3/(3+n) / (0.5 \cdot 3/(3+n) + 0.5 \cdot 2/(3+n)) = 3/5$.
\item (3б) Судья подкидывает монетку неограниченное количество раз. Если последовательность орёл-орёл-решка выпадает раньше
последовательности решка-орёл-орёл, то выигрывает Оля. Иначе выигрывает Рома. Какова вероятность того, что Рома выиграет?
$3/4$.
\end{enumerate}


\end{document}
