%!TEX TS-program = xelatex
%!TEX encoding = UTF-8 Unicode

\documentclass[a4paper, 20pt]{article}

%\newcommand{\LastChange}{Time-stamp: "2014-07-09 15:59:43 aga SeminarProblems.tex"}


% \usepackage[colorinlistoftodos]{todonotes}
\usepackage[colorlinks=true, allcolors=blue]{hyperref}

%\usepackage{euler}
%\usepackage{xltxtra} % loads: fixltx2e, metalogo, xunicode, fontspec

% \usepackage{multicol} % many columns
\usepackage{amsmath,amsfonts,amssymb,amsthm}
\usepackage{fullpage}
\usepackage{graphicx}
\usepackage{bm}
\usepackage{multicol} % текст в несколько колонок

\usepackage{marvosym} % значок мужского туалета

%\usepackage{enumerate}
\usepackage{textcomp} % text in formulas

%\usepackage{paralist}
\usepackage{enumitem} % more options for lists

\usepackage{tikz} % картинки
\usetikzlibrary{arrows.meta, quotes, angles} % tikz-прибамбас для рисовки стрелочек подлиннее

\usepackage[includehead, top=0.5cm, bottom=0.5cm, left=1cm, right=1cm]{geometry}


\usepackage{fontspec} % что-то про шрифты?
\usepackage{polyglossia} % русификация xelatex

\setmainlanguage{russian}
\setotherlanguages{english}

% download "Linux Libertine" fonts:
% http://www.linuxlibertine.org/index.php?id=91&L=1
\setmainfont[Scale=1.8]{Linux Libertine O} % or Helvetica, Arial, Cambria
% why do we need \newfontfamily:
% http://tex.stackexchange.com/questions/91507/
\newfontfamily{\cyrillicfonttt}[Scale=2.0]{Linux Libertine O}

\defaultfontfeatures{Mapping=tex-text}

\AddEnumerateCounter{\asbuk}{\russian@alph}{щ} % для списков с русскими буквами
\setlist[enumerate, 2]{label=\asbuk*),ref=\asbuk*}

%\setmainfont{Times New Roman}
%\setmainfont{Arial}
%\setmainfont{PT Sans}


%\setlength{\topmargin}{0in}
%\setlength{\headheight}{0cm}
%\setlength{\headsep}{0in}
%\setlength{\oddsidemargin}{-0.5in}
%\setlength{\evensidemargin}{-0.5in}
%\setlength{\textwidth}{7.5in}
%\setlength{\textheight}{9.0in}


% \newcommand{\staritem}{\refstepcounter{enumi}\item[\bf *\theenumi.]}

% \newcommand{\bsym}{\boldsymbol}


%\newcommand{\FigWidth}{0.3\columnwidth}



\newtheoremstyle{break}% name
  {}%         Space above, empty = `usual value'
  {10pt}%         Space below
  {}% Body font
  {}%         Indent amount (empty = no indent, \parindent = para indent)
  {\bfseries}% Thm head font
  {.}%        Punctuation after thm head
  {\newline}% Space after thm head: \newline = linebreak
  {}%         Thm head spec

\theoremstyle{break}
\newtheorem{problem}{Задаченька}[subsection]
\renewcommand{\theproblem}{\arabic{problem}}% Remove subsection from the counter representation


\begin{document}

\thispagestyle{empty}
%%%%%%%%%%%%%%%%%%%%%%%%%%%%%%
%%%%%%%%%%%%%%%%%%%%%%%%%%%%%%

Лекторские задаченьки :)

\vspace{20pt}

Борис Демешев

\vspace{20pt}

\begin{problem}
Два лекарства испытывали на мужчинах и женщинах. Каждый
человек принимал только одно лекарство. Общий процент людей,
почувствовавших улучшение, больше среди принимавших лекарство А.
Процент мужчин, почувствовавших улучшение, больше среди мужчин, принимавших лекарство В.
Процент женщин, почувствовавших улучшение, больше среди женщин, принимавших лекарство В.

Возможно ли такое? Если да, то как? Если нет, то почему?
\end{problem}



\begin{problem}
Маша пишет на бумажках два любых различных натуральных числа по своему выбору.
Одну бумажку она прячет в левую руку, а другую — в правую. Саша выбирает Машину
равновероятно наугад.
Маша показывает число, написанное на выбранной бумажке.
Саша высказывает свою догадку о том, открыл ли он большее из двух чисел или меньшее.
Если Саша угадал, то он получает тортик!

Укажите стратегию Саши, гарантирующую ему выигрыш с вероятностью строго более 50\%,
при любой стратегии Маши.
\end{problem}









\end{document}
