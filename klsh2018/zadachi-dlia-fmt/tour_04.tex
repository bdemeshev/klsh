%!TEX TS-program = xelatex
%!TEX encoding = UTF-8 Unicode

\documentclass[a4paper, 11pt]{article}

%\newcommand{\LastChange}{Time-stamp: "2014-07-09 15:59:43 aga SeminarProblems.tex"}


% \usepackage[colorinlistoftodos]{todonotes}
\usepackage[colorlinks=true, allcolors=blue]{hyperref}

%\usepackage{euler}
%\usepackage{xltxtra} % loads: fixltx2e, metalogo, xunicode, fontspec

% \usepackage{multicol} % many columns
\usepackage{amsmath,amsfonts,amssymb,amsthm}
\usepackage{fullpage}
\usepackage{graphicx}
\usepackage{bm}
\usepackage{multicol} % текст в несколько колонок

\usepackage{marvosym} % значок мужского туалета

%\usepackage{enumerate}
\usepackage{textcomp} % text in formulas

%\usepackage{paralist}
\usepackage{enumitem} % more options for lists

\usepackage{tikz} % картинки
\usetikzlibrary{arrows.meta, quotes, angles} % tikz-прибамбас для рисовки стрелочек подлиннее

\usepackage[includehead, top=0.5cm, bottom=0.5cm, left=1cm, right=1cm]{geometry}


\usepackage{fontspec} % что-то про шрифты?
\usepackage{polyglossia} % русификация xelatex

\setmainlanguage{russian}
\setotherlanguages{english}

% download "Linux Libertine" fonts:
% http://www.linuxlibertine.org/index.php?id=91&L=1
\setmainfont{Linux Libertine O} % or Helvetica, Arial, Cambria
% why do we need \newfontfamily:
% http://tex.stackexchange.com/questions/91507/
\newfontfamily{\cyrillicfonttt}{Linux Libertine O}

\defaultfontfeatures{Mapping=tex-text}

\AddEnumerateCounter{\asbuk}{\russian@alph}{щ} % для списков с русскими буквами
\setlist[enumerate, 2]{label=\asbuk*),ref=\asbuk*}

%\setmainfont{Times New Roman}
%\setmainfont{Arial}
%\setmainfont{PT Sans}


%\setlength{\topmargin}{0in}
%\setlength{\headheight}{0cm}
%\setlength{\headsep}{0in}
%\setlength{\oddsidemargin}{-0.5in}
%\setlength{\evensidemargin}{-0.5in}
%\setlength{\textwidth}{7.5in}
%\setlength{\textheight}{9.0in}


% \newcommand{\staritem}{\refstepcounter{enumi}\item[\bf *\theenumi.]}

% \newcommand{\bsym}{\boldsymbol}


%\newcommand{\FigWidth}{0.3\columnwidth}



\newtheoremstyle{break}% name
  {}%         Space above, empty = `usual value'
  {1pt}%         Space below
  {}% Body font
  {}%         Indent amount (empty = no indent, \parindent = para indent)
  {\bfseries}% Thm head font
  {.}%        Punctuation after thm head
  {\newline}% Space after thm head: \newline = linebreak
  {}%         Thm head spec

\theoremstyle{break}
\newtheorem{problem}{Задача}[subsection]
\renewcommand{\theproblem}{\arabic{problem}}% Remove subsection from the counter representation


\begin{document}

\thispagestyle{empty}
%%%%%%%%%%%%%%%%%%%%%%%%%%%%%%
%%%%%%%%%%%%%%%%%%%%%%%%%%%%%%
\subsection*{Четвёртый Тур}
%%%%%%%%%%%%%%%%%%%%%%%%%%%%%%

\begin{problem}
Основания трапеции равны $2$ и $10$. Построены вписанная и описанная окружности.
Чему равен радиус вписанной окружности?
\end{problem}

\begin{problem}
% \begin{multicols}{2}
Брусок массы $3m$ налетает со скоростью $v$ на два неподвижных бруска массами $m$, $2m$
и стену, расположенных как показано на рисунке. Бруски с массами $m$ и $2m$
сторонами обращёнными друг к другу смазаны клеем, остальные столкновения абсолютно
упругие.

\begin{minipage}{0.8\textwidth}
\begin{center}
\resizebox{8.0cm}{2cm}{
\begin{tikzpicture}
\draw (-8,0) -- (8,0);
\draw[line width = 2] (-6,0) -- (-6,2) -- (-4,2) node[midway, above]{\fontsize{40}{48} \selectfont $3m$} -- (-4,0) -- (-6,0);
\draw[line width = 2] (-2,0) -- (-2,2) -- (0,2) node[midway, above]{\fontsize{40}{48} \selectfont $m$} -- (0,0) -- (-2,0);
\draw[line width = 2] (2,0) -- (2,2) -- (4,2) node[midway, above]{\fontsize{40}{48} \selectfont $2m$} -- (4,0) -- (2,0);
\draw[line width = 7] (6,0) -- (6,5);
\draw[line width = 1, -{Latex[length=5mm, width=2mm]}] (-5,1) -- (-3,1) node[above]{\fontsize{40}{48} \selectfont $v$};
\draw[line width = 5, dashed] (0,2) -- (0,0);
\draw[line width = 5, dashed] (2,0) -- (2,2);
\end{tikzpicture}
}
\end{center}
\end{minipage}


Чему равна скорость бруска массы $3m$ после всех столкновений брусков друг с другом?

Дорогой друг, трением разрешаю тебе пренебречь! Твой главный судья $\heartsuit$.
% \end{multicols}
\end{problem}


\begin{problem}
Два числа назовём соседними, если они различаются только одной цифрой
в каком-то из разрядов. Например, 123 и 153 — соседние. Какое наибольшее количество
трёхзначных чисел можно выписать так, чтобы среди них не было соседних?
\end{problem}



\begin{problem}
\begin{multicols}{2}
Зондера Щукин и Лебедев тянут воз со скоростью $v$,
каждый в своём направлении. Верёвки от зондеров до воза нерастяжимые и имеют одинаковую длину.
Чему равна величина скорости воза в момент, когда угол между Щукиным,
возом и Лебедевым равен $\alpha$?

\begin{minipage}{0.4\textwidth}
\begin{center}
\begin{tikzpicture}
  \coordinate (A) at (-3,0);
  \coordinate (B) at (2,1);
  \coordinate (C) at (2,-1);
\filldraw (-3,0) circle (2pt) node[above]{Воз} -- (2, 1) circle (2pt) node[above]{Лебедев};
\draw[line width = 1, -{Latex[length=5mm, width=2mm]}] (2,1) -- (3,1.2) node[below]{$v$};
\filldraw  (-3,0)  -- (2, -1) circle (2pt) node[below]{Щукин};
\draw[line width = 1, -{Latex[length=5mm, width=2mm]}] (2,-1) -- (3,-1.2) node[above]{$v$};
\draw (-2, -0.2) arc (-30:19:0.5) node[right = 7, below]{$\alpha$};
\end{tikzpicture}
\end{center}
\end{minipage}
\end{multicols}

\end{problem}

\subsection*{Четвёртый Тур}
%%%%%%%%%%%%%%%%%%%%%%%%%%%%%%
\setcounter{problem}{0}

\begin{problem}
Основания трапеции равны $2$ и $10$. Построены вписанная и описанная окружности.
Чему равен радиус вписанной окружности?
\end{problem}

\begin{problem}
% \begin{multicols}{2}
Брусок массы $3m$ налетает со скоростью $v$ на два неподвижных бруска массами $m$, $2m$
и стену, расположенных как показано на рисунке. Бруски с массами $m$ и $2m$
сторонами обращёнными друг к другу смазаны клеем, остальные столкновения абсолютно
упругие.

\begin{minipage}{0.8\textwidth}
\begin{center}
\resizebox{8.0cm}{2cm}{
\begin{tikzpicture}
\draw (-8,0) -- (8,0);
\draw[line width = 2] (-6,0) -- (-6,2) -- (-4,2) node[midway, above]{\fontsize{40}{48} \selectfont $3m$} -- (-4,0) -- (-6,0);
\draw[line width = 2] (-2,0) -- (-2,2) -- (0,2) node[midway, above]{\fontsize{40}{48} \selectfont $m$} -- (0,0) -- (-2,0);
\draw[line width = 2] (2,0) -- (2,2) -- (4,2) node[midway, above]{\fontsize{40}{48} \selectfont $2m$} -- (4,0) -- (2,0);
\draw[line width = 7] (6,0) -- (6,5);
\draw[line width = 1, -{Latex[length=5mm, width=2mm]}] (-5,1) -- (-3,1) node[above]{\fontsize{40}{48} \selectfont $v$};
\draw[line width = 5, dashed] (0,2) -- (0,0);
\draw[line width = 5, dashed] (2,0) -- (2,2);
\end{tikzpicture}
}
\end{center}
\end{minipage}


Чему равна скорость бруска массы $3m$ после всех столкновений брусков друг с другом?

Дорогой друг, трением разрешаю тебе пренебречь! Твой главный судья $\heartsuit$.
% \end{multicols}
\end{problem}


\begin{problem}
Два числа назовём соседними, если они различаются только одной цифрой
в каком-то из разрядов. Например, 123 и 153 — соседние. Какое наибольшее количество
трёхзначных чисел можно выписать так, чтобы среди них не было соседних?
\end{problem}



\begin{problem}
\begin{multicols}{2}
Зондера Щукин и Лебедев тянут воз со скоростью $v$,
каждый в своём направлении. Верёвки от зондеров до воза нерастяжимые и имеют одинаковую длину.
Чему равна величина скорости воза в момент, когда угол между Щукиным,
возом и Лебедевым равен $\alpha$?

\begin{minipage}{0.4\textwidth}
\begin{center}
\begin{tikzpicture}
  \coordinate (A) at (-3,0);
  \coordinate (B) at (2,1);
  \coordinate (C) at (2,-1);
\filldraw (-3,0) circle (2pt) node[above]{Воз} -- (2, 1) circle (2pt) node[above]{Лебедев};
\draw[line width = 1, -{Latex[length=5mm, width=2mm]}] (2,1) -- (3,1.2) node[below]{$v$};
\filldraw  (-3,0)  -- (2, -1) circle (2pt) node[below]{Щукин};
\draw[line width = 1, -{Latex[length=5mm, width=2mm]}] (2,-1) -- (3,-1.2) node[above]{$v$};
\draw (-2, -0.2) arc (-30:19:0.5) node[right = 7, below]{$\alpha$};
\end{tikzpicture}
\end{center}
\end{minipage}
\end{multicols}

\end{problem}


\end{document}
