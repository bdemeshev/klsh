%!TEX TS-program = xelatex
%!TEX encoding = UTF-8 Unicode

\documentclass[a4paper, 12pt]{article}

%\newcommand{\LastChange}{Time-stamp: "2014-07-09 15:59:43 aga SeminarProblems.tex"}


% \usepackage[colorinlistoftodos]{todonotes}
\usepackage[colorlinks=true, allcolors=blue]{hyperref}

%\usepackage{euler}
%\usepackage{xltxtra} % loads: fixltx2e, metalogo, xunicode, fontspec

% \usepackage{multicol} % many columns
\usepackage{amsmath,amsfonts,amssymb,amsthm}
\usepackage{fullpage}
\usepackage{graphicx}
\usepackage{bm}
\usepackage{multicol} % текст в несколько колонок

\usepackage{marvosym} % значок мужского туалета

%\usepackage{enumerate}
\usepackage{textcomp} % text in formulas

%\usepackage{paralist}
\usepackage{enumitem} % more options for lists

\usepackage{tikz} % картинки
\usetikzlibrary{arrows.meta, quotes, angles} % tikz-прибамбас для рисовки стрелочек подлиннее

\usepackage[includehead, top=0.5cm, bottom=0.5cm, left=1cm, right=1cm]{geometry}


\usepackage{fontspec} % что-то про шрифты?
\usepackage{polyglossia} % русификация xelatex

\setmainlanguage{russian}
\setotherlanguages{english}

% download "Linux Libertine" fonts:
% http://www.linuxlibertine.org/index.php?id=91&L=1
\setmainfont{Linux Libertine O} % or Helvetica, Arial, Cambria
% why do we need \newfontfamily:
% http://tex.stackexchange.com/questions/91507/
\newfontfamily{\cyrillicfonttt}{Linux Libertine O}

\defaultfontfeatures{Mapping=tex-text}

\AddEnumerateCounter{\asbuk}{\russian@alph}{щ} % для списков с русскими буквами
\setlist[enumerate, 2]{label=\asbuk*),ref=\asbuk*}

%\setmainfont{Times New Roman}
%\setmainfont{Arial}
%\setmainfont{PT Sans}


%\setlength{\topmargin}{0in}
%\setlength{\headheight}{0cm}
%\setlength{\headsep}{0in}
%\setlength{\oddsidemargin}{-0.5in}
%\setlength{\evensidemargin}{-0.5in}
%\setlength{\textwidth}{7.5in}
%\setlength{\textheight}{9.0in}


% \newcommand{\staritem}{\refstepcounter{enumi}\item[\bf *\theenumi.]}

% \newcommand{\bsym}{\boldsymbol}


%\newcommand{\FigWidth}{0.3\columnwidth}



\newtheoremstyle{break}% name
  {}%         Space above, empty = `usual value'
  {1pt}%         Space below
  {}% Body font
  {}%         Indent amount (empty = no indent, \parindent = para indent)
  {\bfseries}% Thm head font
  {.}%        Punctuation after thm head
  {\newline}% Space after thm head: \newline = linebreak
  {}%         Thm head spec

\theoremstyle{break}
\newtheorem{problem}{Задача}[subsection]
\renewcommand{\theproblem}{\arabic{problem}}% Remove subsection from the counter representation


\begin{document}

\thispagestyle{empty}
%%%%%%%%%%%%%%%%%%%%%%%%%%%%%%
%%%%%%%%%%%%%%%%%%%%%%%%%%%%%%
\section*{Свалка}
Свалка проходит в последний день КЛШ и напоминает основной этап регулярного тура ФМТ.
Главное: команда борется не против одного соперника, а против {\bf хронометра}.
Каждая команда получает 4 задачи. Команда решает задачи так, чтобы судья
видел ход решения (а соперники не слышали!) и заявляет свой ответ на специальном бланке.
Судья фиксирует время заявки и число очков за решение: ноль, один или два.
Заявлять задачи на свалке можно в произвольном порядке.

В процессе тура мы обрабатываем данные:
\begin{enumerate}
\item По каждой из четырёх задач мы выстраиваем команды в список (всего четыре списка).
В вершине списка идут команды, которые решили задачу на два балла
(выше стоит та команда, которая решила быстрее), далее те, кто решили на один балл,
и затем команды, решившие на ноль баллов или не решившие. Лучше решить задачу
на два балла за 20 минут, чем на один балл за 1 минуту.
\item По каждому из четырёх списков команда получает число очков, равное числу команд,
решивших данную задачу хуже неё (т.е. находящихся ниже в списке).
\item Результаты команды по каждой задаче суммируются. Таким образом,
по сумме баллов за четыре задачи команда может набрать от $0$ до $76$ очков.
\item По этой сумме мы сортируем команды, и каждая команда получает столько {\bf финальных очков},
сколько команд выступило хуже неё (от 0 до 19).
\end{enumerate}

\section*{Финал и суперфинал}
Как вы помните, за основные туры и за свалку можно было набрать от $0$ до $19$ {\bf финальных очков}.
Сложим эти очки за основные туры и за свалку. По этой сумме финальных очков
мы определяем, какие команды сойдутся в финале и суперфинале: IV и III команды сразятся в финале,
а II и I команды будут бороться в суперфинале.

Суперфинал и финал похожи на обычные туры ФМТ за следующими исключениями:
\begin{enumerate}
\item Нет вольных стрелков.
\item Сложные задачи.
\item Судейская бригада {\bf случайным образом} определяет, кто из членов команды
будет докладывать решение своей задачи на обмене ударами. Таким образом,
решение обменной задачи должны знать {\bf все} участники.
\end{enumerate}

\end{document}
