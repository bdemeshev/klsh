\documentclass[10pt,russian,]{article}
\usepackage{lmodern}
\usepackage{amssymb,amsmath}
\usepackage{ifxetex,ifluatex}
\usepackage{fixltx2e} % provides \textsubscript
\ifnum 0\ifxetex 1\fi\ifluatex 1\fi=0 % if pdftex
  \usepackage[T1]{fontenc}
  \usepackage[utf8]{inputenc}
\else % if luatex or xelatex
  \ifxetex
    \usepackage{mathspec}
  \else
    \usepackage{fontspec}
  \fi
  \defaultfontfeatures{Ligatures=TeX,Scale=MatchLowercase}
    \setmainfont[]{Linux Libertine O}
\fi
% use upquote if available, for straight quotes in verbatim environments
\IfFileExists{upquote.sty}{\usepackage{upquote}}{}
% use microtype if available
\IfFileExists{microtype.sty}{%
\usepackage{microtype}
\UseMicrotypeSet[protrusion]{basicmath} % disable protrusion for tt fonts
}{}
\usepackage[left=2cm, right=2cm, top=2cm, bottom=2cm]{geometry}
\usepackage{hyperref}
\PassOptionsToPackage{usenames,dvipsnames}{color} % color is loaded by hyperref
\hypersetup{unicode=true,
            pdftitle={Игровой тур},
            pdfauthor={Памятка судьям},
            colorlinks=true,
            linkcolor=blue,
            citecolor=blue,
            urlcolor=blue,
            breaklinks=true}
\urlstyle{same}  % don't use monospace font for urls
\ifnum 0\ifxetex 1\fi\ifluatex 1\fi=0 % if pdftex
  \usepackage[shorthands=off,main=russian]{babel}
\else
  \usepackage{polyglossia}
  \setmainlanguage[]{russian}
\fi
\usepackage{graphicx,grffile}
\makeatletter
\def\maxwidth{\ifdim\Gin@nat@width>\linewidth\linewidth\else\Gin@nat@width\fi}
\def\maxheight{\ifdim\Gin@nat@height>\textheight\textheight\else\Gin@nat@height\fi}
\makeatother
% Scale images if necessary, so that they will not overflow the page
% margins by default, and it is still possible to overwrite the defaults
% using explicit options in \includegraphics[width, height, ...]{}
\setkeys{Gin}{width=\maxwidth,height=\maxheight,keepaspectratio}
\IfFileExists{parskip.sty}{%
\usepackage{parskip}
}{% else
\setlength{\parindent}{0pt}
\setlength{\parskip}{6pt plus 2pt minus 1pt}
}
\setlength{\emergencystretch}{3em}  % prevent overfull lines
\providecommand{\tightlist}{%
  \setlength{\itemsep}{0pt}\setlength{\parskip}{0pt}}
\setcounter{secnumdepth}{5}
% Redefines (sub)paragraphs to behave more like sections
\ifx\paragraph\undefined\else
\let\oldparagraph\paragraph
\renewcommand{\paragraph}[1]{\oldparagraph{#1}\mbox{}}
\fi
\ifx\subparagraph\undefined\else
\let\oldsubparagraph\subparagraph
\renewcommand{\subparagraph}[1]{\oldsubparagraph{#1}\mbox{}}
\fi

%%% Use protect on footnotes to avoid problems with footnotes in titles
\let\rmarkdownfootnote\footnote%
\def\footnote{\protect\rmarkdownfootnote}

%%% Change title format to be more compact
\usepackage{titling}

% Create subtitle command for use in maketitle
\newcommand{\subtitle}[1]{
  \posttitle{
    \begin{center}\large#1\end{center}
    }
}

\setlength{\droptitle}{-2em}

  \title{Игровой тур}
    \pretitle{\vspace{\droptitle}\centering\huge}
  \posttitle{\par}
    \author{Памятка судьям}
    \preauthor{\centering\large\emph}
  \postauthor{\par}
      \predate{\centering\large\emph}
  \postdate{\par}
    \date{Тур 5}

\newfontfamily{\cyrillicfonttt}{Linux Libertine O}
\newfontfamily{\cyrillicfont}{Linux Libertine O}
\newfontfamily{\cyrillicfontsf}{Linux Libertine O}

\begin{document}
\maketitle

\subsection{Регламент тура}\label{-}

\subsubsection{Первый этап, индивидуальный}\label{--}

\begin{itemize}
\tightlist
\item
  10 минут первый этап.
\item
  Игроки играют пара на пару на поле \(1 \times 5\).
\item
  Сажаем игроков команды друг напротив друга.
\item
  В каждой паре играющих (четыре пары) одну игру начинает игрок команды
  А, и одну игру --- команда Б.
\item
  Одна победа приносет команде 1 очко. Итого команда может получить от 0
  до 8 очков.
\item
  Ограничение 30 секунд на ход. Один судья следит за двумя парами,
  поэтому судейство здесь мягкое.
\end{itemize}

\subsubsection{Перерыв между этапами}\label{--}

\begin{itemize}
\tightlist
\item
  Перерыв на обсуждение стратегии 10 минут.
\item
  В течение первых 5 минут вольные стрелки могут передать письменно
  команде свои соображения по оптимальной стратегии
\end{itemize}

\subsubsection{Второй этап, командный}\label{--}

\begin{itemize}
\tightlist
\item
  Каждая команда выбирает капитана
\item
  Команды играют на поле \(1 \times 9\).
\item
  Играем две партии, в одной начинает команда А, в другой первый ход у
  команды Б
\item
  Ход делает только капитан
\item
  Ограничение 30 секунд на ход. Здесь следим строго.
\item
  Команда, набравшая больше баллов на первом этапе, получает право
  выбора, будет ли она сначала ходить первой, а потом второй, или
  наоборот
\item
  Одна победа приносит команде 4 очка
\end{itemize}

\subsection{Правила игры}\label{-}

\begin{itemize}
\tightlist
\item
  Игра ``Лужи на дороге'' или ``Домики-поддавки''
\item
  Поле состоит из полоски \(1 \times n\) клеточек
\item
  Игроки по очереди соединяют две соседние ещё не соединённые вершины.
\item
  Соединяем по горизонтали или вертикали, НЕ по диагонали.
\item
  Тот, кто закончил соединять ребра вокруг клетки считает клетку своей
  лужей, и обязательно делает дополнительный ход. Для удобства занятую
  клетку можно пометить буквой игрока.
\item
  Тот, кто собрал к концу игры больше всего луж проиграл.
\end{itemize}

\newpage

\subsection{Выигрышная стратегия}\label{-}

\begin{itemize}
\item
  Все ребра делятся на граничные (целиком лежит на границе) и
  внутренние.
\item
  Выигрышная стратегия есть у первого игрока
\item
  Первый ход --- выбираем любое внутреннее ребро кроме ребер, касающихся
  самого левого и самого правого квадратика.
\item
  Если все внутреннии ребра заняты, то делаем любой ход граничным
  ребром, кроме замыкания лужи.
\item
  Если не все внутренние ребра заняты, то смотрим на предыдущий ход
  второго игрока.
\item
  Если второй игрок только что выбрал внутренее ребро, то выбираем любое
  внутреннее ребро.
\item
  Если второй игрок только что поставил граничное ребро, которое
  касается РОВНО ОДНОГО внутреннего ребра, то ставим второе внутреннее
  ребро, касающееся только что поставленного вторым игроком граничного
  ребра. При этом может оказаться, что мы захватили лужу.
\item
  Иначе выбираем любое внутреннее ребро.
\end{itemize}


\end{document}
